% Разум-борец © 2023-2024 by that1357 is licensed under CC BY-NC-SA 3.0 

\documentclass[a4paper,12pt]{book}
\usepackage[T2A]{fontenc}
\usepackage[utf8]{inputenc}
\usepackage[russian,english]{babel}
\usepackage{tocloft}
\usepackage{sectsty}
\usepackage[symbol]{footmisc}

\let\cleardoublepage\clearpage
\chapterfont{\large\centering}
\renewcommand{\cftchapfont}{\normalfont}
\renewcommand{\cftchappagefont}{\normalfont}
\addto\captionsenglish{\renewcommand\contentsname{\begin{center}\largeОГЛАВЛЕНИЕ\end{center}}}
\addto\captionsenglish{\renewcommand\chaptername{Глава}}
\linespread{1.1}
\setlength\parindent{24pt}
\pagestyle{plain}
\interfootnotelinepenalty=10000

\begin{document}

\title{\textbf{\HugeРАЗУМ-БОРЕЦ\\
    \normalsize M I N D F I G H T E R}\vspace{40pt}}
\author{\LargeАнна Поупкесс\vspace{320pt}}
\date{\Large1987}

\maketitle
\tableofcontents

\chapter*{\HugeПосвящается Этельстану.}
\newpage
\begin{center}
\noindentРАЗУМ — БОРЕЦ\\
\end{center}
\par
\noindent(ЭТОТ ГОД МОГ БЫ СТАТЬ ГОДОМ НОВОГО БОЛЬШОГО ВЗРЫВА)\\
\par
% justify block
\tolerance 1414
\hbadness 1414
\emergencystretch 1.5em
\hfuzz 0.3pt
% ends here
В прошлом, словосочетание "Большой взрыв" использовалось как название теории о зарождении Вселенной и, следовательно, нашей планеты — Земли. Имея это в виду, мы не станем возражать и против его использования в противоположном смысле, говоря о её разрушении.\\
\par
О парапсихологии мало кому известно в подробностях, и зачастую это является причиной пренебрежительного отношения к ней, поскольку подавляющее большинство людей недоверчиво и агрессивно относятся к тому, чего они не понимают или не желают понимать. Парапсихологию принято обходить стороной, а те, кто участвует в подобных экспериментах, считаются чудаками или сумасшедшими. И предсказания, и предчувствия рассматриваются большинством как нечто из области фантазий и снов. «Ну и что?» - говорят эти люди. «Даже если сон сбылся, то это ничего не значит — всего лишь совпадение, случайность.»\\
\par
Возможно ли в будущем использовать парапсихологию на благо всего человечества? Возможно ли, что, как и большинство научных открытий, она будет использоваться нечистыми на руку политиками для достижения их разрушительных целей?\\
\par
В следующих строках рассматривается степень вероятности, с которой некоторая группа одарённых людей, обладающих исключительными способностями, могла бы влиять на будущее.\\
\par
Представьте себе весь людской род, идущий прямой дорогой к Т-образному перекрёстку, на котором у него есть только два варианта:
\par
1) Дверь ко злу, то есть полное разрушение мира и неизбежная смерть всего живого.
\par
2) Дверь к добру, то есть всеобщее процветание и жизнь.
Звучит слишком просто, не правда ли? Но так ли просто на самом деле выбрать правильное направление, ускоряясь под давлением обстоятельств, над которыми никто не властен?\\
\par
Во времена кризисов, прислушиваемся ли мы к тому, что говорят наши Лидеры — главы государств, и доверяем ли всему сказанному как Евангельской истине? Или же, подчиняясь своему внутреннему голосу и убеждениям, мы оказываем сопротивление властям, пытаясь таким образом привлечь внимание широкой общественности к происходящим событиям?\\
\par
Альберт Эйнштейн написал — «Мы никогда не должны ослаблять наших усилий по пробуждению в народах мира, и, в особенности, в их правительствах, осознания беспрецедентной катастрофы, которую они безо всякого сомнения навлекут на себя, если только не произойдёт коренного изменения в их отношении друг к другу, а также в их представлениях о будущем».

\chapter{СТРАННЫЕ ИДЕИ}
\noindent\par«Р{\scriptsizeОБИН, ТЫ УЖЕ РАССЛАБИЛСЯ? НЕ ТОРОПИСЬ. ТЫ ТОЧНО ЗНАЕШЬ, ЧТО НУЖНО} делать? Помни! Не важно, если на этот раз у тебя ничего не получится, профессор Шовен из Страсбургского университета уже показал нам, что это возможно.»
\par
Профессор Фергер всегда подбадривал своих студентов перед началом эксперимента.
\begin{quote}
"From harmony, from heavenly harmony,\\
The universal frame began:\\
When Nature underneath a heap\\
Of jarring atoms lay,\\
And could not heave her weary head,\\
The tuneful voice we heard from high:\\
Arise ye more than dead." \footnote[1]{Дословно: "Из гармонии, небесной гармонии родилась частица Вселенной. Когда Природа лежала под грудой дрожащих атомов и не могла поднять усталой головы, мы услышали мелодичный голос, звучащий с высоты: Вставайте, вы больше не мертвы."}
\end{quote}
\parРобин повторял эти слова снова и снова. Затем он стал расслаблять своё чувствительное тело, сосредоточившись на пламени свечи. Вскоре оно налилось свинцом, и он начал медленно закрывать глаза.
\par
Теперь Робин был расслаблен.  Он начал вдыхать на счёт "три". Он задержал дыхание, считая в уме до двенадцати. Досчитав, выдохнул. Эту дыхательную процедуру он повторил трижды.
\par
Перед тем как погрузиться в транс, Робин пробормотал:
\par
«Я готов…»\\
\par
С этого момента наступила полная тишина. Профессор и его студенты наблюдали за Робином через одностороннее зеркало.\\
\par
Робин безжизненно сидел на простеньком деревянном стуле лицом к светло-серой стене. Он был один в маленькой, тускло освещённой комнате.
\par
В течении нескольких первых минут неподвижная фигура сосредотачивала свои мысли на ускорении радиоактивного распада изотопа урана. Затем Робин сосредоточился на его замедлении. Наконец, на третьей минуте, он полностью избавился от всех своих мыслей.
\par
Каждую минуту автоматически выключался счётчик Гейгера — прибор измерял и записывал скорость радиоактивного распада.\\
\par
Робин вывел себя из транса. Он начал "облегчать" своё тело тем же способом, каким расслабил его. Он закончил тем, что открыл глаза, позволив им полностью открыться на счёт семь, и, наконец, повторил дыхательную процедуру.
\par
Он сидел молча, его лицо было бледным и не выражало никаких эмоций. Когда были проверены результаты, никто из присутствующих не мог поверить своим глазам:\\
\par
«Он это сделал!»\\
\par
Робину удалось управлять радиоактивным распадом изотопа урана, используя свои психические способности.
\par
Элисон вошла в комнату первой, чтобы поздравить его. Она легонько поцеловала его в лоб и похлопала по спине.
\par
«Молодец, Робин. Я знала, что ты это сделаешь!» - тихо сказала она.\\
\par
Он посмотрел на неё, улыбнулся и прошептал:
\par
«Спасибо.»
\par
Она жалела его. Робину было всего одиннадцать лет. Его родители погибли во время пожара в их доме, в Берсилдоне. Было удачей, что ему удалось спастись, однако причина пожара по-прежнему оставалась неизвестной.
\par
Четыре года он прожил в детском приюте. Затем, в возрасте девяти лет, его доставили сюда, в Саутгемптонский университет. Представители Органов Образования известили профессора Фергера об уникальных телепатических способностях, которыми обладал этот мальчик.
\par
Два года с Робином обращались как с морской свинкой. Он принимал участие во многих экспериментах, которые шли один за другим, и даже в таких, где опыты ставились над ним самим. Всем казалось, что его способности не имеют границ.\\
\par
Старшие студенты и профессор были искренне восхищены.
\par
Вперед вышел Гарри:
\par
«Хочешь куда-нибудь сходить, развлечься?»
\par
«Я хочу всего лишь пойти домой и поспать. Может быть, сходим сегодня вечером или завтра?»
\par
Его большие зелёные глаза безразлично смотрели вверх — лицо Робина редко выражало какие-либо эмоции. Оно было белого цвета, который частенько сравнивали с цветом мраморных статуй.
\par
«Конечно, я понимаю. Отправляйся домой и как следует отдохни. Мэтью отвезёт тебя.»\\
\par
Пока они ехали в машине, Робин спросил:
\par
«Что будешь делать днём?»
\par
«Не знаю. Полагаю, что Гарри, Элисон и я проведём несколько своих обычных экспериментов. Пока, Робин! Увидимся позже.»
\par
Робин выбрался из машины и медленно побрёл к входной двери. Он не стал утруждать себя ответом Мэтью, и даже не обернулся, чтобы помахать ему рукой на прощание.\\
\par
Для обратного пути в университет Мэтью выбрал маршрут подлиннее. Профессор нанял его шесть лет назад, когда ему было восемнадцать лет. Сейчас он начинал чувствовать скуку от той работы, которую приходилось выполнять.
\par
Казалось, что всеобщее внимание сосредоточено на одном лишь Робине. Большинство тестов и необычных экспериментов устраивалось специально для Робина. Всё, о чём когда-либо говорил профессор, было о Робине: «Робин сделал это. Робин сделал то. Ну разве он не чудесен? Вы когда-нибудь видели что-то похожее?»
\par
Но это была правда: в этом столетии никто не обладал такими же способностями, как у Робина.\\
\par
Мэтью был высоким и худощавым. На его лице отчётливо выступали скулы, а холодные голубые глаза были глубоко посажены в глазницах. Его короткие и прямые волосы каштанового цвета были небрежно откинуты назад, обнажая высокий лоб.
\par
Пока он вёл машину, его обида на Робина постепенно угасала.\\
\par
Наконец, Мэтью вернулся в лабораторию и обнаружил там склонившегося над столом профессора, проверяющего и перепроверяющего показания счётчика Гейгера.
\par
Профессор Фергер был пожилым мужчиной примерно шестидесяти лет. Он был руководителем Отделения Парапсихологии в университете. Верхняя часть его шевелюры поредела, но большая часть побелевшей копны волос всё ещё оставалась на своём месте. Своей внешностью он напоминал типичного "безумного профессора" из старинных фильмов ужасов тридцатых годов. Его густые брови торчали поверх круглых, в металлической оправе, очков.
\par
Пробормотав что-то тихим голосом, он вздохнул и заикнулся, прежде чем продолжить бубнить под свой нос:
\par
«Невероятно, невероят… Никогда не думал, что проживу достаточно долго, чтобы увидеть такую работу…»
\par
Он ненадолго прервался, а затем продолжил:
\par
«Невероя..»
\par
Не закончив фразы, он снова погрузился в глубокие размышления. Казалось, что они приводили его в экстаз.
\par
Но что дальше? Чем или как могли бы помочь ему эти сведения? Каким образом он мог управлять распадом радиоактивных изотопов, не пользуясь уникальными способностями разума Робина и не завися от них?
\par
Профессор стремился к тому, чтобы контролировать распад радиоактивных элементов при помощи химических или физических методов. Его целью было спасение человечества от "ядерного самоуничтожения" поскольку Третья мировая война казалось многим неизбежной.\\
\par
«Нет-нет, всё в порядке! У меня много работы. Уходи! Уходи! Сегодня ты мне не нужен!» - воскликнул он со своим сильным немецким акцентом.
\par
«Увидимся завтра. Присмотри за Робином — после сегодняшнего эксперимента он стал самым ценным человеком на свете. Крайне важно, чтобы то, что произошло здесь, держалось в строжайшем секрете!»
\par
Свободная от работы половина дня — это было редким, почти неслыханным событием. Элисон, Гарри и Мэтью по-быстрому собрали свои вещи и покинули лабораторию, оставив профессора бормотать наедине с собой.\\
\par
«Какие планы на сегодняшний день?» - спросил Гарри.
\par
Он взглянул на Элисон, которой на самом деле предназначался этот вопрос. Гарри всегда испытывал к ней нежные чувства.
\par
Они познакомились три года назад, в июне. Перед Гарри и Мэтью предстала стеснительная восемнадцатилетняя девушка. Был безоблачный день. Ярко светило солнце — его лучи, словно мерцающие языки пламени, протянулись поперёк небосвода и влекли за собой колесницу, на которой катались боги.
\par
Она робко стояла перед ними и, покраснев, смотрела себе под ноги. Пожав им руки, прошептала:
\par
«Здравствуйте. Я — Элисон Уайтли.»\\
\par
Она выглядела такой невинной, словно была одним из ангелов, влекущих за собой колесницу-солнце.
\par
С того самого дня она нравилась Гарри. Он притворялся перед собой и остальными, что его чувства к ней были только дружескими, однако глубоко внутри себя он чувствовал, что влюблён.\\
\par
Хотя Элисон уже исполнился двадцать один год, но она всё ещё оставалась ребёнком — ранимой и открытой для огромного мира. Она слишком легко доверяла окружавшим её людям и отчаянно нуждалась в их внимании и заботе.
\par
До сих пор у неё не было никаких отношений с мужчинами, и её это немного беспокоило. Детские психиатры сообщили монахиням в монастыре, что испуг покинул Элисон, но в глубине души, сейчас, она чувствовала его точно так же, как и тогда — шестнадцать лет назад.\\
\par
…в ту страшную, ужасную ночь пронзительно завывал ветер. В оконные стёкла стучал дождь вперемешку с градом. Оглушительно грохотал гром, словно кимвалы и барабаны — монотонный стук, предвещающий неизбежное зло.\\
\par
Раскаты грома болезненно  отдавались в её голове.\\
\par
Время от времени на небе вспыхивали молнии, освещая серое и, местами, чернильное пасмурное небо.
\par
Она била его! Её сердце бешено колотилось, когда она умоляла его оставить маму в покое. Но он не хотел. Он делал ей больно. Она умоляла его — но нет, он не слушал.
\par
Мама кричала ему:
\par
«Уходи и не возвращайся, пока не протрезвеешь…»\\
\par
"Он" — именно так Элисон называла отца после той ночи. У него был ужасный характер, а под воздействием алкоголя отец становился очень жестоким. Тем вечером он выпил гораздо больше обычного и проиграл все наличные деньги в казино "Силуэт". В этом он обвинил свою жену. Она была необычайно красивой и хрупкой.
\par
Той ночью он надругался над ней. Но тогда Элисон не до конца поняла, что на самом деле произошло между её матерью и отцом.
\par
На следующее утро, после пробуждения, Элисон обнаружила на кухонном полу полуголое, в изодранной одежде и покрытое синяками, охладевшее тело матери — та покончила с собой.
\par
Элисон помнила, как сидела рядом с её безжизненным телом. Затем она побежала к соседскому дому и дёргала открывшую ей дверь женщину за платье. Слёзы текли по щекам девочки, когда она в отчаянии звала её вслед за собой:
\par
«Пожалуйста, пожалуйста… Пойдёмте со мной. Помогите мне! Там моя мама…»
\par
Это было всё, что она успела произнести до того, как снова захлебнулась в рыданиях.
\par
Больше она никогда не видела своего отца. Именно тогда её и отправили в женский монастырь, в котором она жила и воспитывалась до восемнадцати лет.\\
\par
Сейчас Гарри, Мэтью, Элисон и Робин работали бок о бок, участвуя во множестве экспериментов, поскольку каждый из них обладал уникальными психическими способностями.\\
\par
Когда они подошли к скамейке на территории университета, Элисон уселась напротив Гарри. Мэтью остался стоять, оперевшись одной ногой на скамью. Он выбил пригоревший табак из калабаша, своего главного сокровища — типичной трубки Шерлока Холмса. Заново наполнив его, стал раскуривать.\\
\par
«Хочу вам кое-что почитать» - произнесла Элисон. Затем продолжила:
\par
«Я обнаружила эту статью вчера. Её автор пытается выяснить, является ли существование оборотней всего лишь мифом времён Средневековья, передаваемым из поколения в поколение как один из элементов фольклора.»\\
\par
«Что вызвало такой внезапный интерес к оборотням?» - заинтересованно спросил Гарри. В его глазах засверкало любопытство. Мэтью остался стоять — его локоть покоился на колене, а сам он наклонился вперёд и, глубоко задумавшись, курил свою трубку.
\par
«Считается, что оборотни были тесно связаны с людьми с особой формой психического расстройства, именуемого ликантропией. Человек, подвергшийся нападению волка, начинал считать сам себя волком».
\par
«О, да! Ва-а-у! И что тут такого? Мы и раньше знали об этом!» - насмешливо ответил ей Гарри.
\par
«Не смейся! Я ещё не закончила... Но то, что сделал сегодня Робин, а также то, на что способны мы сами — ну, всё это заставило меня задуматься. Если это правда, что в прошлом люди, находясь в припадке безумия, могли превращаться в оборотней, то… О, разве вам ещё не понятно, куда я клоню? А что насчёт нас? Мы не безумцы, но мы умеем управлять скрытыми силами наших разумов. Мы можем заставить их делать всё, что нам нужно, в любое удобное время.»
\par
Гарри и Мэтью ошарашенно уставились на неё.
\par
«Не понимаете, верно?»
\par
Элисон разволновалась и её дыхание начало немного сбиваться, но она продолжила:
\par
«Что, если нам попробовать это сделать? Представьте, что каждый из нас сможет превратиться в любое животное, когда захочет, а вовсе не под влиянием какого-то безумного психического состояния и, таким образом, сохранит полный контроль над своими поступками!»\\
\par
Её сердце уже выпрыгивало из груди, а в глубоких карих глазах поблёскивал яркий озорной огонёк.
\par
Наступила тишина — Гарри и Мэтью пытались понять, о чём она говорит. Оба были изумлены.
\par
Она что, сказала об этом серьёзно? Но разве такое возможно? Это были лишь некоторые вопросы из тех, что крутились сейчас в их головах.
\par
«Ну, что вы думаете?» - спросила она, тревожно вглядываясь в их удивлённые лица.
\par
«Да… М-м-м-м… Конечно… Но всё же, я полагаю, это немного другое...»
\par
Гарри запинался, пытаясь подобрать нужные слова. Он надеялся, что придёт внезапное озарение. Но идея Элисон застала его врасплох — он был ошеломлён, и не знал, что ответить.
\par
«Да, нам нужно попробовать! Это фантастическая идея, Элисон. В любом случае, мне наскучили наши обычные эксперименты. Итак, чего мы ждём?» - воодушевлённо воскликнул Мэтью. Так же как Элисон, он был взволнован и полон энтузиазма. Убрав ногу со скамейки, он повторил:
\par
«Ну, чего мы ждём? Вставайте! Мы идём в библиотеку.»
\par
«Нет необходимости. Я уже была в библиотеке и узнала всё, что нужно.» - ответила она, поднимаясь со скамьи.
\par
Гарри не мог произнести ни слова. Обычно именно он придумывал идеи и планы для того, что они собирались делать. Он смотрел на Элисон и Мэтью, сбитый с толку и потрясённый.
\par
«Вы оба говорите серьёзно? Я не уверен, что каждый из вас хорошо понимает, чего следует ожидать от самого себя. В любом случае, я считаю, что эта идея слишком опасна, и, возможно, нам следует о ней забыть.»
\par
Он сделал паузу, после которой сдал все свои позиции:
\par
«Ну ладно. Вам вовсе не обязательно выглядеть такими унылыми и разочарованными. Я всегда поддерживал вас обоих и ваши прежние проекты, поэтому вы можете рассчитывать на меня и сейчас. В таком случае — вперёд! Как уже дважды сказал Мэтью — чего мы ждём? Начинаем? Я, должно быть, свихнулся, раз стал участвовать в этой затее!»
\par
Гарри вздохнул, встал со скамейки, отряхнул свою куртку и направился к красной Сиерре, принадлежавшей Мэтью — автомобиль был припаркован в десяти минутах ходьбы от входа в университет.
\par
Мэтью и Элисон быстро подбежали к Гарри. Элисон была счастлива. Наконец-то, её идея может оказаться полезной, и у неё появится возможность проявить себя! До сих пор все только баловали её и никогда не ждали конструктивных или творческих идей с её стороны. Для них она являлась частью окружения, "тётушкой-агонией" — человеком, к которому могли обращаться Гарри, Мэтью и, особенно, Робин, когда у них возникали какие-либо проблемы. Она всегда находила для них сочувственный и деликатный совет.
\par
Когда они добрались до дома Элисон, она приготовила каждому по чашке чая. Затем начала читать для них свой конспект об оборотнях:
\par
«Во-первых, оборотни тесно связаны с вампирами. Истории о них известны во всём Старом Свете — от Португалии до берегов Тихого океана, а также от внутренних районов Африки до мыса Нордкап.
\par
Там, где не водятся волки, мы находим истории об оборотнях-тиграх и оборотнях-леопардах.
\par
В древнем скандинавском фольклоре упоминаются люди со сросшимися над переносицей бровями, обладающие, по мнению очевидцев, сверхъестественными способностями. В континентальной Европе таких людей подозревали в том, что они — оборотни.
\par
Я обнаружила кое-что ещё,  и мне кажется, что это может оказаться полезным: ведьма становится беспомощной, если кому-то удаётся вытянуть из её тела хотя бы каплю крови. Превращённому человеку — например, оборотню — можно вернуть его прежний облик тем же самым радикальным способом.
\par
Это основные моменты и руководящие принципы, которые, как я считаю, нам следует принимать во внимание и следовать им, если мы хотим добиться успеха. Ну как, вы будете комментировать?\\
\par
Мэтью снова начал курить трубку. Он откинул волосы назад и нахмурился, обдумывая представленные только что сведения.\\
\par
Внезапно Гарри поднялся на ноги и воскликнул:
\par
«Блестяще! Абсолютно превосходно! Элисон, ты — гений! Если всё получится, как задумано, то нас уже никто не сможет остановить!»
\par
На сияющем лице Элисон отразились удовольствие и восторг. Покраснев, она спросила:
\par
«Когда приступаем? Мы станем сообщать профессору о нашем новом проекте?»
\par
Она помолчала, а затем добавила:
\par
«Мне кажется, что будет лучше держать его в тайне.»
\par
Гарри одобрительно кивнул и спросил:
\par
«А что насчёт Робина? Он сможет участвовать в этом приключении вместе с нами?»
\par
«Нет. Я думаю, что он ещё слишком молод и неопытен. Как ты уже сказал, мы и сами недостаточно осведомлены.» - уверенно ответила Элисон. Наконец-то ей улыбнулась удача.
\par
«Правильно, я согласен. Кстати, я обещал, что мы ненадолго заглянем к нему этим вечером.» — добавил Мэтью.
\par
«Не возражаю» — последовал ответ Элисон.\\
\par
Она убрала свои записки, а Гарри отнёс чашки на кухню, прежде чем они собрались уходить.

\chapter{СКВОЗЬ ТЬМУ}
\noindent\parС{\scriptsizeТОЯЛ ПРЕКРАСНЫЙ ВЕСЕННИЙ ДЕНЬ. СВЕТИЛО СОЛНЦЕ. НА НЕБЕ НЕ БЫЛО ВИДНО НИ ЕДИНОГО} облачка. Нарциссы в парке слегка покачивались на ветру, словно волны в море жёлтого цвета — их яркие, жёлто-охристые цветки склонялись к солнцу, как на богослужении. Ветер доносил запах свежескошенной травы.\\
\par
Гарри проснулся от головной боли. Накануне, покинув остальных, он случайно встретил в кинотеатре своего старого приятеля. До раннего утра они выпивали и веселились, вспоминая прежние студенческие годы. На двоих они выпили бутылку вина, бренди и немного шотландского виски, оставшегося с Рождества.
\par
Он стоял возле окна и глубоко вдыхал свежий воздух, прикрыв глаза рукой от яркого, как ему казалось, солнечного света.
\par
Он снова поклялся, что больше никогда не станет пить так много, но это был его обычный ритуал, проводимый каждым утром после очередной разгульной вечеринки. Хотя он долго оставался в трезвом уме, падение наступило вскоре после того, как он начал смешивать напитки.
\par
Внешний вид Гарри был таким же ужасным, как и его самочуствие: небритый, с налитыми кровью глазами. Он подловил себя на том, что щурится, оглядывая свою неубранную комнату.
\par
После ухода от остальных он не мог припомнить точную последовательность дальнейших событий.
\par
Мэтью и Робин жили в одном доме на Девоншир-роуд, в поместье Бедфорд-плейс. Элисон жила с Гарри на Оксфорд-стрит, на другом конце города.\\
\par
«Гарри, ты уже собрался? Помнишь, ты сказал, что сегодня утром отвезёшь Робина в университет. Гарри, с тобой точно всё в порядке? Скажи хоть слово!» - прокричала Элисон, постучав в дверь его спальни.
\par
«Ты можешь войти, Эли.»
\par
Войдя, она увидела неубранную комнату и "несчастный" вид бедного Гарри.
\par
«Да, я слышала, что ты вернулся домой в три безбожных часа утра. Что, чёрт возьми, с тобой случилось? Сегодня ты выглядишь весьма не впечатляющим образцом существа человеческого рода. Я бы даже сказала, что на самом деле ты выглядишь хуже, чем смерть на огне.»
\par
«Я знаю, знаю… Тебе не обязательно продолжать. Я встретил старого, ещё со студенческих времён, друга. Ты должна понимать, как случаются такие вещи… просто у нас их было слишком много, вот и всё…»
\par
«Послушай, Элисон. Ты не могла бы оказать мне услугу и забрать Робина вместо меня? Я не могу даже подумать о встрече с профессором этим утром.»
\par
Он прервался и печально посмотрел на неё, словно ребёнок, который знает, что совершил проступок, и слишком напуган, чтобы нести за него наказание. Затем, нежным голосом, произнёс:
\par
«Пожалуйста?»
\par
Она кивнула и оставила Гарри, с плеч которого свалился тяжкий груз, искать в ванной парацетамол.\\
\par
Подойдя к дому Робина, Элисон увидела мальчика в окне верхнего этажа.
\par
«Доброе утро, Робин. Ты уже собрался? Надеюсь, ты позавтракал?»
\par
Робин вернулся в спальню, отвечая по пути на вопрос Элисон:
\par
«Нет, я ничего не ел. Поднимайся, я буду готов минут через пять.»
\par
Он уселся на пол и продолжил чтение — тем летом он готовился к экзаменам.
\par
«Тебе нужна помощь? Что ты рассматриваешь?» - спросила Элисон, заглядывая через его плечо.
\par
«Всё в порядке. Я просто положу эти книги в сумку, и мы пойдём.»
\par
«В чём дело, Робин? Последние несколько дней ты ведёшь себя очень странно…»
\par
Она сделала паузу, а затем продолжила:
\par
«Прошлым вечером ты был не слишком разговорчив. Давай, Робин, ты же знаешь, что можешь довериться мне.»
\par
Он посмотрел на пол, в отчаянии заламывая свои худые, бледные руки. Затем он заговорил, но не обычным спокойным тоном — нет, в его холодных зелёных глазах бушевал огонь мести:
\par
«Здесь нет никакой ошибки. Я просто… просто чувствую… ох, я не знаю. И вообще, какое тебе дело? Неужели я никогда не смогу побыть один? Кто-то из вас всё время мешает тому, что я делаю! Я не могу даже чихнуть или сходить в туалет без идущей по пятам инквизиции!»
\par
«Я не имела в виду…»
\par
«Да! Вот оно! Это именно то, что имею в виду я. Все просто хотят быть добренькими и заботиться обо мне, вот и стараются. Ах, бедный Робин! Вы знаете, он — сирота! Не правда ли, он милый и обаятельный? Погладьте его по головке. Никто из вас никогда не воспринимал меня всерьёз. Ради Бога, я уже почти взрослый!»
\par
Он сорвался на крик. На его глаза навернулись слёзы.\\
\par
«Хорошо, я подожду внизу, пока ты будешь собирать сумку для учёбы» - спокойно ответила Элисон и вышла из комнаты, находясь в состоянии потрясения. Ещё никогда на неё не кричали вот так, как сейчас. Крик исходил от Робина? Что ж, он удивил её сильнее, чем когда-либо прежде. Она не могла понять причин, вызвавших этот внезапный срыв.\\
\par
Пока они ехали в университет, Робин и Элисон не обменялись ни словом. В конце пути Элисон прервала ледяное молчание:
\par
«Пообедаешь с нами? Если не против, мы встретим тебя в университетской столовой около часа дня.»
\par
Она завела машину, собираясь уехать.
\par
«Да, спасибо.»
\par
В интонации его ответа всё ещё чувствовалось раздражение.\\
\par
Управляя автомобилем и слушая по радио классическую музыку, Вагнера, Элисон внезапно стала дрожать. Она не могла не заметить сковавшего её холода и вряд ли смогла бы описать словами свои ощущения. До сих пор с ней не случалось ничего похожего и это её пугало. Она как будто почувствовала, что с ней должно произойти что-то плохое — возможно, это было дурным знамением. Без видимых причин её глаза застлала мутная пелена, а по щекам потекли слёзы. Впервые за много лет она снова почувствовала себя ранимой и беззащитной. Ей захотелось остановить машину и убежать…
\par
Но куда? От чего?\\
\par
Когда она вернулась домой, её всю трясло от страха. Гарри открыл ей дверь:
\par
«Элисон, что случилось? Эли, проходи, присядь.»
\par
Он взял её за руку и повёл к дивану.
\par
«Почему ты дрожишь? Посмотри на меня и скажи, что случилось. Ты попала в аварию? Просто успокойся — сделай несколько глубоких вдохов.»
\par
Гарри разволновался, не зная, что предпринять. Он крепко сжимал её запястья. Выдержав паузу, он внимательно рассмотрел её в надежде найти хотя бы какой-то ключ к пониманию, что случилось с милым его сердцу другом. И продолжил выяснять:
\par
«Тебе плохо? Ты выглядишь так, как будто увидела привидение.»
\par
«О, Гарри! Пожалуйста, обними меня покрепче и скажи, что я сплю. Мне было так страшно! Пожалуйста, не оставляй меня одну — ни сейчас, ни в другое время.»
\par
«Она села на диван рядом с Гарри, и его руки обвили её — как она и просила. Всё ещё белая, словно мел, она находилась на грани истерики.»
\par
«Не волнуйся. Я не оставлю тебя. Ну же, перестань плакать и успокойся. Сейчас ты в полной безопасности.»
\par
Гарри произносил эти слова и легонько целовал макушку девушки, изо всех сил стараясь её утешить. Он продолжал уверять её в том, что нечто, так сильно её расстроившее, не сможет дотянуться до неё, пока она находится здесь, внутри этих стен.
\par
«Всё в порядке. А теперь расскажи мне, что произошло. Кто испугал тебя? Я должен знать, Элисон!» - сказал Гарри строго.
\par
«Никто. То есть мне некого винить. Просто… ну, ох… Я не знаю, с чего начать.»
\par
Она была сбита с толку и понять её было очень трудно, но всё же продолжила:
\par
«Я так и не поняла, что случилось. Я ехала обратно и внезапно в машине упала температура. Меня охватил озноб. Затем в салоне появились и стали перемещаться расплывчатые чёрные тени — из-за них я несколько раз чуть не разбилась… Ох, Гарри, это было ужасно. На меня навалилось чувство невыносимого горя, словно мрачная серо-чёрная туча. Я как будто видела тьму, которая собирается в одну могучую бурю — она не принесёт мне ничего, кроме несчастья. Я была в ужасе.»
\par
«Я чувствую, что это был дурной знак. И я не сомневаюсь в том, что скоро произойдёт что-то плохое.»
\par
«Эли, ничего не произойдёт. Никто и никогда не хотел тебя обижать. Послушай, что я скажу: сегодня вечером я никуда не пойду; я буду рядом с тобой весь день и всю ночь. Ну что, теперь тебе стало легче?»\\
\par
\parГарри не знал, как отнестись к загадочному утреннему происшествию, о котором поведала ему Элисон, и стоило ли воспринимать всерьёз слова столь впечатлительной девушки. Единственное, что он мог сделать — просто поддержать и утешить её. Он был уверен в одном — ему нельзя оставлять её в одиночестве, пока она находится в таком мрачном переменчивом настроении. Элисон выглядела ненормально подавленной, и Гарри не представлял, как она будет справляться с ожидаемой бедой.
\par
«Спасибо, Гарри. Теперь я чувствую себя немного лучше. Ты, наверное, считаешь меня чокнутой. Я понимаю — сказанное прозвучало невероятно, но…»
\par
Гарри прервал её:
\par
«Нет, я верю тебе. Прежде мне не доводилось видеть никого, кто был испуган точно так же, как ты двадцать минут назад.»
\par
«Ещё раз спасибо.»
\par
Теснее прижавшись к Гарри, она положила голову на его руки и уснула. Он крепко обнял её и не шевелился, пока не прозвучал дверной звонок. Элисон сразу же проснулась.
\par
«Прости, Гарри. Я не думала, что усну. Сколько времени я была в таком состоянии?» - спросила Элисон, протерев глаза и вытягивая руки.
\par
«Не волнуйся, всего один час.»
\par
«Я открою...» - сказала она, вставая.
\par
«Наверное, это Мэтью. Чёрт возьми, мы опоздали! Мы должны были встретиться с Робином в столовой в час дня. Не знаю, что с ним произошло, но что бы это ни было, сегодня утром, когда я забирала его из дома, он безо всякой видимой причины закатил мне истерику. Его что-то беспокоит, но он не сказал мне, что именно. Он поставил ментальный блок и я не смогла получить никакой информации из тех мыслей, что неслись через его голову. Должно быть, происходит что-то довольно серьёзное. Будет лучше, если мы станем внимательнее следить за ним. Ладно?»
\par
Она закончила свою речь и пошла открывать дверь.\\
\par
«Здравствуйте, я могу вам чем-то помочь?»
\par
На пороге стоял мужчина приблизительно сорока лет — неподвижно, его руки покоились в карманах.
\par
Незнакомец не ответил. Его взгляд был направлен прямо сквозь неё, словно горячий нож, с лёгкостью разрезающий кусок холодного масла.
\par
«Вы — один из приятелей Гарри?» - спросила Элисон.
\par
«Элисон, это ты?» - с оптимизмом спросил неизвестный. Он вытащил руки из карманов и поманил её к себе.
\par
«Элисон Уайтли, не так ли? Ты не помнишь меня? И даже не пригласишь войти?»
\par
Он вошёл сам и, осматривая по пути холл, направился в гостиную.
\par
Элисон выглядела озадаченной. Кем был этот человек, который буквально ворвался внутрь, даже не представившись?
\par
Она быстро последовала за ним. Гарри удивлённо взглянул на свою соседку, с которой они делили квартиру, словно спрашивая, кто этот человек. В ответ Элисон лишь пожала плечами. Внезапно она побледнела.
\par
Это был он! Именно его приход сюда заставил её так испугаться этим утром. Он был злом!
\par
«Кто вы? Что вам нужно от меня...?»
\par
На мгновение она умолкла, а затем:
\par
«Мне всё равно. Пожалуйста, просто уйди.»
\par
Она почувствовала слабость, как будто он вытягивал из неё жизнь. Его острый взгляд пронзил её, когда он начал с ней говорить. Его голос был гладким, как чёрный бархат, а слова лились словно вода, лениво скользящая по по округлым валунам в ручье.
\par
«Разве ты не помнишь и не узнаёшь меня после всех этих лет? Я твой отец! Прошло около шестнадцати лет с тех пор, когда я видел тебя в последний раз. Мне потребовалось много времени, чтобы отыскать тебя. Ты не собираешься поздороваться со мной?»
\par
Он протянул к ней жуткие длинные руки, предполагая заключить в объятия свою давно потерянную дочь.
\par
«Ты! Ты мой отец? Как ты вообще мог подумать о том, чтобы прийти сюда после того, что ты сделал с мамой? Неужели у тебя нет чувства порядочности? Уходи! Уходи и не вздумай возвращаться. Я не хочу тебя видеть!» - мстительно сказала она, стараясь не повышать голос.
\par
На её глаза навернулись слёзы, когда он вспомнила события той кошмарной ночи. Её охватило несчастье. Она почувствовала внутри себя воспаление и боль — её словно предали так называемые силы свыше. Что она сделала неправильно, раз это «нечто» снова вернулось в её жизнь? Он не мог принести с собой ничего, кроме неприятностей.
\par
Этот злодей, называемый человеком, возвращается по истечении шестнадцати лет — и ради чего? Что он мог доказать, появившись вот так — ниоткуда? За всё это время он ни разу не проявил ни малейшего интереса к её судьбе.
\par
Элисон в спешке покинула комнату и побежал наверх. Гарри не дал мистеру Уайтли последовать вслед за ней:
\par
«Вы слышали её! Убирайтесь!»
\par
Гарри был полон решимости — одна его рука преграждала путь по лестнице, а другая указывала на входную дверь.
\par
«Но я всего лишь хотел…»
\par
«Послушайте, просто уходите! На сегодня вы причинили уже достаточно беспокойств.»
\par
Незваный гость ушёл — невозмутимо и без признаков сожаления о том несчастье, которое навлёк так легко.
\par
Гарри побежал в спальню к Элисон.
\par
«Эли, пожалуйста, открой эту дверь. Он уже ушёл. Не волнуйся, я не пущу его обратно.»
\par
Гарри выдержал паузу, однако ответа не последовало. Тишина была оглушительной. Гарри больше не мог её терпеть и снова попытался открыть дверь. Она была заперта.
\par
«Эли, пожалуйста? Я не могу помочь тебе, пока ты держишь себя там взаперти, словно пленника. В любом случае, разве у тебя нет для меня каких-нибудь объяснений? Ты не рассказала мне правду о своём прошлом, не так ли? Пребывание за дверью не решит проблему и не заставит её исчезнуть.»
\par
Он отказался от мягкого подхода, пытаясь убедить Элисон выйти из спальни — вместо этого Гарри говорил строгим тоном. Он давал ей ясно понять, кто тут главный, и что в конце концов ей придётся обратить на него внимание — другого выбора у неё не было. Он больше не говорил, но выжидал и прислушивался к тому, что она ответит.
\par
Дверь медленно отворилась, и из-за неё осторожно выглянуло робкое лицо девушки, проверяющей, не обманул ли Гарри. Да, "злодей" ушёл.
\par
Элисон бросилась в его объятия — такие крепкие. Она расплакалась, но плакала больше от радости и облегчения, чем от страха. Гарри ласкал её, нежно поглаживая шею, и пытался успокоить.
\par
«Эли, милая, он не сможет сделать ничего такого, что причинит тебе боль.»
\par
Оттолкнув его от себя и удерживая на расстоянии вытянутой руки, она отвергла его слова:
\par
«Ты же ничего не знаешь о нём! Как же ты можешь вот так самоуверенно стоять здесь и говорить всё это? Ты не можешь даже представить, на что он способен. Тебя там не было, когда я видела его в последний раз! Ты не имеешь ни малейшего представления о том, каким пыткам он подверг меня и мою мать. Так что не смей говорить мне о том, что он не сможет мне навредить! Когда он зашёл сюда, ужасные кошмары и воспоминания, которые, как я думала, были наконец-то похоронены, нахлынули снова, в одно мгновение — как только я узнала его!»
\par
«Мне очень жаль, правда. У меня сложилось впечатление, что оба твоих родителя погибли в автокатастрофе. Почему ты солгала нам? Что случилось, Элисон? Я не могу помочь или попытаться дать совет, если не знаю правды.»
\par
«Полагаю, ты прав. Возможно, мне следовало рассказать тебе правду о моих родителях, когда я только приехала сюда. Думаю, я рассказала вам всю эту чепуху о том, что мои родители попали в аварию потому что… ну, это казалось мне лёгким выходом. Я рассказала эту историю стольким людям, что кажется, начала верить в неё сама — вероятно, принимая желаемое за действительное, я старалась забыть как можно больше из своего прошлого. Вы должны попытаться понять, что я была маленькой девочкой — в то время мне было всего пять или шесть лет.»
\par
Немного выждав, она продолжила:
\par
«На улице лил дождь, была гроза. Я помню, как кричала и умоляла этого… монстра оставить маму в покое. Он вернулся с работы — снова пьяный. Он отвешивал ей пощёчины и бил. Повалив её на пол, сорвал с неё блузку и полез с поцелуями. Можешь догадаться, что последовало вслед за этим, я уверена, что мне не нужно вдаваться в подробности, как он её насиловал. Это было ужасно, отвратительно! Короче говоря… потом он ушёл, а мама убила себя. Я нашла её охладевшее тело, полуголое и избитое, лежащим на полу в кухне.»
\par
«Господи, у меня кружится голова, я так растеряна. Обними меня! Держи меня крепче, пожалуйста. Прошу, не оставляй меня одну пока я не буду уверена, что он ушёл навсегда. Тогда я поклялась, что отомщу ему за маму.»
\par
Она прервалась и утёрла глаза. Её слёзы глубоко ранили Гарри. Она посмотрела на него, а затем положила голову ему на плечо. Он целовал её мягкие локоны, едва прикасаясь к ним.
\par
Гарри отвёл её на кухню внизу, усадил на стул и приготовил им обоим по чашке дымящегося горячего кофе.
\par
«Элисон, ты хочешь, чтобы Мэтью и Робин знали о том, что произошло сегодня? Я просто подумал, что будет к лучшему, если они узнают правду.»
\par
«Я в сомнениях. Честно говоря, я не знаю, что делать. Как я уже сказала, я в ужасном замешательстве и не могу обдумать всё за пару минут. Я была уверена в том, что моя жизнь наконец-то сложилась идеально, но тут внезапно случается это — после стольких лет появляется мой отец. Чего он может хотеть от меня? Хорошо, если ты думаешь, что это к лучшему, расскажи им. Только, пожалуйста, без подробностей.»
\par
«Эли, не обижайся на мой следующий вопрос — тебе вовсе не обязательно отвечать на него. Твой отец… Он…»
\par
Гарри прокашлялся, прежде чем продолжить:
\par
«Ну… Понимаешь? М-м-м… ну… каким-то образом прикасался к тебе? Ты понимаешь, о чём я.»
\par
Её лицо побледнело опять, а навязчивые воспоминания стали более яркими. Она кивнула, но не сказала ни слова. Гарри оставил в покое эту неудобную для разговора тему и стал предлагать другие, никак не связанные с её нынешними неприятностями.\\
\par
	Наступила глубокая тишина, во время которой они просто смотрели друг на друга, не зная, как подойти к деликатной теме:
\par
«У тебя был парень?»
\par
«Нет, я никогда ни с кем не встречалась» - ответила она неохотно.
\par
«Это из-за того, что произошло в прошлом? Если сейчас тебе не хочется говорить об этом, то так и скажи. Я не стану возражать, но мне не нравится видеть тебя такой расстроенной. Элисон, ты ведь знаешь, что я люблю тебя и не хочу торопить. Я понимаю, как все эти события должны были повлиять на твои эмоции — очевидно, они произвели на тебя какое-то очень сильное впечатление.»
\par
Элисон была в замешательстве. Она покраснела и стала глядеть на свои руки, нервно выкручивая их над столом. Сердце её колотилось, она не могла смотреть прямо в блестящие глаза Гарри. Это было словно сон наяву. Она всегда думала, что он ухаживал за ней как друг, а не как поклонник.
\par
«Гарри, ты действительно имеешь в виду именно это?»
\par
Она посмотрела на него, вглядевшись в его серо-голубые глаза.
\par
«Просто я никогда не думала, что ты влюбишься в кого-то вроде меня, особенно после того, как увидела некоторых девушек, с которыми у тебя были связи.»
\par
Гарри встал, подошёл к Элисон и взял её за руки.
\par
«Можно мне поцеловать тебя?» - спросил он, внимательно глядя в её глаза, чтобы увидеть реакцию — ему не хотелось быть назойливым.
\par
Она выглядела испуганной и неуверенной.
\par
«Всё в порядке, Элисон. С моей стороны было неосмотрительно спрашивать тебя об этом. Мне бы следовало проявить больше здравомыслия. Пожалуйста, не обращай на это никакого внимания и забудь о том, что я спросил. Я знаю, что выбрал не лучшее время, чтобы рассказать тебе о своих чувствах — ясно, что у тебя на уме есть кое-что поважнее.»
\par
«Спасибо за то, что понял. Только не думай, что я пытаюсь тебя оттолкнуть, но… вместе с появлением моего отца по прошествии бесчисленных лет, всё это определённо меня удивило. Вновь нахлынули плохие воспоминания… а ведь я считала, что они хорошо и по-настоящему похоронены.»
\par
«Господи! Я только что вспомнила, что мы должны были встретиться с Робином и Мэтью за обедом в университете. Который час? Мы уже опаздываем. Машину лучше вести тебе.»\\
\par
К тому времени, когда Элисон и Гарри входили в столовую университета, Мэтью и Робин уже закончили обедать и нетерпеливо смотрели на часы.
\par
«Почему вы так задержались? Мы ждали вас целую вечность» - спросил Мэтью, глядя на Гарри. Робин сидел молча, мрачно поглядывая на остальных. Гарри обернулся к Элисон и тихонько спросил:
\par
«Расскажем им о твоём знамении и неожиданном посетителе?»
\par
«Думаю, да. Вероятно, рано или поздно они узнают обо всём сами. Ты можешь им рассказать.»
\par
Гарри поведал им о странной последовательности событий этого утра и о печальных эпизодах из прошлого Элисон. Никто не пытался его перебивать. Они просто сидели молча, сочувственно вслушиваясь в каждое слово Гарри. Время от времени Элисон вытирала слёзы со своих остекленевших глаз.
\par
Робин пытался казаться безразличным к тяжёлому положению Элисон, но на самом деле ему хотелось крепко обнять её и извиниться за то, как он вёл себя и разговаривал с ней этим утром.
\par
«О, кстати — профессор не хочет, чтобы мы возвращались сегодня днём в лабораторию» - сказал Мэтью, пытаясь переменить тему для разговора, потому что почувствовал дискомфорт. Он хотел сказать ей, как ему жаль, но знал, что Элисон не любит, когда люди жалеют её и сочувствуют напоказ.
\par
«До свидания! Увидимся позже, Мэтью. Если сможешь, забери меня у главного входа в пять часов. Пока!»
\par
Собрав свои книги, Робин ушёл.\\
\par
Вскоре после ухода Робина, остальные решили вернуться в дом Мэтью и обсудить своё новое предприятие. Совещание было катастрофическим — Гарри и Мэтью не могли договориться, как им следует приступить к превращению в других существ. Тогда Элисон предположила, что будет полезнее, если они составят списки своих идей о том, что им следует предпринять.
\par
Элисон и Гарри вернулись к себе домой, сделав по пути несколько покупок, в то время, когда Мэтью забирал Робина из университета.
\par
Вечер проходил весьма приятно до тех пор, пока Элисон вновь не ощутила то странное знамение, которое явилось ей этим утром.
\par
В гостиной стало холодно, но не на долгое время, а лишь на мгновение. Это было похоже на пребывание в холодильной камере. Вслед за этим почувствовался ужасный смрад — тошнотворный запах, заполнивший собой всю комнату, как будто в ней неделями хранились полусгнившие трупы…
\par
Схватившись за голову руками и дрожа всем телом, Элисон закричала. Она наблюдала, как бешено пляшут по комнате неясные чёрные тени. Они издевались над ней. Затем раздался детский смех, дразнивший её. Она всё больше и больше всматривалась в эти тёмные пятна и ей стало казаться, что она различает в них пару рук, манящих её присоединиться к ним. Нет, они не манили её — это означало бы, что ей предоставили выбор. Нет-нет, это было похоже на то, что они угрожали. Да, верно, они угрожали ей! Они как будто говорили, что если она не подчинится им, то станет ещё несчастней, чем прежде, и будет горько сожалеть о том, что не поступила иначе.
\par
Гарри пытался успокоить Элисон и призывал её бороться с духами, которые появились в доме незваными гостями.
\par
Наконец, её вывел из лёгкого оцепенения телефонный звонок. Она сняла трубку:\\
\par
«Элисон Уайтли. Чем могу помочь?»\\
\par
Ответа не последовало — вместо него мертвящая тишина. Это было "зло" - она чувствовала зло через трубку. Послышался неописуемый гул — он звал её и умолял выйти на встречу со злом.
\par
Она отбросила телефон и указала на трубку. Вот где он, монотонный, сбивчивый гул. Он всё ещё был внутри.
\par
«Это он. Это может быть только он. Гарри, это чистое зло, как если бы сюда позвонил сам сатана! Мне страшно! Этот гул похож на какое-то жужжание, и мне кажется, что он в моей голове. Положи трубку обратно на телефон. Пожалуйста, поторопись!» - сухо произнесла Элисон. Её голос стал хриплым, а лицо — белее мела. Она походила на ледяную статую — холодную и безжизненную. Её силы были вытянуты из неё, а злом был вампир, пьющий кровь своей жертвы.
\par
Не задавая вопросов Гарри исполнил её просьбу. Затем, взяв её руки в свои, он крепко сжал их и пристально посмотрел ей в глаза. Они сидели так целую минуту, не обращая внимания на происходящее и осознавали лишь силу, которая связывала их воедино. Постепенно смятение исчезло, и Элисон, продолжая смотреть в глаза Гарри, почувствовала, что её затягивает в безопасное место.
\par
Медленно шло время. Наконец, он отвёл взгляд и спросил:
\par
«Думаешь, на другом конце линии был твой отец? Он что-нибудь тебе сказал? Присядь, я принесу бренди — оно должно тебя согреть.»
\par
Она сделала так, как ей сказали.
\par
Гарри налил ей немного бренди. Она сидела молча, раскачиваясь взад и вперёд и бессмысленно глядела на телефон.
\par
Внезапно он зазвонил снова. Элисон вздрогнула от испуга и тревожно посмотрела на Гарри, ища помощи — её глаза умоляли его не ждать, что она ответит на этот звонок. Затем раздался звонок возле входной двери. Кто бы это мог быть? Они никого не ждали.
\par
Гарри снял телефонную трубку:
\par
«Это я — Мэтт. Хочу поинтересоваться — вы не будете возражать, если мы придём к вам в гости? Или, если хотите, вы вдвоём можете заглянуть к нам.»
\par
Мэтью сделал паузу, а потом спросил:
\par
«Ты всё ещё здесь? Это на тебя совсем не похоже — вести себя так тихо. Что-то случилось? У вас всё в порядке?»
\par
«Ничего не случилось — у нас всё хорошо. Ты не станешь возражать, если мы не увидимся этим вечером? Элисон сильно устала, и я подумал, что будет лучше, если она ляжет спать пораньше. Извини за то, что произошло сегодня днём. Если подумать, то ты был прав — завтра, когда мы начнём наши превращения, мы воспользуемся твоим методом самовнушения в качестве отправной точки. Ну, пока! После того, как отвезёшь Робина в университет, приходи, и мы сразу начнём.»
\par
Прежде, чем Мэтью успел ответить, Гарри положил трубку.
\par
Дверной звонок продолжал звенеть. Элисон с опаской сидела на краешке дивана. Она указала на дверь и тихо произнесла:
\par
«Я уверена, что это он. Силы зла глубоко и основательно вошли в его чёрную душу, действуя, вероятно, рука об руку с силами самого сатаны.»
\par
«Элисон, ты не можешь прятаться от него вечно. Он знает, что ты здесь. Если он поймёт, что ты напугана, то и дальше продолжит преследовать тебя. Пожалуйста, позволь мне разрешить ему войти. Я обещаю, что не позволю ему причинить тебе какой-либо вред!»
\par
Закусив нижнюю губу, она смиренно кивнула.
\par
Гарри подошёл к двери. Осторожно открыв её и выглянув из-за угла, он увидел хладнокровную, спокойную фигуру мистера Уайтли, стоявшего неподвижно.\\
\par
«Входите.»
\par
Гарри широко открыл дверь и направил незваного гостя в гостиную, последовав вслед за ним. Он быстро подошёл к Элисон, сел на подлокотник дивана и обнял её.
\par
Как ни в чём не бывало, Ник продолжал стоять в своей властной позе. Его лицо не выражало никаких чувств. Его пристальный взгляд, казалось, проникал прямо внутрь каждого из них. Дрожь пробежала по спине Элисон, и она снова вознесла короткую молитву Богу, прося дать ей силы выдержать этот "социальный допрос". Она испытывала ужасное чувство, как будто её разум опустошился. Гул возобновился.
\par
«Прекрати это, прекрати! Чего ты хочешь от меня после стольких лет? Скажи мне правду — хотя я сомневаюсь, что тебе известно значение этого слова — скажи мне, почему ты делаешь это со мной? Чего ты хочешь добиться, обратившись ко мне вот так… это… это абсолютно выше моего понимания!»
\par
Она помолчала, собираясь с силами, а затем продолжила:
\par
«Извини за то, что я разочарую тебя, но ты меня не пугаешь, а попросту раздражаешь. Я нахожу твоё поведение довольно досадным и утомительным.»
\par
Он приблизился к ней, но Элисон отреагировала почти мгновенно:
\par
«Ближе не подходи! Отойди и сядь на тот стул! Я не доверяю тебе. Сиди там и положи руки на стол, чтобы я могла внимательно следить за тобой.»
\par
Элисон разозлилась. Она поклялась, что никому не удастся сделать ей так больно, как он уже делал прежде.
\par
«Как ты можешь говорить в таком тоне со своим отцом? Ты обращаешься со мной, как с каким-то недостойным бродягой. Я пришёл сюда, чтобы помириться с тобой. В настоящее время я нахожусь в процессе покупки дома на Милтон-роуд. Я надеялся, что ты приедешь и будешь жить вместе со мной. Я очень хочу загладить свою вину.»
\par
Он говорил очень убедительно и на его глазах заблестели слёзы, когда он сглотнул слюну, пытаясь прочистить горло.
\par
«Что?! У тебя хватает наглости предполагать, что я когда-нибудь буду жить под одной крышей вместе с тобой? Ты надеешься, что я прощу тебя после того, что ты сделал с мамой?»
\par
Она на миг прервалась, а затем заговорила спокойнее:
\par
«Прежде, чем ты уйдёшь… Мне стало любопытно, каков истинный мотив твоего внезапного интереса к родительским обязанностям? Если ты действительно так сильно хотел найти меня, то почему не появился раньше? Должно быть, прошло по меньшей мере пятнадцать лет.»
\par
Она посмотрела прямо в его глаза. Она чувствовала себя сильной и уверенной. Он разозлил её, и она хотела доказать, что не напугана, и не допустит, чтобы с ней обращались так же неуважительно, как он обращался с её матерью много лет назад… пока она не покончила со всем этим.
\par
«Ну, ты моя единственная дочь. Ты знаешь, ты точно такая же, как твоя мать. Я любил её. Я знаю, ты мне не веришь, но я любил её. Мне нужна твоя любовь. В моей в жизни были очень  тяжёлые времена после того, как она… ну, ты знаешь, окончила жизнь самоубийством. Из-за этого я чувствовал себя таким виноватым. Пожалуйста, я умоляю тебя попытаться простить меня — просто дай мне шанс. Это всё, о чём я прошу. В конце концов я — твой отец. Не мог бы твой друг оставить нас наедине — скажем, на тридцать минут?»
\par
Он сделал паузу и замолчал. Заговорив первой, Элисон потерпела поражение и проиграла в войне молчания.
\par
«Ну, может быть, Гарри мог бы уйти в соседнюю комнату всего на полчаса. Имей в виду, не дольше, чем на полчаса!»
\par
«Ты уверена, Элисон?» - спросил Гарри, внимательно глядя на неё.
\par
Она согласно кивнула.
\par
Гарри не поверил "плаксивой" истории мистера Уайтли. Он подумал, что это тщательно продуманный план, целью которого было остаться наедине с Элисон.\\
\par
Как только "плохой" Гарри покинул комнату, мистер Уайтли встал и подошёл к Элисон.
\par
«Не подходи ко мне! Вернись на своё место» - сказала она решительно.
\par
Но он не стал слушать и продолжал приближаться к ней. Грубая, хитрая улыбка растеклась по его лицу, и он хрипло рассмеялся:
\par
«Глупая сука. Ты такая же доверчивая, как твоя мать. Твоя мамочка думала, что сможет сбежать от меня, окончив жизнь самоубийством — теперь тебе придётся занять её место.»
\par
Он уселся рядом с ней, крепко сжимая её запястья и заводя руки за спину.
\par
«Уходи, ты больной, безумный, умалишённый. Уйди от меня!» - прошипела Элисон.
\par
Его глаза были странными. Он больше не слышал её жалоб. Он находился в своём собственном мире. Это выглядело так, словно сейчас им владел потусторонний дух.
\par
Она откинула голову назад и прохрипела:
\par
«Я буду кричать! Уйди отсюда — ты отвратителен!»
\par
Он не обратил внимания на её хныканье и столкнул на пол.
\par
«Отойди от меня! Ты такой противный, меня тошнит от тебя.»
\par
Она заплакала.
\par
Он ударил её ладонью по лицу. Затем закрыл её рот своим предплечьем.
\par
«Заткнись! Твоя мать вечно делала так, а я ненавижу плачущих женщин!»
\par
Он ударил её снова.
\par
Оттолкнув его, Элисон побежала к двери, нападавший быстро последовал за ней.
\par
Она схватила вазу и швырнула в него, но промахнулась и успела увидеть, как та ударилась о бежевую стену. Это была отчаянная последняя надежда спастись, но она разбилась так же, как эта ваза. Он схватил её за волосы и повалил на пол.\\
\par
Внезапно открылась дверь — в комнату вошёл Гарри и обнаружил в ней плачущую от боли Элисон. Её отец вскочил и, яростно оттолкнув Гарри, выбежал через дверь и исчез в темноте.
\par
«Элисон, какого чёрта тут…?»
\par
Не договорив, он подбежал к ней и взял за руки. Он осмотрел синяки на её лице и теле. Затем поднял её, отнёс в гостиную и осторожно уложил на диван.
\par
«Я услышал, как что-то разбилось, поэтому вернулся как можно скорее. Он не сделал этого? Ну, он не… ну, понимаешь…?»
\par
«Нет. Слава Богу, что ты вернулся вовремя и ему не хватило времени. Как только ты ушёл, он изменился. Его глаза… Ну, глаза были словно не его. И голос тоже — как будто вместо него говорил другой человек, временно завладевший его телом. Мне запомнился смех — он был ужасен.»
\par
«Боже, я чувствую себя такой грязной — он прикасался ко мне. Мне нужно принять душ и оттереться от него. Ты сходишь со мной наверх и подождёшь возле ванной? Я слишком напугана, чтобы оставаться там одной.»
\par
Её дыхание сбилось и вся она сжалась в тугой комок, а руки протянулись к Гарри, вцепившись в его жакет.
\par
«Да, конечно» - ответил Гарри, помогая ей встать.\\
\par
Когда Элисон вымылась и оделась, они вместе спустились по лестнице вниз. Гарри позвонил Мэтью и Робину, и попросил их прийти как можно скорее, объяснив, что произошло этим вечером.
\par
Они договорились о том, что, поскольку отец Элисон знал, где она живёт, было бы лучше, если бы Гарри и Элисон поменялись домами с Мэтью и Робином, временно решив их проблему.
\par
Лишь около двух часов следующего утра Элисон, наконец, удалось заснуть.\\
\par
Она с криком проснулась. Гарри успокоил её, уверив в том, что это был только сон.
\par
«Всё в порядке, малышка, я рядом. Здесь больше никого нет. "Он" не знает, где мы. Просто поспи» - прошептал Гарри, поглаживая её шелковистые волосы. Она была похожа на ребёнка, которому приснился кошмар, нуждающегося в материнской любви и заботе. Гарри чувствовал себя виноватым в том, что произошло. Если бы только он не стал слушать Элисон, поверившую в раскаяние её отца. Нет, ему следовало прислушаться к голосу своей интуиции. Теперь он чувствовал, что предал доверие, которое она оказала ему, особенно после их разговора в тот день.
\par
«Не волнуйся, Эли. Я заставлю его заплатить за это. Бог мне свидетель — он дорого заплатит» - поклялся Гарри прежде, чем задремал. Время приближалось к четырём часам…\\
\par
Мистеру Уайтли было около сорока лет, возможно немного меньше или чуть больше. Ростом он был примерно шести футов и физически крепко сложен. Его настоящим именем было Теренс, но он предпочитал называться Ником — таково было его второе имя.
\par
Он был хорош собой — смуглая кожа, светлые волосы и голубые глаза-жемчужины, глубокие, словно океанские впадины. Он имел яркую и запоминающуюся внешность и, когда был моложе, за ним бегало много девиц. При этом он никогда не довольствовался тем, что имел, ему всегда хотелось заполучать больше: заполучать  то, что многим, на первый взгляд, казалось недостижимым — замужних женщин. Ему нравилось наводить страх на людей, он получал огромное удовольствие, когда запугивал женщин.
\par
Многих из них ему удавалось очаровывать, независимо от того, были они замужем или нет — для него это не являлось проблемой. Если у них были мужья, то это было дополнительным бонусом, ведь лёгкая добыча не доставляла ему особенного удовольствия.
\par
Нет! Что ему действительно нравилось, так это острые ощущения, удовлетворение и собственное могущество: он был на высоте, когда преследовал свою следующую жертву, загоняя её в затруднительное положение, в котором она начинала испытывать страх. Он чувствовал удовлетворение только тогда, когда знал, что она испугана и ненавидит его.
\par
Много раз он был застигнут врасплох разгневанными мужьями, но ему всегда чертовски везло и у него находилось достаточно времени, чтобы сбежать, оставив позади все неприятности.
\par
У Ника не было постоянной работы. Он путешествовал по стране, устраиваясь на временную работу то здесь, то там — от бармена в придорожной закусочной до водителя большегрузных авто. В случаях, когда события принимали дурной оборот и на горизонте маячили выяснения отношений с мужем очередной обманутой женщины, он просто исчезал и больше никогда не появлялся вновь.
\par
До сих пор Ник никогда не бывал в Саутгемптоне, и ему потребовалась всего неделя, чтобы выяснить, где живёт дочь.\\
\par
Прошло семь дней, и за это время Элисон ни разу не видела своего отца. Они продолжали жить в доме Мэтью. Ник дважды врывался в дом Элисон, но обнаруживал, что теперь в нём живут Робин и Мэтью. Когда он спросил их, где Элисон и Гарри, Мэтью ответил ему:
\par
«Ох! Та парочка, что жила здесь до нас, переехала куда-то в Лондон. Извините, но они не оставили своего адреса.»
\par
После этого мистер Уайтли больше не беспокоил Робина и Мэтью, но остался на Милтон-роуд.\\
\par
Робин усердно готовился к экзаменам. Для остальных это было удачей, поскольку они сумели достичь ощутимого прогресса в своём новом проекте.
\par
Профессор дал им всем месячный отпуск.
\par
Самовнушение оказалось весьма успешным — все трое смогли перевоплощаться в разных животных и даже подражать другим людям. Они здорово веселились, когда  превращались в своего наставника, имитируя его континентальный акцент и чудные манеры. Благодаря своему новому дару, Элисон больше не чувствовала угрозы со стороны отца.
\par
Единственной проблемой, с которой они столкнулись, было время, в течение которого продолжались метаморфозы. Их превращения в живых существ длились не больше четверти часа. Вот почему, находясь в телах птиц, они даже не пытались летать.
\par
Странные ощущения студентов были похожи на лёгкое умопомрачение. В приступе увлечённости их самым удивительным и многообещающим предприятием, никто из них не задумывался о том, насколько опасными могут стать эти знания, окажись они в чужих руках при неправильно сложившихся обстоятельствах.
\par
Ощущения от обретённой ими свободы до неузнаваемости изменять внешность полностью захватили ребят. Их новое занятие было настолько увлекательным, что они посвящали ему девяносто девять процентов своего свободного времени. Но, несмотря на это, они не стали рассказывать Робину о своих успехах.\\
\par
С того дня, когда Ник изнасиловал маму Элисон, прошёл ровно год. Ночь была тёмной — бушевала гроза. Отовсюду слышались раскаты грома, как будто боги откуда то сверху разгневались на простых смертных, обитавших внизу. Вскоре за ними последовали смертоносные молнии, безжалостно разбрасываемые с небес. Окна заливало дождём. В тот вечер было слишком сыро, чтобы выходить из дома. Улицы были пустынны, словно в старых городах-призраках.
\par
Ник ехал домой из Линдхерста в Саутгемптон. Шум дождя, бьющего по лобовому стеклу, заглушал музыку, игравшую из автомобильной стереосистемы. Дворники его авто, казалось, двигались в такт со вспышками молний, освещавшими пасмурное небо.
\par
Смотреть вдаль было невозможно: яркие огни машин, едущих по встречному направлению, не были видны до тех пор, пока те не оказывались совсем рядом. Из-за плохой погоды Нику пришлось вернуться в своё жилище — в своё логово — около одиннадцати-тридцати того вечера. Это было большим разочарованием — его огорчало, что в ту ночь он так и не смог никого соблазнить.
\par
Гроза повлияла на Ника. Было ли это чувством вины или нет, но он вспомнил, как мучил свою хрупкую жену — тот тонкий изящный предмет, похожий на прекрасную фарфоровую статуэтку, который ему с такой лёгкостью удалось расколоть на тысячу мельчайших осколков…\\
\par
Мистер Уайтли не был единственным человеком, на которого подействовала гроза. Зрелище было страшным — разгорячённая, вся в поту, беспокойная фигура крутилась в раскиданной постели.
\par
Внезапно послышался странный голос. Голос кричал, проклинал и клялся, что "Он" умрёт и что "Он" жил слишком долго. Затем, на несколько секунд, наступила тишина. То, что произошло затем, было бы трудно описать словами.
\par
После того, как яростный крик затих, тело обмякло, приняло позу покойника и начало глубоко дышать. Лицо стало бледным, а глаза внезапно открылись и бессмысленно уставились в темноту. Затем! Затем послышалось бормотание — невнятное и непонятное. Тело начало левитировать, зависнув в воздухе. Его руки были вытянуты и двигались вверх и вниз. Наконец, это произошло — оно превратилось в хищную птицу. Её глаза были пугающими — изумрудно-зелёными, светящимися и похожими на факелы. Птица проследовала через открытое окно в море бушующей бури, которая вела той ночью войну против мира…\\
\par
На следующее утро ненастье утихло, и воздух наполнился свежестью — как будто разом исчезли все тревоги и волнения минувшего дня.
\par
Гарри проснулся пораньше и приготовил завтрак для Элисон.\\
\par
«Эли? Эли! Просыпайся, уже десять часов.»
\par
Слегка встряхивая её, он говорил тихо — его голос был гладким, словно шёлк. Она медленно пошевелилась и протёрла глаза.
\par
«Боже, я чувствую себя такой уставшей. Я ужасно спала этой ночью и… ох, у меня онемели руки» - вяло произнесла она.
\par
Поев и одевшись, Элисон принялась за чтение утренней газеты.
\par
Внезапно она уронила её и, побледнев, словно увидела привидение, закричала:
\par
«Гарри! Гарри! Ты это читал? Ты что-нибудь знаешь об этом?!»
\par
Гарри вбежал в гостиную.
\par
«Что? Что ты имеешь в виду? Нет, я ещё не читал газету. Говори помедленнее — объясни, что я должен был прочитать?»
\par
Она указала на статью.
\par
«Вот и всё!» - сказала она, дрожа.
\par
Он прочитал статью —\\
\par
Сегодня ранним утром был убит мистер Теренс Николас, сорока пяти лет. Его обнаружили соседи, встревоженные криками, доносящимися из верхней комнаты его дома на Милтон-роуд, Бедфорд-плейс.
\par
Лицо жертвы было сильно исцарапано, а горло — разорвано. Тело было сильно изуродовано глубокими порезами. Полиция рассматривает это дело как жестокое убийство, но не может определить, каков был мотив. Никакого взлома не зафиксировано, равно как и фактов кражи имущества…\\
\par
Гарри сел, потрясённый известием.
\par
«Что ты собираешься делать? Ты пойдёшь в полицейский участок, чтобы узнать подробности, или как?»
\par
Такой вопрос был типичным для Гарри. Он смотрел на вещи реалистично и рассудительно.
\par
«Полагаю, что мне придётся пойти туда и объяснить, что он был моим отцом. Я знаю, что ненавидела его, но я никогда не желала ему смерти. Подумать только! Он умер в ту же ночь, когда покончила с собой мама — может быть, его наконец-то настигла совесть и он тоже совершил самоубийство… Нет, этого не может быть. Судя по краткому описанию в газете, он умер насильственной смертью. Интересно, кто это сделал…»
\par
Гарри придвинулся к Элисон и обнял её.
\par
«Эли, я знаю, что такое нельзя говорить, но это к лучшему. По крайней мере, он больше не сможет тебе досаждать. Наконец-то он получил то, что заслуживал. Вероятно, его убил кто-то из тех отбросов, кому он был должен деньги. Ты должна понимать лучше всех, что он не был святым.»
\par
«Ты прав! Я не знаю почему чувствую грусть и смятение. Судя по его прошлому, вероятно, многие люди имели на него зуб… Хорошо! Пойдём в полицию и, может быть, они расскажут нам больше о том, что произошло.»
\par
Они провели несколько часов в полицейском участке. Единственный вывод, к которому пришли полицейские и медицинская экспертиза, заключался в том, что порезы на теле погибшего были нанесены когтями какой-то крупной хищной птицы.
\par
Элисон засыпали вопросами об истории его жизни и о том, почему она не жила с ним, когда была моложе. На каком-то этапе Элисон почувствовала, что находится под подозрением — идеальный кандидат с идеальным мотивом убийства собственного отца.\\
\par
Она вышла из участка бледной и замкнутой. Большую часть пути назад, находясь в машине, Элисон хранила молчание. Гарри подпевал песням, звучавшим по радио. Элисон заговорила первой:
\par
«Может быть, это была я. Я ненавидела его достаточно сильно. Что, если это была я? Что, если ночью я превратилась в хищную птицу и напала на него, не осознавая, что делаю? Поэтому моё присутствие в его доме осталось незамеченным…»
\par
Она сделала паузу, а потом продолжила:
\par
«О, Гарри! Вдруг это была я? И что теперь — ты не думаешь, что в будущем, если меня кто-то сильно расстроит, я убью его, даже не подозревая об этом? Боже, лучше бы я никогда не упоминала об оборотнях и метаморфозах несколько недель назад. Хорошо, что Робин никогда не участвовал в этом проклятом проекте!»
\par
«Не вини себя, Эли. Ты ни за что не сделала бы такого — для этого ты слишком мягкосердечна и добродушна! Но, если подумать, это мог сделать Мэтью или даже я!»
\par
Вместо того, чтобы вернуться домой, они навестили Мэтью и рассказали ему о смерти мистера Уайтли. Теперь все их мысли были заняты выяснением того, кто же совершил это неожиданное убийство.

\chapter{ВОПЛОЩЕНИЯ}
\noindent\par«Р{\scriptsizeОБИН. РОБИН!» - ПРОКРИЧАЛ МЭТЬЮ ИЗ-ЗА ДВЕРИ СПАЛЬНИ. ОТВЕТА} не последовало. Он снова постучался и попытался открыть дверь, но, к его удивлению, она была заперта.
\par
«Давай, Робин, открывай. Тебе пора вставать, ты опаздываешь. Я уже давно поднялся с постели.»
\par
Мэтью приложил ухо к двери и напряг слух, пытаясь расслышать какое-нибудь движение. Послышался низкий, приглушенный голос:
\par
«Уходи! Ты, наверное, шутишь — ещё слишком рано!»
\par
«Что значит "слишком рано"? К твоему сведению, сейчас одиннадцать-тридцать. Я принёс тебе чашку кофе. Пожалуйста, открой дверь.»
\par
Мэтью прождал ещё несколько минут, прежде чем безжизненное существо выползло из своего логова и открыло дверь, протирая заспанные глаза. Мальчик взял кофе и снова закрыл её.\\
\par
Необычно вялое поведение Робина продолжалось несколько недель и становилось только хуже. Его работа не клеилась — всем было ясно, что полученные результаты абсолютно не устраивают профессора Фергера. В конце концов профессор не выдержал и отправил Робина и остальных студентов в недельный отпуск. Загадочная усталость мальчика списывалась на растущее напряжение от предстоящих ему этим летом экзаменов.
\par
Никто не думал о ней целую неделю, в течение которой они не видели и не слышали Робина. Они не заглядывали к нему в комнату, боясь потревожить — полагая, что он занят своей учёбой.
\par
В воскресенье, ближе к вечеру, Элисон вспомнила о Робине, и только тогда до всех наконец-то дошло, что его затянувшееся одиночество выглядит слишком необычно.
\par
Тем же вечером студенты вернулись в дом Мэтью, чтобы выяснить, как там поживает их мальчик.
\par
Они стучали в дверь и громко звали его, но Робин не откликался. Это продолжалось около получаса. Так и не услышав из запертой комнаты ни слова, ни какого-либо другого звука, студенты решили выломать дверь.
\par
Они обнаружили испуганного ребёнка, скрючившегося и завёрнутого в грязное одеяло, в самом дальнем углу. Несмотря на то, что он находился в сознании, было похоже, что он не замечает их появления, дико озираясь по сторонам самым тревожным образом. Это бледное, затравленное существо, глядящее в пустоту, не видело ничего необычного в том, что кто-то вломился в его комнату.
\par
Элисон подбежала к завязанному узелком пакету и вынесла его прочь из дома.
\par
Он был отвратителен. Студентов чуть не стошнило от мерзкой вони наполовину сгнившей еды, полностью захваченной колонией плесени. Почувствовав противный запах, им стало понятно, что по какой-то причине Робин не покидал свою комнату, по крайней мере, несколько дней.
\par
Гарри вызвал врача.\\ 
\par
После внимательного осмотра стало ясно или, если точнее, было решено, что Робину необходимы только хороший отдых и тщательный уход. По всей видимости, его разум переутомился от слишком большого количества повторений во время подготовки к экзаменам. Кроме того, было похоже, что всё это время он не уделял должного внимания своевременному здоровому питанию.\\
\par
«Я был слишком глуп, полагая что Робин сумеет позаботиться о себе, и что у него всё будет в порядке. Ну как я мог быть настолько глупым? Как я мог вообразить, что у такого ребёнка, как Робин, хватит рассудка, чтобы ухаживать за собой, да ещё в таких обстоятельствах — ведь он испытывает сильный стресс.
\par
Из-за плохого самочувствия Робина Мэтью чувствовал себя виноватым.\\
\par
Шли дни, но несмотря на особый уход и внимание, Робину становилось всё хуже и хуже. Он значительно ослабел и уже не мог связать даже пары слов. Он словно окаменел. От чего именно — никто не знал.
\par
Робин не хотел, или, точнее, не мог спать. Дошло до того, что как только закрывались его глаза, он сразу же вскрикивал и пробуждался.
\par
Мэтью просыпался каждую ночь от пронзительных криков Робина и находил его сидящим "столбиком" посреди своей постели, бессмысленно глядящим в пустоту — по лицу мальчика стекали капли холодного пота, а тело сотрясала дрожь. Его недуг обострился до крайнего предела — теперь он больше не мог уснуть без помощи седативных средств.\\
\par
О неожиданном ухудшении самочувствия увядающей натуры Робина было доложено профессору — он был полон решимости обнаружить источник всех его проблем. Прибегнув к крайним мерам, он смог придумать всего один способ. На глазах у студентов профессор Фергер погрузил его в гипноз. И вот тогда, наконец, им открылась истина и все ответы.
\par
Студенты облегчённо вздохнули, но в то же время были потрясены тем, что палец обвинения указывал уже не на них, а только на Робина. Им самим никогда не пришло бы в голову расспрашивать его об этом…
\par
Профессору удалось ввести Робина в транс, не используя "опорную точку", на которой мог бы сосредоточиться хрупкий мальчик — вместо этого он воспользовался силой своего ровного, монотонного голоса, чтобы загнать его могучий разум в ловушку.
\par
Однако эта задача оказалась совсем не простой. Долгое время Робину удавалось окружать себя ментальной преградой, сопротивляясь убедительному голосу, который призывал его подсознание сбросить с себя тяжкий груз, так долго не дававший ему покоя.
\par
«Это я... Это я убил их! Не знаю, зачем я это сделал, но это был я!» - внезапно воскликнул он.
\par
«Робин, кого ты убил?» - спросил профессор, записывая каждое произнесённое слово.
\par
Седого профессора, похоже, совсем не беспокоило, что один из его студентов, особенно Робин, мог быть убийцей. Он смотрел на это как на очередной эксперимент. Робин продолжил рассказывать ему историю своей жизни и, что гораздо важнее, дал подробное описание странных видений, ставших причиной его теперешнего плохого самочувствия. Его пугали малейшие движения и незнакомые звуки.
\par
«Они! Они называли себя родителями? Они не могли меня любить — нет… они обращались со мной совсем не так. Она пытала меня, прижигала дымящимися сигаретами. Так почему бы им не пройти через ту же боль и страдания, что и я? Понимаете, пожар был устроен мной. Они умерли, зная, как я боюсь огня и раскалённых предметов. Они это заслужили!»
\par
Затем, как показалось, Робина отвлёкло что-то другое. Он невнятно забормотал, обращаясь к самому себе.
\par
Профессору снова пришлось проявить настойчивость. Это было похоже на попытку договориться с пережившим издевательства ребёнком, который слишком напуган, чтобы сообщить о своих мучителях. Разум мальчика чувствовал себя так, словно был зажат — как обвиняемый на скамье подсудимых, которого успокаивают и подводят к тому, что признание и "откровение" станут единственным разумным решением для смягчения приговора.
\par
Ему сообщили, что они, его друзья, могут лучше понять проблему и, следовательно, постараться решить её.
\par
«Мы на твоей стороне, Робин. Не бойся, мы сможем помочь тебе, если ты расскажешь нам обо всём. В чём твоя проблема? Что случилось такого ужасного? Почему ты не можешь спать?» - повторял уставший профессор.
\par
Робин рассмеялся, а затем прошептал:
\par
«У меня есть секрет. Бедняжка Элисон не знает, что я проник в её разум, когда она спала, чтобы узнать о метаморфозах. Я знаю, что из-за смерти своего отца она чувствует себя такой виноватой, но она не должна…»
\par
Он прервался и засмеялся — почти истерически.
\par
«Да! Извини, Элисон, но это сделал я… Я убил его — твоего отца! Я думал, что так будет лучше, что ты сама этого хотела — избавиться от него! Разве ты не хотела?»
\par
Его настроение внезапно переменилось, и сразу после смеха он сильно расстроился, едва не заплакав. Он проглотил слюну, пытаясь прочистить горло, и его голос задрожал.
\par
«Ну же, Робин — чего ты боишься? Убийства остались в прошлом. Раньше они тебя не беспокоили, значит есть что-то другое, произошедшее позже. Что это? Давай, ты же знаешь, что можешь рассказать нам, и что никто тебе не навредит.»
\par
Профессору пришлось быть строгим и настойчивым, ведь он хотел добиться хоть сколько-нибудь важных результатов.
\par
«Я должен сказать это вам? Я не могу! Это значит, что мне придётся заново вспоминать обо всём. А я не хочу — это слишком ужасно, это гораздо хуже, чем вы когда-нибудь могли представить…»
\par
«Нет, Робин, ты ошибаешься. Я не смогу помочь тебе бороться с этим, если ты не откроешься мне.»
\par
Мальчик в отчаянии заломил руки, пытаясь набраться смелости и описать свой повторяющийся сон, преследовавший его каждую ночь.
\par
«Мне очень страшно быть здесь. Разве вы не можете прийти и забрать меня из этого убогого места? Я умру, если вы не придёте сюда в ближайшее время.»
\par
Робин задрожал всем телом. Он свернулся калачиком и начал энергично тереть руки.
\par
«Нет причин, чтобы так сильно нервничать. Перестань трястись» - успокаивающим голосом произнёс профессор.
\par
«Меня трясёт вовсе не от того, что я нервничаю. Нет! Мне очень холодно.» - заикаясь, ответил Робин, и его зубы непроизвольно застучали.
\par
Профессор прикоснулся к его рукам и был сильно изумлён. Врачи, которые были должны наблюдать за этим уникальным ребёнком, невнимательно следили за перепадами его температуры. Профессор Фергер накрыл мальчика своим твидовым пиджаком.
\par
«Где ты сейчас? Где мы можем тебя найти?»
\par
«Я не понимаю, где нахожусь. Я заблудился. Я как будто знаю это место… имею в виду, что оно такое знакомое… но, с другой стороны, этого не может быть. Это Саутгемптон — возможно, через много лет после окончания существования.»\\
\par
«На что ты смотришь, Робин? Ну же, не бросай нас. Если ты хочешь, чтобы тебе помогли, то придётся сосредоточиться на том, чтобы давать нам, твоим друзьям, какие-то подсказки. Не позволяй никому и ничему отвлекать тебя. Ты ещё слушаешь?»
\par
«Здесь повсюду мор и болезни. Я слышу стоны — мучительные стоны, исходящие от странных скитающихся существ. Они — такие же люди, как и мы… ну, почти как мы. Большинство из них ходят полуголодными. Осматриваясь по сторонам, я вижу две группы обезумевших существ, дерущихся из-за трупа собаки. Должно быть, она умерла недавно, поскольку её только что нашли. Я видел, как убивали других животных. Это было настолько по-варварски: их забивали до смерти деревянными досками. Здесь не держат никаких продовольственных запасов — всё, что было найдено каким-либо способом текущим днём, съедается как можно скорее, чтобы не украли. Опять этот резкий ветер. Он воет и кричит, искажая стоны множества бездомных и обездоленных людей. О, нет! Что мне делать? Они здесь — Система! Мне нужно найти место, где можно спрятаться.»\\
\par
Последовало долгое молчание. Профессор пытался привлечь внимание Робина, но это ему не удавалось. Он забеспокоился о благополучии своего проекта. Должно быть, это была его вина. Убийство отца Элисон, исчезновение или, скорее, уход Робина из реального мира — ничто из этого не стало бы возможным, если бы он, ведущий в своей области профессор, не оказывал на Робина такого сильного давления. Он воспитывал Робина с самого первого дня, когда тот попал под его опеку. Он учил его пользоваться своим разумом, чтобы иметь возможность сознательно использовать весь потенциал подсознания. Кроме того, он очень гордился тем, что обучал его широкому кругу предметов и научил нескольким языкам. Робин был уникален. От природы его разум был наделён необыкновенными способностями и, учитывая дополнительную тонкую настройку профессора Фергера, было сомнительно, что кто-нибудь из когда-либо существовавших людей мог в одиночку обладать такими же силами.\\
\par
Тишина была нарушена криком Робина — ужас хлынул из тёмных пещер его разума, пронзительный визг отвращения и страха эхом отозвался в глубокой, извилистой дорожке его внутреннего уха.
\par
Его сотрясли рвотные позывы, однако выделилась лишь слюна.
\par
«Что случилось, Робин?» - спросил профессор с ноткой настойчивости в голосе. На пожилом лице профессора отразилось явное облегчение, когда Робин снова заговорил.
\par
«Я не знаю, почему закричал. Здесь это достаточно частое явление, когда людей наказывают таким образом: старому, немощному мужчине только что отпилили тупым ножом левую руку. На ноже ещё оставались пятна крови от предыдущей ампутации. Система преуспела в устрашении народных масс, используя такие инциденты в качестве наглядных примеров…»
\par
Заикающийся голос Робина снова затих.
\par
«Хорошо, Робин, когда ты проснёшься, то уже не вспомнишь, что когда-то боялся этих странных снов. Тебе предстоит долгий сон и хороший отдых, где бы ты сейчас ни был. Пока, Робин! Не забудь позвать нас, когда проснёшься. Помни! Там никто не может помочь тебе, кроме нас — твоих друзей. Ты можешь доверять только нам!»
\par
Когда профессор закончил говорить, Робин уснул. Впервые за несколько недель, закрыв глаза, он не проснулся сразу же от собственного крика. Профессор и студенты были в ужасе от того, что рассказал Робин, но им хотя бы удалось погрузить его в сон.
\par
Они покинули комнату, оставив в ней медсестру. Ей дали инструкцию немедленно известить их, когда "спящая красавица" начнёт шевелиться.
\par
Робин проспал восемнадцать часов. Он ни разу не закричал от ужаса и даже не всхлипнул, когда становился свидетелем отвратительных событий, происходящих в его новом мире.\\
\par
Тем временем профессор Фергер изучал свою подробную запись повествования Робина. Вместе со своими студентами он приступил к обсуждению её возможных значений. Что их смутило по-настоящему — это время, в котором, по словам мальчика, происходили эти события. Никто из них не смог припомнить ни одного момента в истории Великобритании, в котором страной управляло нечто, называемое Системой.
\par
Выводы были разнообразными. Во-первых, это могло быть временем неудачного правления Оливера Кромвеля. Или, во-вторых, периодом Великой чумы. Но это были только предположения — в краеведческом отделе городской библиотеки не нашлось никаких доказательств и каких-либо исторических записей. Возможно ли, что Робин ошибся и это был исторический период какой-то другой страны?\\
\par
Прежде чем Робин окончательно проснулся, профессор Фергер ещё раз подверг его гипнозу. На этот раз ему было проще. Вероятно, это было связано с тем, что мальчик больше не боялся этого загадочного места, и с тем, что впервые за долгое время ему удалось хорошо поспать. Поэтому он больше не нервничал и был готов к сотрудничеству.
\par
«Робин, ты ещё там? О чём ты думаешь? Мы хотим тебе помочь — мы сделаем всё, что от нас зависит. Не переживай — Система не сможет узнать, что сейчас ты общаешься с нами.»
\par
«Я знаю, что нахожусь в другом месте, но не знаю, как мне вернуться обратно, в наш мир. Я расскажу вам, что вижу поблизости, а затем подробно опишу то, что посчитаю интересным. Вы готовы?»
\par
Профессор был поражён спокойствием Робина. Было похоже, что он осознал ситуацию и был полон решимости предпринять всё возможное, чтобы вернуть свою жизнь в то время, которому принадлежал.
\par
«Я уверен, что это Саутгемптон. Но если бы сейчас вы были здесь, никто из вас не поверил бы мне. Высотных зданий больше нет — этих чудовищных, разрывающих небеса плоскостей. Теперь это всего лишь руины, обозначающие места своего прежнего существования. Ненавижу ходить среди них по ночам — это действительно жутко, и я чувствую глубокую печаль. Я слышал крики людей, зовущих на помощь. Должно быть, это потерянные души, блуждающие в поисках покоя… Здесь скудная растительность. По какой-то причине…  не знаю, по какой… растения и всё остальное больше не растёт здесь обильно и быстро. Высокие деревья в парках мертвы — они сожжены.»
\par
Он остановился, замолчал и задумался.
\par
«У меня плохо получается описывать здешнюю местность, это слишком сложно. Мне придётся подумать о другом способе сделать это.»
\par
«О чём ты думаешь, Робин?»
\par
Ответа не было — лишь холодная, леденящая тишина.
\par
«Ш-ш-ш… Я пытаюсь сконцентрироваться. Сейчас я не могу в подробностях объяснить, что пытаюсь сделать, но, пожалуйста, очень внимательно следите за моими глазами. Если это сработает, то я смогу получать от вас указания и советы.»
\par
Профессор Фергер был в замешательстве. Что имел в виду Робин, говоря о том, что небоскрёбы разрушены? Что он пытался сделать сейчас, требуя от всех полного молчания?  Единственный раз, когда им пришлось молчать — перед одним важным экспериментом, когда мальчик впал в глубокий транс. Профессор был крайне обеспокоен. Это место кишело опасностями, а у Робина не было времени на то, чтобы проводить новые исследования. Ему приходилось уделять всё внимание тому, что происходило в непосредственной близости…\\
\par
Долгое молчание действовало им на нервы. Прошёл целый час, но они всё ещё ничего не слышали. Они не хотели говорить, опасаясь, что побеспокоят Робина. С другой стороны, вдруг он попал в крупную неприятность и сейчас не имеет возможности с ними общаться?
\par
Они терпеливо ждали, пока не прошёл ещё один час. Элисон тихонько выскользнула из спальни и приготовила каждому по чашке чая, который был принят остальными с благодарностью. Они сидели на стульях с жёсткими спинками вокруг одноместной кровати Робина и ждали, но время тянулось слишком медленно.
\par
«Это бесполезно — наверное, он уснул. Пойдёмте отсюда. Уже ничего не произойдёт» - прошептал Гарри профессору.
\par
Они тихонько поднялись, распрямили руки, выждали несколько секунд, чтобы пробудить свои ноги, и подкрались к двери. Внезапно Мэтью тихим голосом произнёс:
\par
«Смотрите! Что это? С Робином происходит что-то странное!»
\par
Остальные повернули головы и уставились на Робина. Они никогда не видели ничего, столь же впечатляющего, если бы это можно было описать так. То, что он проделывал сейчас над собой, бросало вызов всем устоявшимся научным теориям о том, как функционирует человеческое тело. Должно быть, это галлюцинация, ведь такое не может быть правдой. Как он это делал? Робин в очередной раз удивил группу ошеломлённых зрителей. Когда они подумали, что он достиг предела своих возможностей в использовании психических сил разума, он снова доказал их полную неправоту.\\
\par
Зрители были восхищены его уникальной особенностью. Они внимательно вглядывались в его зелёные глаза. Это было похоже на просмотр фильма, проецируемого на светящийся зелёный экран. Они смогли увидеть Робина в его странном окружении. Каким-то образом ему удалось спроецировать своё положение и окружающую обстановку на зрачки своих глаз. Они размышляли над увиденным чудом. Чем дольше они смотрели на мир глазами Робина, тем сильнее болели от переживаемого волнения их тела и разум. Они никогда не смогли бы представить того, что так наглядно продемонстрировал Робин.\\
\par
Профессор наблюдал и подробно записывал, куда шёл Робин, и что он делал. Ему было интересно, как долго сможет продержаться мальчик в таком состоянии, проецируя каждый шаг через свой разум. Как скоро ему потребуется переночевать в каком-нибудь разрушенном и бесплодном месте, в котором он сможет рассчитывать на безопасность?
\par
«Робин, подойди к тому обугленному пню. Что это там лежит? Что-то вроде старой газеты?»
\par
Робин выполнил указание профессора Фергера. Там действительно была старая, измятая газета. Он пробежал по ней глазами, не особо вникая в текст — так, чтобы профессор смог прочитать её глазами Робина. Наконец, они смогли понять, где находился мальчик. Это была национальная газета, однако она была датирована 5 апреля 1988 года — это было будущее. Выдержка с первой полосы гласила:\\
\par
…Напряжение между странами Западного и Восточного блоков усилилось вчера вечером, когда представитель России покинул экстренное заседание в Организации Объединённых Наций.
\par
Между сверхдержавами произошёл обмен угрозами, поскольку русские по-прежнему отказываются вывести свои военные корабли из Ормузского пролива. В течение трёх недель они препятствовали проходу западных нефтяных супертанкеров в ближневосточную зону и из неё. Было подсчитано, что в течение двух дней все западные промышленные предприятия будут вынуждены закрыться, если важные поставки нефти не достигнут этих стран.
\par
Поступают тревожные сообщения, в которых описывается массовое развёртывание военно-морских сил и армий США и СССР. Последние новости касаются городов Ном — штат Аляска и Анадырь — Чукотский автономный округ, СССР. США имеют мощное подкрепление в виде крупных боевых кораблей и авианосцев, базирующихся в порту на острове Святого Лаврентия в Беринговом проливе.\\
\par
…Вчера вечером премьер-министр Великобритании выступил по радио с обращением к нации…\\
\par
…Похоже, что война неизбежна и мы должны готовиться к худшему, поскольку переговоры между сверхдержавами прекратились…\\
\par
Как только профессор закончил читать статью, Робин отбросил газету и вытер испачканные руки о своё грязное пальто. То, что он произнёс потом, показало его полное понимание своего призвания и обстоятельств —
\par
«Не обращая внимания на ход времени, пока сгущающиеся сумерки не сделали невозможным моё дальнейшее исследование, я вошел внутрь через обломки двери и обнаружил, что здешний довольно мрачный интерьер на удивление чист и сух. Сегодня ночью я буду спать здесь — в этом запущенном и полном призраков месте. Но вот что я должен сказать: в тихие мгновения я всё ещё ощущаю страх, который преследовал меня в первые дни моего пребывания в этом мире, но моя решительность легко отгоняет его, и я с радостью принимаю возложенный на меня долг. Я знаю, что взялся за трудный проект, который может стать спасением для человечества и подарить ему свободный мир, если я буду достаточно осторожен. Спокойной ночи! Не волнуйтесь, я буду держать вас в курсе событий.»\\
\par
Робин свернулся клубком и уснул. Маленькая, утомлённая фигурка время от времени вздрагивала, когда холодный, неподвижный воздух, словно острые кинжалы, пронзал его изношенное пальто. На следующий день он проснулся и осторожно выбрался из укрытия, чтобы посмотреть на восходящее солнце, лучи которого пытались пробиться сквозь пасмурное небо. Из-за необычайно холодной ночи его конечности онемели, но эту неприятность затмевала боль в его животе. Он не ел по меньшей мере три дня, и об этом ему постоянно напоминал собственный желудок, урча и рыча на него.\\
\par
«Профессор Фергер, вы здесь?» - спросил Робин. Его глубокие зелёные глаза открылись и снова показали окрестности, а также серьёзное неудобство, которое испытывал мальчик. По его щеке скатилась слеза, когда он задумался, вспоминая беззаботную жизнь в его настоящем мире, а главное — любовь и поддержку, которую дарили ему друзья. Если бы он мог оказаться сейчас рядом с ними, то попросил бы прощения у Элисон за то, что накричал на неё, и рассказал бы, как сильно любит её…
\par
Эти размышления быстро вылетели из его головы, когда заговорил профессор. Вспомнив, что теперь у него есть важная обязанность — спасти мир от катастрофы, Робин понял, что в данный момент его печали не так важны.
\par
«Робин, теперь мы подготовились. Ты можешь снова открыть глаза? Я не был готов, когда ты открыл их в прошлый раз. Робин выполнил то, что сказал профессор. В поисках еды он прошёл большое расстояние от того места, где заночевал прошлым вечером.»
\par
«Вы можете как-нибудь передать мне, сюда, еду? Боюсь, что скоро от меня не будет никакой пользы — с каждой минутой я становлюсь всё слабее. Вот уже два часа я ищу что-то съедобное, но, к сожалению, ничего не нашёл.»
\par
«Не волнуйся, Робин. Мы что-нибудь придумаем. Где ты сейчас находишься?»
\par
«Я приближаюсь к старому порту. До меня дошли слухи, что Система забирает отсюда припасы, но когда — я не знаю.»
\par
Наступила тишина. Зрители молча смотрели в глаза Робина — сияющие изумруды — эти сокровища хранили тайн больше, чем история любых других драгоценных камней.
\par
Мэтью первым нарушил ледяное молчание:
\par
«Возвращаясь к вопросу о том, как накормить Робина: главная проблема заключается в том, что технически он находится без сознания, поэтому мы не можем кормить его обычным образом, через рот. Но что, если мы будем кормить его через капельницу? Таким образом, мы будем уверены, что он получает необходимое количество витаминов и всего такого. А ещё меня не слишком радует мысль, что Робин будет есть всё подряд, ведь если мы правы и произошла ядерная катастрофа, то не грозит ли ему опасность от пищи, заражённой радиацией? Нам совсем не нужно, чтобы Робин умер от лучевой болезни.»
\par
То, что сказал Мэтью, было правдой: никто из них не подумал о том, как опасно есть найденную там еду. Кроме того, идея кормить Робина с помощью капельницы была простым и разумным решением их проблемы.
\par
«Отлично, Мэтью! Это блестящая идея! Элисон, сходи и сделай все нужные приготовления. Мы хотим, чтобы капельница была установлена как можно скорее.»\\
\par
Элисон поспешила сделать то, о чём попросил профессор, и вышла из комнаты, прихватив по дороге своё пальто. Она несколько дней не покидала дома Мэтью и сейчас её глаза болели, пытаясь привыкнуть к дневному свету. Теперь, когда она снова очутилась на улице и почувствовала себя частью реального мира, до неё дошел весь смысл происходящего в, казалось бы, ничем не примечательном доме Мэтью. Её раздирали сомнения… Было ли правильным их вмешательство в события будущего, пусть даже и с целью недопущения ядерной войны? Или им следует сосредоточить усилия на возвращении Робина в его родной дом, где она снова могла бы заменять ему мать, и быть уверенной в том, что он не ранен и не убит?
\par
Глядя на встречных людей, она размышляла об их дальнейшей участи. Находясь в неведении, они проживают день за днём, усердно работают, чтобы обеспечить себе безбедную старость, и стараются выглядеть культурными и обеспеченными среди таких же посредственностей, как они сами. Если бы они знали, что ждёт их в ближайшем будущем — всего через год! В таком случае каждый из них несомненно распрощался бы с образом приятного и обходительного человека. Там их ждёт совершенно иная жизнь — выживание и бесконечная борьба за существование. С другой стороны, они, возможно, предпочтут более лёгкий путь, зарыв в песок свои бестолковые головы в надежде спрятаться от надвигающейся беды. Но разве заслуживают спасения такие люди? Это больное общество попытается договориться с Системой любой ценой, включая полный отказ от личной свободы.\\
\par
Ей не потребовалось прилагать больших усилий, чтобы получить разрешение поместить Робина под капельницу в доме Мэтью, а не в больнице, хотя было бы забавно взглянуть на удивлённые лица медсестёр и врачей, рассматривающих необыкновенные глаза мальчика.
\par
Перед тем, как вернуться обратно, Элисон решила сделать покупки. Как и у Робина этим утром, её живот возмущённо протестовал против голода. Снова начался дождь. Когда она ехала по Хай-стрит, возвращаясь в дом Мэтью, то видела, как прятались от него случайные прохожие, забегая под навесы и в ближайшие магазины.
\par
После совместного завтрака они продолжили путешествие с Робином в мире будущего.
\par
«Сегодня в порту происходит что-то очень важное, я наблюдаю тут значительную активность — не думаю, что она как-то связана с теми поставками, которых ждёт Система. Как видите, здесь много вооружённых патрульных. Интересно, к чему они готовятся? Последние несколько дней они изо всех сил пытаются навести в городе порядок, а самым беспомощным жителям даже выделили кое-какую одежду — в основном, лохмотья, но она всё же лучше, чем полная нагота. Вы, наверное, помните, что все эти люди распрощались с одеждой в тот же день, когда были разрушены их дома. Как вы считаете, что может служить причиной всему происходящему? Вы видели то же, что и я.»\\
\par
«Робин, эта Система когда-нибудь упоминала о своём Лидере? Проводили ли её представители массовые собрания, провозглашали какую-нибудь доктрину? Знаешь ли ты хоть что-то об их политических идеалах? Хотят ли они, чтобы жители следовали им?»
\par
Интонация голоса профессора кардинальным образом поменялась. Он больше не считал происходящее всего лишь интересным экспериментом над возможностями разума — нет, напротив — теперь будущее представлялось ему в серьёзной опасности, и он со всей ответственностью подходил к своей работе. Поскольку причиной разрушения города, почти наверняка, послужила произошедшая в будущем ядерная война, то выяснив, какой политики придерживается правящая там сила — Система, можно было понять, кто победил в этой кровопролитной бойне, и были ли в ней победители.\\
\par
Вдалеке они увидели приближающийся большой корабль — его чёрный силуэт отчётливо выделялся на фоне бледного горизонта. Жёлто-серый туман, словно тонкий слой краски на картине художника, покрывал всё пространство над морем, слегка затуманивая вид для всех, кто находился на суше. До прибытия корабля в порт оставалось несколько часов.
\par
Внезапно на территории порта послышались чьи-то возмущённые выкрики, и внимание Робина сместилось от корабля в сторону группы людей, закованных в цепи и находящихся под конвоем — их заставляли идти к причалу. Было похоже, что похитители потратили немало времени, чтобы отыскать их. Они не походили на обычных здесь голодающих и лишённых имущества людей, бродивших с плачем и стонами среди руин. Нет, это были крупные и физически крепкие люди, которые сумели обеспечить себе приемлемые условия существования, применив силу и способность к организации, а также промышляя воровством еды, не прилагая для этого больших усилий. Вызывало недоумение, каким образом Система смогла отыскать их убежище, и чего она хотела хотела от этого сброда. Было весьма сомнительным, что их доставили сюда, чтобы встречать гостей и приветственно махать руками — в таком случае, в цепях не было необходимости. Возможно, Система сочла их опасными и решила избавиться от них, отправив на казнь. Могло случиться и так, что пленники, становившиеся с каждым днём всё сильнее — по мере присоединения к ним других людей — каким-то образом прознали о том, что должно было произойти сегодня в порту, и запланировали чьё-то убийство. Понимая, как устроена Система, Робин предположил, что вскоре она обязательно известит всех о том, как собирается поступить. Здесь было принято демонстрировать и всячески освещать любые телесные наказания и казни, вызывая страх у оставшейся части публики, и поддерживать таким образом сложившийся порядок вещей.\\
\par
«Робин, такие мероприятия… Они происходят часто?»
\par
«Нет, раньше я не видел ничего подобного. Могу попытаться это выяснить, но придётся рисковать — никогда нельзя быть уверенным в том, что твой собеседник не является информатором Системы. Это означало бы разоблачение самого себя — если бы мне пришлось выбраться из укрытия. Меня может заметить один из её стражников — не знаю, что они со мной сделают, если узнают, что я всё утро наблюдал за их стремительными приготовлениями. Эй! У меня появилась идея! Скажите, что вы о ней думаете — если, как мне кажется, здесь собрана большая часть людей Системы, то, может быть, мне стоит отправиться на их основную базу и посмотреть, что я смогу там найти. У них должна иметься какая-то письменная документация. Отсюда совсем недалеко — примерно четверть часа ходьбы, если идти быстрым шагом. В любом случае, не похоже, что корабль войдёт в док в ближайшее время.
\par
«Ты уверен, что это будет не слишком опасно? Как ты планируешь пробраться на её территорию, оказавшись поблизости? Там всё ещё будет охрана, и должна иметься разветвлённая система безопасности.»
\par
«Ну, мне нечего терять, отправляясь туда. Раньше я никогда не пытался проникнуть в это место, и мне ничего не известно о том, что кто-то другой пытался сделать то же самое.»\\
\par
Робин осторожно выбрался из своего укрытия и тихонько прокрался мимо охранника за сложенными штабелями ящиков, после чего дождался, когда тот ушёл патрулировать другой район, и благополучно скрылся из виду.
\par
Вместо того, чтобы идти вдоль дороги он предпочёл незаметно пробираться к базе окольными тропами, через развалины каких-то строений, попадавшихся на его пути. Основная масса служащих Системе людей сейчас направлялась по остаткам дорог в направлении доков, и для Робина не было ничего хуже, чем испытать судьбу, оказавшись на их пути.
\par
Профессор и его студенты чувствовали себя беспомощными — они слишком мало знали об этом месте и не понимали, как устроена Система. Но каждый из них надеялся, что в будущем станет хоть чем-то полезен для мальчика. То, что делал Робин, вызывало в них смешанные чувства. С одной стороны, они гордились тем, как он справился со сложившейся ситуацией и как мужественно ведёт себя в дикой местности, где правят насилие и жестокость. Они приходили в трепет, когда смотрели в его глаза и видели опасности, подстерегающие его на каждом шагу, куда бы он ни шёл. С другой стороны, им хотелось плакать, поскольку они всё чаще задумывались о бедности и страданиях, которые видел Робин. Никто из родителей, переживающих за судьбы своих детей, не пожелал бы им стать свидетелями или частью такого безжалостного мира. Их бросало в дрожь, глядя, как мучительно умирают от брюшного тифа, желтой лихорадки и прочих болезней люди, живущие среди нечистот, и ежедневно употребляющие загрязнённую пищу и воду.\\
\par
Когда Робин добрался до регионального штаба Системы, его сердце забилось чаще. Мысль о том, что он задумал, привела его в ужас и заставила волноваться. Здесь никто и никогда даже не помышлял о проникновении в этот огромный, тщательно охраняемый комплекс. Конечно, не могло быть и речи о том, чтобы воплотить такую идею в жизнь.
\par
Здание штаба не изменилось с тех пор, когда он видел его в последний раз. Сейчас оно располагалось прямо перед ним, поражая своим уродливым видом. В отличие от прочих построек, оно не являлось неустойчивой бетонной конструкцией, готовой рухнуть от малейшего прикосновения. Нет, штаб был другим, и при его строительстве погибло много людей.
\par
Когда Система впервые обосновалась в этом районе, она использовала местное население, чтобы воздвигнуть эту громадину. Для перевозки из порта тяжелых грузов с материалами для постройки были задействованы женщины и дети, работавшие не меньше восемнадцати часов каждый день до тех пор, пока работа не была завершена. Окон здесь не было — вместо них лишь ряды узких прорезей, закрытых изнутри толстыми стёклами.
\par
«Большую часть времени внутренние помещения должны находиться в полной темноте, если только у них нет какого-нибудь источника энергии, чтобы вырабатывать электричество. Если им нужен свет, то без него не обойтись.» — предположил Робин, размышляя, безопасно ли прикасаться к очень высокой ограде, окружавшей здание по периметру. Сетка ограды была слишком плотной и не оставляла Робину возможности пробраться через неё.
\par
«Есть идеи, как мне преодолеть этот забор? Я не могу просто подойти к главным воротам и там спросить, не будут ли они возражать, если я пройду внутрь и немножечко за ними пошпионю.»\\
\par
Прошло около пяти минут, прежде чем кто-нибудь произнёс хотя бы слово.\\
\par
«Не знаю, возможно ли это сделать, но, поскольку раньше это работало, почему бы тебе не попробовать превратиться в птицу, чтобы перелететь через ограду, а затем, в зависимости от того, какие лазейки ты сможешь найти, стать каким-нибудь другим существом.»
\par
Мэтью снова озвучил идею, родившуюся в его голове, но хватит ли у Робина сил, чтобы осуществить её? На этот вопрос мог ответить только Робин. Ничего не отвечая, мальчик огляделся по сторонам в поисках какого-нибудь укрытия, в котором можно было бы спрятаться.
\par
Найдя подходящее место, он лёг на спину, словно подражая покойнику, и закрыл глаза, на время отгородившись от профессора и студентов. Хотя они не могли видеть Робина, у них всё ещё оставалась возможность слышать его. Он начал глубоко дышать, как обычно делал перед началом серьёзных экспериментов, чтобы погрузиться в глубокий транс. Прошло немного времени, прежде чем его тело начало левитировать, и на руках, совершавших движения вверх и вниз, стал появляться слой перьев. Помимо обычных птичьих черт, Робину удалось уменьшить себя до размеров совы. Последним штрихом в его превращении стало открытие глаз, и они засветились, как фантастический криптонит — ярким, изумрудно зелёным светом. Профессор никогда не был свидетелем подобного превращения, и для него оно было таким же красивым, как рождение живого существа.\\
\par
Робин взмыл в небо и обогнул здание. Он тщательно осматривал его сверху, пытаясь найти слабое место — возможно, какую-нибудь полость, через которую он смог бы проникнуть внутрь. Ему повезло: облетев здание во второй раз, он заметил небольшую щель. Рассмотрев её поближе он выяснил, что это была вентиляционная шахта площадью около половины квадратного метра. Она была единственной и располагалась в верхней части западной стены, прямо под крышей. Пользуясь своими когтями, Робин оторвал решётку, прикрывающую вход в вентиляцию, и стремительно влетел внутрь, грациозно планируя на каждом повороте. Путешествие по ней показало, что все внутренние шахты ведут к одному центральному выходу. Робин потратил немало времени, чтобы узнать, для чего используется каждое помещение, наблюдая за работой Системы со своей стороны вентиляционных решёток.
\par
Находясь там, Робин опасался говорить, боясь, что его голос разнесётся эхом по всем шахтам, и тогда о его присутствии станет известно Системе. Он предоставил возможность своим друзьям увидеть всё собственными глазами и сделать выводы. Мальчик был уверен, что они обязательно подскажут ему, когда заметят что-то важное. В конце концов Робин потерял счёт количеству комнат в этой "тюрьме" — некоторые из них были пустыми, похожими на штабели клеток, используемых для содержания заключённых. Робин сорвал решётку, ведущую в одно из таких помещений, и влетел внутрь.\\
\par
Сова уже собиралась приземлиться на пол этой пустой камеры, когда Элисон предупредила о звуке приближающихся шагов.
\par
«Робин, быстрее, кто-то идёт. Я не знаю, что ты задумал, но, в любом случае, сделай это позже.»
\par
«Эли, на это уйдёт меньше минуты» - прошептал Робин в ответ.
\par
И тут случилось то, чего он опасался больше всего — его приглушённый голос услышал один из охранников и сразу же поднял шум:
\par
«Кто здесь? Немедленно выходи!»
\par
Робин не стал обращать внимания на окрики и продолжил то, что задумал. Он оказался прав и ему не потребовалось много времени, чтобы стать мышью — он сделал это, минуя промежуточную стадию превращения в человека. Это была впечатляющая демонстрация его паранормальных способностей, позволявших ему добиваться именно того, чего он хотел. Ему удалось видоизменить себя прежде, чем охранники ворвались внутрь. Он заполз в тёмный угол и прижал хвост к своему белому пушистому тельцу. Наконец они разошлись, обвинив друг друга в несоблюдении тишины, и продолжили патрулирование своих участков. Всё это время мышке пришлось держать глаза закрытыми, опасаясь, что зелёный огонь её глаз привлечёт к себе внимание и приведёт к тому, что охранники попытаются её поймать — не каждый день выпадает возможность владеть белой мышью с мерцающими глазами.
\par
Робин был прав — здесь действительно имелись осветительные приборы, работающие от электричества. Должно быть, где-то поблизости располагался и генератор.\\
\par
Профессор был занят обдумыванием планов вентиляционных шахт, которые он составил, когда сова летела через них. На своих чертежах от сделал отметки, для чего, по его мнению, использовалась та или иная комната. Ему удалось обнаружить одну особенную область, в которой не было шахты, но если планы были нарисованы верно, в ней было большое, неучтённое им пространство. Профессор несколько раз проверил свои расчёты и, убедившись в своей правоте, сообщил Робину о своей находке.\\
\par
Неизученное пространство оказалось именно тем, что они надеялись отыскать. Внутри находилось множество картотечных шкафов и полок, на которых пылились стопки папок. В углу, между полками и самым большим шкафом, стояло несколько картонных тубусов. Гарри оказался первым, кто заметил их.
\par
«Робин, тебе не кажется, что было бы лучше снова стать собой? Ты не сможешь прочитать ни одного документа, пока находишься в мышином тельце. В любом случае, я полагаю, что в том углу — справа от тебя — хранятся какие-то планы. У тебя осталось совсем мало времени и скоро тебе придётся оттуда уйти, если ты собираешься вернуться в порт.»
\par
Гарри говорил спокойно, но в его голосе слышалась строгость. Казалось, что он слегка завидовал Робину, у которого было такое приключение, за участие в котором Гарри отдал бы что угодно — всё это волнение и ужас, где время было самым важным и самым ограничивающим фактором. Это лучше, чем просто смотреть что-то похожее в кино и жить скучной жизнью, которая была однообразной и просто тянулась.
\par
С другой стороны, возможно, было бы лучше, если бы там находился именно Робин — ни один из прочих студентов не обладал такой силой, как их конкурент. Если они собирались спасти мир от полного разрушения и, как им казалось, от радикально-фашистской системы правления, тогда только один человек мог оставаться в безопасности, и только один человек мог значительно снизить риск быть пойманным Системой. После обдумывания плохих и хороших сторон пребывания Робина в мире будущего, Гарри охватило непреодолимое чувство раскаяния. Он не имел права на зависть. Она была одной из его скверных черт. Снова и снова он пытался бороться с ней, но безуспешно — зависть была заложена в его характере. Он жаждал всего, что сопровождалось волнением и опасностью. Он был из тех людей, которые не бывают счастливы до тех пор, пока не завершат начатое, и если задуманное не получалось с первого раза, это могло бы стоить им жизни в качестве наказания.
\par
Элисон и Мэтью были полными противоположностями Гарри. Они также наслаждались лёгким волнением, но никогда не завидовали, если кому-то удавалось сделать что-то необычное.
\par
Робин вернулся к своему прежнему виду и тщательно просматривал планы. Пока что он не нашёл ничего, что могло бы им пригодиться. Профессор отметил наиболее важные факты на случай, если они понадобятся позже. Внезапно из самого старого рулона с чертежами выпал листок бумаги. Прочитав его, Робин побледнел. Реакция потрясённых студентов и профессора была такой же, как у мальчика.
\par
Наконец-то, что-то начало проясняться. Как они могли быть настолько глупыми? Всё это время они были уверены, что знают исход войны — что, победителем, скорее всего, стала одна из сверхдержав. Во время ядерной войны информация о стране, вышедшей из неё победителем,  имела бы огромное значение, позволив большому числу жителей планеты пережить ужасы и разрушения, вызванные взрывами ядерного оружия..\\
\par
Снова послышались шаги охранников. Их голоса звучали приглушенно из-за эха, разносившегося по узким коридорам. Робин ещё раз проделал ту же процедуру подготовки к перевоплощению, превратившись на этот раз в орла. Он стремительно полетел по извилистым шахтам к вентиляционному отверстию, через которое попал сюда, и вылетел на холодный воздух.\\
\par
Наступил полдень, и солнечные лучи стали немного ярче. Им удалось пробиться сквозь густой покров облаков, излучая тепло внизу. Глаза птицы ненадолго наполнились слезами, пока она парила высоко над землёй. Робину стало грустно оттого, что он смотрел вниз и был свидетелем печального зрелища.\\
\par
Вскоре Робин добрался до порта. Он был как раз вовремя. Корабль оказался намного больше, чем они ожидали — он был похож на один из тех, что можно увидеть в учебниках по истории, где изображены мореходные суда древних римлян. Его вёсла двигались в такт, напоминая перекатывание крупных морских волн.
\par
Робин поверг в шок своих друзей, пролетев над кораблём и спланировав на его корпус. Там он остался незамеченным, усевшись на одну из балок. Он видел, как избивали кнутами рабов, приказывая им грести быстрее. Зрелище было варварским — люди, скованные цепями словно звери, хотя с этими несчастными обращались гораздо хуже, чем кто-либо мог бы представить, думая о жестоком обращении с животными.
\par
Он улетел с корабля незадолго до того, как тот достиг пристани и убедился, что там, где он приземлился, его никто не заметит, но сам он сможет видеть всё, что происходит в порту.\\
\par
Внезапно все звуки стихли. На лицах служителей Системы отразился ужас, когда невысокая, крепко сложенная фигура сошла на берег, покинув хорошо защищёный корабль. Его лицо было наполовину скрыто необычной шляпой, которую он носил. Робин сумел разглядеть глаза незнакомца — холодные, безжалостные — в них не было ни капли сострадания.
\par
Внимание присутствующих было отвлечено, когда к незнакомцу подбежал какой-то беспокойный мужчина и тихим голосом испуганно с ним заговорил. Что бы он ни сказал, это вызвало гнев его начальника, что привело к хладнокровному убийству человека, который, видимо, был частью Системы. Незнакомец выхватил меч и одним точным ударом отсёк голову несчастного. Затем он повернулся к толпе и, медленно протянув лезвие между губами, счистил кровь со своего клинка. Красная липкая жидкость закапала с одной стороны его улыбающегося рта…
\par
Среди слуг Системы распространился ужас, когда незнакомец заговорил…
\par
«Кто-то ворвался в Региональный штаб и прочитал документы Системы. Человек или лица, совершившие данное злодеяние, должны быть наказаны.»
\par
«Да, Буси-Ян» - ответила хором толпа. Большая часть стражей разделилась на группы, которые разошлись в разные стороны.
\par
Робина и его друзей охватило отчаяние. Орёл полетел к улицам разрушенного города. Он видел, как беднякам сообщили о взломе.
\par
Система выждала несколько минут, дав им возможность выйти вперёд и признать вину.
\par
Однако никто этого не сделал…
\par
И тогда Система перешла к насилию. Сначала избивали мужчин — до тех пор, пока они не теряли от боли сознание и подавали признаки жизни.
\par
Ветер разносил по округе крики отчаяния, люди разбегались в разные стороны, пытаясь найти место, где можно спрятаться. Лишь немногим удавалось спастись, увернувшись от метательного ножа и избежав удара мечом. На улицах больше не было дождевых луж — вместо воды они были наполнены кровью невинных жертв. Люди были слишком слабы, чтобы попытаться постоять за себя, они даже не помышляли о том, чтобы проникнуть внутрь какого-то комплекса, в особенности такого, который тщательно охранялся Системой.\\
\par
Охранники жгли недостроенные жилища нищих и сдирали со многих из них остатки ветхой одежды, таким образом унижая их ещё сильней.
\par
Они находились в состоянии исступления — истерия протекала по ним, словно электрический ток, в результате чего они прибегали к ещё более жестоким методам устрашения.
\par
Над женщинами издевались на глазах их мужей и детей. Когда это не давало нужного результата, детей собирали вместе, связывали и угрожали испепелить огнём.
\par
От этого отвратительного зрелища стало физически тошно не только Элисон — лица остальных, сидевших рядом с ней наблюдателей, также залила смертельная белизна.\\
\par
Бежать по улицам было трудно — они были завалены телами людей. Повсюду слышались стоны и крики боли, ветер разносил их на многие мили. Дым от тлеющей земли, закручиваясь по спирали, поднимался высоко вверх.
\par
«Я не могу позволить этому продолжаться… Мне придётся сдаться» - сказал Робин остальным. Из его остекленевших глаз потоком лились слёзы. Он проглотил слюну и попробовал отдышаться. Он не мог предположить, что произойдёт нечто столь ужасное. Люди не могут совершать такие ужасные злодеяния — никто не может быть так жесток!
\par
Как они могли?
\par
На что ещё они готовы пойти, прежде чем поймут, что все их действия бесполезны?\\
\par
Профессор переживал вместе с мальчиком. Он чувствовал ту же боль, что и Робин — но нет, с его стороны было бы неправильно сдаваться в плен.
\par
«Робин, не выдавай себя. Если ты сдашься, то всё будет будет напрасно. Подумай об этом — всё происходит в будущем. Если ты не вернёшься к нам, то оно обязательно свершится. Я знаю — мне очень легко сидеть здесь и говорить эти слова, но, поверь, я могу понять твои чувства. Подумай о том, что мы можем предотвратить все эти события. Было неизбежно, что рано или поздно Система сорвётся и прибегнет к насилию. Не сдавайся, ты единственная надежда для всего человечества, только ты один можешь помочь нам предотвратить Третью мировую войну. Пожалуйста, пойми это!»
\par
На глазах профессора заблестели слёзы, он сочувствовал Робину всем сердцем. Он знал, как это должно быть трудно для мальчика в таком юном и нежном возрасте, тем более для такого впечатлительного, как Робин.\\
\par
«Тогда мне придётся пожертвовать своим человеколюбием и готовиться к битве» - хладнокровно произнёс Робин. Затем от попытался забыть обо всём, но увиденное ещё долго стояло перед его глазами, а жуткие звуки проникали сквозь стены в его убежище.\\

\chapter{ДЖИММИ}
\noindent\parД{\scriptsizeЕНЬ БЛИЗИЛСЯ К ЗАВЕРШЕНИЮ. СОЛНЦЕ МЕДЛЕННО УХОДИЛО ЗА ГОРИЗОНТ И ВСКОРЕ} видимой осталась лишь тонкая полоса красного цвета, разделявшая небо и землю. Её отражения переливались на тёмной поверхности спокойного моря, словно кровь, пролитая на улицах уходящим днём.
\par
В темноте беззвучно двигались тени. Чудом сумевшие избежать смерти люди, оглядываясь по сторонам, бродили среди многочисленных мёртвых тел, в надежде отыскать кого-то из своих родственников или друзей. Иногда тишину разрывали рыдания, означавшие, что кому-то удавалось найти любимого человека — в разорванной одежде, израненного и залитого кровью, убитого при попытке спрятаться или спастись бегством.
\par
Воспоминания об этом проклятом дне прочно засели в памяти Робина. Чем больше он думал об увиденной им резне, тем сильнее закипали в нём гнев и желание отомстить лидеру Системы — человеку, которого звали Буси-Ян.
\par
Профессор вместе со студентами пытались утешить мальчика, изо всех сил стараясь не дать ему впасть в глубокую депрессию и тем самым полностью отрезать себя от них.
\par
Увидев зверства, на которые оказались способны люди, Робин почувствовал к ним отвращение, посчитав их бессердечными варварами. По какому праву одни позволяют себе относиться к другим как к какому-то мусору или к бездушным механизмам, как будто вторые существуют лишь для удобства первых и находятся в их полном распоряжении? Было омерзительным зрелищем — наблюдать, как называющие себя людьми животные убивают 
безвинных и беззащитных бедняков.\\
\par
«Робин, послушай! То, что произошло сегодня — не твоя вина. Ужасно осознавать, что люди, которые значительно продвинулись вперёд благодаря тому, что умеют думать, изобретать машины и даже открывать лекарства от болезней, до сих пор прибегают к насилию как к главному способу достижения своих целей. В генетике человека заложено использование примитивных средств для демонстрации собственного могущества.»
\par
«Робин, не закрывай глаза. Отгородившись от нас, ты не почувствуешь себя лучше. Да, я согласен с тобой — такие люди, как Буси-Ян, злоупотребили своим правом на жизнь и потеряли его. Но мы — не Бог и не можем судить других. Всё, что мы можем сделать — не поступаться своим достоинством и не опускаться до их уровня невежества.»
\par
Профессор сделал паузу, ожидая увидеть, оказало ли то, что он сказал, желаемый эффект на успокоение Робина.
\par
Наступила тишина — мальчик обдумывал слова профессора. Он прокручивал их в своей голове, но время от времени ему чудились крики детей и женщин, чьи жизни были так жестоко оборваны.
\par
Он находился в раздумьях… Воспоминания об улыбающихся лицах охранников, довольных тем, как хорошо выполнена работа, преследовали его.
\par
«Робин, не думай об этом. Всего этого не произошло! Помни, ты — в будущем. Если ты поможешь нам, то, надеюсь, этой резни никогда не случится!»
\par
«Хорошо, Вы правы. Просто я никогда не думал, что кто-то может быть настолько бессердечным. Эти образы отчётливо запечатлелись в моём сознании. Но не волнуйтесь — я не стану отгораживаться от вас. Я сделаю всё, что смогу.»
\par
На лице профессора отразилось облегчение, когда Робин произнёс эти слова.
\par
Наступила тишина. Все находились в глубоких размышлениях, пытаясь понять, не упущено ли что-то важное. Они вернулись к тому, что прочитали в документах, а также к событиям, происходившим до прибытия Буси-Яна.
\par
Мэтью первым прервал молчание:
\par
«А что насчёт тех мужчин и женщин, которых привели в порт вы цепях? Мы так и не узнали, для каких целей их собирались использовать. Известно ли нам, куда их увели, когда началась резня? Полагаю, нам следует отыскать их и выяснить, можем ли мы получить больше сведений. Должно быть, они очень важны, раз Система не тронула ни одного из них, находясь в приступе ярости.»
\par
«Профессор? Мэтью сделал отличное замечание — именно это было упущено нами во время всех потрясений. Сперва я попробую осмотреть корабль — после сегодняшнего вторжения в крепость его будут хорошо охранять. Буси-Ян, вероятно, сделал строгий выговор своим людям. Интересно, какую позицию он занимает в иерархии Системы? Возможно, мы скоро узнаем…»
\par
Робин закончил говорить, полетел к кораблю и обнаружил, что его усиленно охраняли. События этого утра, должно быть, значительно повлияли на это решение. У него не было ни единого шанса миновать стражей — по всей видимости они были слишком напуганы и опасались, что их казнят. Добравшись до корпуса корабля, он превратился в коричневую мышь и как можно бесшумнее побежал в трюм — ему вовсе не хотелось стать чьим-то лакомством.
\par
Вонь от рабов, которых использовали на судне для гребли, была невыносимой, способной довести кого-нибудь до потери сознания. Въевшиеся в доски пыль и грязь, покрывавшие палубу, мешали мыши быстро передвигаться. Неподалёку измождённый голодом мужчина сосредотачивал своё внимание на таракане. Затем его рука, словно молния, метнулась вперёд и схватила вожделенную добычу. Осмотрев свой приз, мужчина сунул его в рот, после чего послышался громкий хруст. Когда он сглотнул, на его лице отразилось блаженство.
\par
На голых спинах большинства гребцов виднелись шрамы. У некоторых имелись открытые раны, по краям которых свисали куски ободранной кожи. Отовсюду раздавались стоны и кашель. Инфекции и болезни, которыми были заражены эти люди, вероятно, являлись многочисленными. Кто-нибудь мог подумать, что подобные условия труда были отменены ещё во времена Средневековья.
\par
В трюме пахло не только гребцами. Вдохнув, Робин почувствовал сильный запах отхожего места. Здесь не было никаких ёмкостей, в которые могли бы помочиться мужчины. Было сомнительно, что когда-нибудь их освобождали от тяжёлых железных цепей, крепко сковавших ноги. Многие из них лежали на вёслах, пытаясь занять наиболее удобное положение для сна в тесном пространстве. Двое спустившихся вниз стражников забрали одного из членов команды — он был мёртв. Несчастная душа оставила своё тело либо от чрезмерного истощения, либо от какой-то болезни — первое казалось наиболее вероятным.\\
\par
Покинув это жалкое обиталище, Робин отправился исследовать оставшуюся часть корабля. Он не нашёл того, что искал.
\par
Прежде чем уйти, Робин поджог каюту капитана — навигационные карты загорелись почти мгновенно. Охране пришлось освободить всех гребцов — рабочая сила доставалась им нелегко — после чего судно было покинуто. Зрелище было необыкновенным: пламя охватило весь корпус и выглядело очень красиво на фоне закатного неба. Чёрный дым поднимался высоко вверх, где встречался с порывами ветра, уносящими его в даль.
\par
Среди команды послышался довольный шёпот. Большинство рабов не видело света и не дышало свежим воздухом больше месяца. Раздался мощный хлопок, по деревянному корпусу корабля пробежала трещина и он быстро затонул в Те-Соленте. Там, завязнув в илистом дне, судно упокоится с миром навечно.
\par
Между охранниками разгорелся яростный спор о том, как начался пожар и кто заходил в каюту капитана. Среди Системы распространился ужас — они поняли, что кому-то придётся сообщить эту новость Буси-Яну. Зрелище было довольно комичным — они тянули жребий, пытаясь выяснить, кому из них не посчастливится встретиться с начальником.
\par
Ошеломлённых рабов выстроили в один ряд и повели в сторону крепости.
\par
Это был шанс Робина проникнуть на базу через главный вход. Он снова превратился в орла и полетел в её сторону, достигнув нужного места раньше Системы. Там он снова стал мышью — было бы менее заметно, если бы он вполз, а не пролетел над головами охранников.
\par
Он выждал некоторое время до прибытия Системы, внимательно изучая здание. Оно производило тяжёлое впечатление: по спине мальчика пробежала дрожь, когда он смотрел на него — идеально прямые углы, в его планировке чувствовалась строгость и методичный идеализм. Можно было представить, что это своего рода фабрика, на которой людям промывают мозги, удаляя их разум и заменяя его роботизированным, компьютеризированным механизмом — в конце производственной линии выбрасывается множество зомби, проповедующих странную доктрину Системы…\\
\par
Солнце скрылось за горизонтом, и наступили сумерки. Вдалеке Робин увидел охрану, сопровождавшую пленников — их колонна двигалась очень медленно. Команда бывших гребцов еле могла идти — они были сильно истощены, их ноги были покрыты язвами, а ступни были босыми.\\
\par
Прошло не так много времени с тех пор, когда Робин впервые увидел этот обширный комплекс — и вот он снова проник в него. Вестибюль главного входа был пустым. На лицах дежуривших здесь стражей отражалось недоумение — никто не ожидал, что сюда приведут рабов.
\par
Тот мужчина, которому не повезло вытащить короткую соломинку, с опаской выступил вперёд и заговорил:
\par
«Могу ли я увидеть господина Буси-Яна? Свободен ли он? Я должен объяснить ему, что произошло.»
\par
Бледный, как полотно, он был направлен к жилищу Буси-Яна. В памяти стражника прочно засели воспоминания о том, что случилось с его предшественником. Он боялся, что его постигнет та же участь, и тогда придётся увидеть, как Буси-Ян достаёт из-под мантии свой клинок и одним точным ударом срезает его голову с плеч…\\
\par
Громогласный рёв Буси-Яна эхом разнёсся по зданию, затем послышался вопль охранника, молившего о пощаде. Буси-Ян достал меч. Раздался пронзительный визг, а затем глухой удар головы о пол.\\
\par
«Бедный человек» - подумал Робин, —
\par
«В этом нет никакой справедливости. Этот стражник не несёт ответственности за пожар.»\\
\par
Очевидно, поджог и затопление корабля не на шутку встревожило Буси-Яна. Лидер боялся — он гневно кричал и угрожал казнить своих советников, если они не предложат подходящего решения в установленный срок.
\par
«Мы должны как можно скорее вернуться с этими людьми. Мао Цзе-чейк будет крайне недоволен, если мы не вернёмся с ними, как он приказывал. Кроме того, жизненно необходимо, чтобы этот план был выполнен без промедления.»\\
\par
«Робин, мне кажется, я знаю, о чём он говорит. Сейчас я не могу этого объяснить, но мы должны найти и освободить этих людей как можно скорее.»
\par
Профессор был очень решителен, поэтому Робин не стал задавать вопросов и сразу же покинул резиденцию Буси-Яна, отправившись на поиски «элиты» — наиболее заметных представителей народных масс.
\par
Это не заняло много времени. Несмотря на помощь профессора, большая часть здания всё ещё оставалась необследованной. Профессор Фергер воспользовался методом исключения и, имея общее представление о предназначении некоторых зон, ему не составило труда догадаться, где содержались пленники.
\par
Было непривычно видеть, как Система делится с кем-то своими запасами продовольствия, но сейчас она раздала заключённым миски со свежими овощами и ещё чем-то, похожим на мясо. Эти люди, должно быть, имели важное значение, поскольку Система прилагала усилия, заботясь о них. Сцена напомнила Робину цыплят на птицефабрике, которых откармливают перед тем, как отправить на убой.
\par
Он оглядел камеру заключения. Вывести пленников на свободу, в безопасное место, будет чрезвычайно сложно. Насколько он мог видеть, здесь имелось только одно место, через которое можно было войти или выйти — дверь. Было бы не лучшей тактикой пытаться выходить через дверь, которую охраняют.
\par
«Профессор, что я должен сделать, чтобы вытащить их отсюда? Это почти невозможно!»
\par
«Робин, как насчёт того, чтобы найти генератор? Если ты сможешь вывести его из строя, то, надеюсь, охранники будут заняты какое-то время. По крайней мере, это даст нам возможность избежать встречи с ними.»
\par
«У вас есть идеи насчёт того, где находится генератор? И ещё… Я полагаю, что здесь должно быть ещё одно вентиляционное отверстие, ведущее на первый этаж и наружу. Возможно, будет лучше, если я попробую вывести их через него? После того, как они покинут здание, им, по сути, останется лишь перебраться через ограду. Затем я отведу их в то старое, заброшенное здание, где ночую.»
\par
«Мы можем попробовать, Робин» - ответил профессор.
\par
«Я думаю, что если здесь действительно имеется вентиляционное отверстие, ведущее на нижний этаж, » - начал Мэтью,
\par
«...то оно должно располагаться на противоположной к главному входу стене.»
\par
В его голосе слышалась неуверенность. Мэтью не любил делать замечания, когда сомневался в своей правоте.
\par
Робин поспешил прочь, оставив пленников доедать ужин. Он направился туда, где, как он надеялся, должно было находиться это отверстие. Тщательно обыскав все комнаты у северной стены, он обнаружил, что оно находится под столом. Теперь, когда Робин об этом узнал, он мог приступить к поиску генератора.
\par
«Робин, подожди минутку. Если мы применим логику, то, возможно, сможем избавить тебя от необходимости напрасно растрачивать энергию в попытках найти генератор. Итак, наземное вентиляционное отверстие расположено на северной стене. Главный вход — на южной. Заключённых, которых мы хотим освободить, держат возле восточной стены. Вентиляционное отверстие, расположенное высоко, прямо под крышей, если не ошибаюсь, находится на западной стене. Генератор будет в подвале у западной стены, потому что, если ты помнишь, там была одна большая труба, ведущая прямо вниз от вентиляционного отверстия. Я просто пытаюсь представить, как расположены помещения Буси-Яна и стражников относительно остальных объектов. О! Вспомнил! Они — возле комнаты, в которой хранятся документы и планы — на западной стороне.»
\par
«Если вы правы насчёт того, где находится генератор, то вывести их будет нетрудно — мне почти не придётся прилагать усилий.\\
\par
Робин направился к западной стене. Ему пришлось быть весьма осторожным, поскольку профессор был прав насчёт того, где находились Буси-Ян и охранники. Он направил свою маленькую мышиную мордочку вниз, чтобы никто не заметил его зелёных флуоресцентных глаз.
\par
Комнат и коридоров было великое множество и все они были похожи между собой. Все комнаты имели одни и те же размеры, словно множество прямоугольных коробок или детских кубиков, аккуратно расставленных кем-то в ряды и столбцы. В конце ряда обнаружилась незапертая дверь. Убедившись, что в коридоре никого нет, мышь вновь обрела свой прежний, человеческий вид. Робин открыл скрипучую, тяжёлую дверь и прокрался внутрь. Вот она, шумная машина! Она стояла посреди комнаты и вибрировала, производя электричество.
\par
«Что мне с ней делать?»
\par
«Робин, обойди вокруг неё, чтобы я мог хорошенько рассмотреть её устройство.»
\par
Мальчик выполнил указание профессора. Он медленно прошёл вокруг генератора, задерживаясь возле участков с наибольшим количеством трубок и проводов.
\par
«Робин, ты видишь две трубки, идущие параллельно? По одной из них в генератор поступает вода. Вот этот вентиль на трубке — именно его тебе придётся закрыть. Это приведёт к тому, что генератор перегреется. В таком случае он должен прекратить подачу электроэнергии, и, если быстро не устранить неисправность, произойдёт возгорание и, в конечном итоге, взрыв. Итак, тебе нужно вернуться к пленникам и начать выводить их через вентиляционное отверстие, а затем, используя способности своего разума, отключить воду — полностью перекрыть вентиль. Подожди три минуты, после чего накоротко замкни электрику. Искра должна вызвать воспламенение всей этой штуковины. Удачи, Робин! Я уверен — ты справишься.»\\
\par
Робин вернулся к заключённым. Прежде, чем зайти в камеру, он выждал, когда охранники пойдут в противоположном направлении. Затем он принял человеческий облик и пробрался в неё.
\par
«Не говорите ни слова! Я должен вытащить вас отсюда — прежде, чем они примут новые меры для вашей отправки.»
\par
Публика была в ошеломлении — люди смотрели на него с широко открытыми ртами. Как попал сюда этот малыш и откуда узнал, что их держат здесь? Примерно такие вопросы рождались сейчас в их головах. Опыт подсказывал им, что он мог работать на Систему. Казалось невозможным, что кто-то станет рисковать своей жизнью ради того, чтобы прийти и спасти их.
\par
«Как бы то ни было, сбежать их этого места невозможно. Интересно, как этот клопик собирается нас спасти?» - подумал один из пленников. Это был высокий мужчина — широкоплечий и крепкий — вероятно, их лидер.
\par
Робин повернулся к нему и заговорил. Мужчина был удивлён тем, что Робин ответил на его не прозвучавший вопрос.
\par
«Да, сэр, я понимаю, что вы относитесь ко мне с подозрением, но в этом нет необходимости. Я могу вытащить всех вас отсюда. Просто слушайте, что я скажу, и следуйте моим указаниям. Когда мы окажемся на свободе, я отведу вас в безопасное место.»
\par
«Вот что я имею в виду: за дверью — охранники; мы не можем просто так отсюда уйти.»
\par
«Я сказал вам слушать и выполнять мои инструкции.»
\par
Лидер кивнул в знак согласия и озадаченно склонил голову набок.
\par
«Сейчас не нужно раздумывать о том, что я собираюсь сделать — просто следуйте за мной и, прежде всего, храните молчание. Я не хочу слышать ни слова, пока мы не окажемся достаточно далеко отсюда. Понятно?»\\
\par
Робин неподвижно стоял и молчал. Он начал глубоко дышать, а затем свершилось то, что можно было считать его визитной карточкой — он превратился в орла. Его когти оторвали толстый лист металлической сетки от входа в вентиляционное отверстие. Он спрыгнул на пол шахты и пошёл по ней, а остальные быстро последовали за ним. Робину хотелось проделать этот путь в полёте, но ему пришлось вести заключённых через сложный лабиринт тоннелей. Пленники пребывали в состоянии потрясения — никогда прежде они не видели ничего подобного!
\par
Гордо вышагивая, Робин сосредоточил свои мысли на том, чтобы сначала установить на место сетку, преграждавшую вход в вентиляцию, а затем на перекрытии водяного вентиля. Он позаботился о том, чтобы через пять минут после отключения подачи воды в электоцепи генератора произошло короткое замыкание.
\par
Вскоре они достигли северной стены. Робин снова оторвал когтями сетку. Выбравшись из тесной шахты, орёл вернул себе человеческий облик. Беглецы вылезли вслед за ним, и, в очередной раз испытав шок, увидели, как их спаситель опять превратился в неряшливого мальчишку. Однако они продолжали молчать. Робин отвёл их к ограде.
\par
Внезапно это произошло — раздался мощный взрыв, и вверх по стене восточной стороны здания взметнулись языки огня. Над ним вознёсся дым, распростёршийся вдоль пасмурного неба. Красное пламя выглядело невероятно красиво на фоне абсолютно чёрного неба. Слышались крики, свидетельствующие о хаосе и неразберихе.
\par
Они добрались до ограждения. Робин снова использовал свой разум, наделив его силой поднять стальную сетку на достаточную высоту, чтобы следующие за ним люди могли пробраться под ней.
\par
«Молодец, Робин» - похвалил его профессор.
\par
«Это было сделано превосходно! Я думаю, что какое-то время Система будет полностью занята решением возникших проблем.»
\par
Он остался очень доволен мальчиком, хотя и не был уверен в том, правильно ли тот поступил, показав посторонним людям, на что способен. Это могло стать плохим ходом в игре и иметь одно из двух последствий: во-первых, они могут почувствовать угрозу со стороны Робина и попытаться его убить, пока он спит, или, во-вторых, они могут стать очень хорошими союзниками и, следовательно, принести пользу.
\par
Беглецы проследовали за Робином до старых развалин, где он спал прошлой ночью.
\par
По иссохшей скрипучей лестнице он провёл их вниз, в подвал. Здесь было очень пыльно и пахло сыростью, а в щелях между кирпичами рос мох и какие-то мелкие грибы.
\par
Как только они расселись на полу, Робин обратился ко всей группе, состоявшей из мужчин и женщин:
\par
«Теперь можно не опасаться и говорить» — сказал Робин. Он обернулся к их вожаку:
\par
«Видишь? Выбраться оттуда оказалось вполне возможно. Я знаю, ты сомневался во мне, но у тебя была веская причина. Полагаю, что у тебя есть несколько вопросов, на которые ты хотел бы получить ответы. После того, как ты закончишь, я, в свою очередь, хотел бы получить некоторые сведения от тебя. Кстати, как твоё имя?»
\par
«Джеймс, но все зовут меня Джимми. Я был журналистом… раньше. Не знаю, что ты там делал, но это тебя мы видели превращённым в орла?»
\par
«Да, вам это не приснилось — я стал орлом. Прежде, чем вы спросите — я не какой-то там колдун. Вам нечего бояться. Я здесь для того, чтобы помочь вам бороться с Системой. Мне нужно сотрудничество — чтобы вы сражались вместе со мной, а не против меня.»
\par
«Как ты добрался до нас, не будучи замеченным охраной?»
\par
«Это не так уж и сложно, если имеется стратегия. Я устроил диверсию. У Системы, как вы, наверное, знаете, есть генератор для производства электроэнергии. Всё, что я сделал — убедился, что трубки, по которым вода текла вокруг двигателя, больше не могли выполнять эту жизненно важную для него функцию. Я перекрыл вентиль, в результате чего двигатель перегрелся и, следовательно, прекратил подачу энергии. Взрыв, который вы слышали, произошёл из-за того, что я своевременно замкнул его электрическую часть и, таким образом, вызвал пожар через пять минут после закрытия вентилей. Это дало мне достаточно времени, чтобы вывести вас из здания.»
\par
«Здесь у тебя есть родители?»
\par
«Нет, они погибли во время пожара.»
\par
«Ну, это просто чудо, что ты так долго выживал в одиночку! Итак, ты хотел получить ответы на некоторые вопросы — задавай их. После того, как ты помог нам, мы перед тобой в долгу. Если тебе нужна помощь — проси и не стесняйся.»
\par
«Во-первых, мне нужно знать, когда именно начали сбрасывать бомбы — этого я не знаю. По моим предположениям, сначала война велась между Россией и Америкой, тогда почему Китай управляет сейчас Великобританией?»
\par
«Дата падения первой бомбы — седьмое апреля тысяча девятьсот восемьдесят восьмого года. Насколько я помню, как только всем стало ясно, что ядерной войны не избежать, появилось множество сообщений об исчезновении людей из самых густонаселённых городов Китая. Эти известия заставили задуматься представителей многих наций. Китайский лидер потребовал от глав мировых сверхдержав передать ему свои полномочия до того, как начнётся обмен ядерными ударами. Думаю, ты можешь представить, что вместо ответов Мао Цзе-Чейк получил насмешки со всех уголков планеты. Затем миру стало известно, что в Китае уже построено некоторое количество подземных городов в рамках подготовки к ядерной катастрофе. Также ходили слухи, что за последние полтора десятилетия они вложили огромные средства в научные исследования по выращиванию грибов, мхов и прочих мелких растений без света и почвы — в условиях, которые являются единственно доступными в подземной среде. Естественно, сверхдержавы восприняли это с изрядной долей скептицизма. Это казалось им невозможным, поскольку у их учёных не было каких-то крупных прорывов на этом фронте, и было совершенно немыслимо, чтобы из всех народов мира, этого достигли только китайцы.»
\par
«Ты хочешь сказать, что им удалось найти альтернативный способ производства растительности? Теперь понятно, что в этом был смысл. Зачем подвергать свой народ опасности заражения через продукты питания, на которые воздействовали радиоактивные осадки?»
\par
«Хочешь узнать что-то ещё? Думаю, всё остальное достаточно очевидно.»
\par
«Где вы брали запасы пищи? Понятно, что была и другая еда, кроме той, которую вы могли найти на улицах.»
\par
«Большую часть припасов мы получали у Системы. Она не давала их — мы понемногу крали припасы, когда приходил корабль снабжения. Он бывает почти каждые две недели. Следующая партия должна поступить завтра.»
\par
«Хорошо. Я думаю, лучшее, что сейчас мы можем сделать, — это хорошо выспаться. Завтра у нас будет очень напряжённый день. Я объясню, что предлагаю сделать, сразу после того, как придумаю план.»
\par
«Ещё раз спасибо за то, что ты сегодня сделал для нас. Ты можешь рассчитывать на нашу поддержку» — ответил Джимми. Затем он начал отдавать распоряжения своим людям, после чего распределил их по местам для ночлега. Он поручил им заботиться об этом загадочном мальчике. Они не знали даже его имени, но были рады довериться ему и помогать всем, чем могли.\\
\par
Робин не спал до самого раннего утра. Он покинул своё убежище, чтобы поговорить с профессором.
\par
«Теперь мне понятно! Могу поспорить на всё, что угодно — этих людей собирались отправить обратным рейсом в Китай. Они пригодились бы для строительства новых подземных городов. Скорее всего, там произошёл демографический взрыв, и теперь китайцам не хватает жизненного пространства.»
\par
«И что может быть эффективнее, чем рабский труд? После завершения стройки от них можно избавиться, словно от пары изношенных перчаток, которые выбрасывают по окончании зимы» - ответил Робин.
\par
«Было бы интересно узнать, как китайцам удавалось выращивать растительность в таких неблагоприятных условиях» - сказал профессор, изо всех сил стараясь придумать, как можно выяснить это. Однако учитывая, что продукты доставлялись сюда из Китая, становилось понятно, что здесь им не удастся найти никаких документов на эту тему.
\par
«Просто удивительно — сверхдержавы так долго заботила и поглощала трата миллиардов на оружие, что никто из них даже не подумал о том, что оно может стать бесполезным после войны, если такая страна, как Китай, хорошо подготовится к такому развитию событий. Забавно думать, что страна, считавшаяся очень бедной, могла иметь множество преимуществ. Китайские лидеры хорошо позаботились о судьбе своего народа и сумели найти способ его защитить.»
\par
Робин был впечатлён. Однако он возненавидел Систему за сегодняшнее хладнокровное убийство невинных людей.\\
\par
В одном из разделов тех документов, которые они прочитали, нашлись письменные свидетельства убийств в Китае. Выяснилось, что китайцы медленно лишали жизни неизлечимо больных и стариков. Они говорили их семьям, что ничего не могут сделать для несчастных пострадавших и что государство предприняло все возможные меры. Китайцы обвиняли другие страны в том, что те не сотрудничают и не делятся информацией о методах лечения болезней и прочих недомоганий. На самом же деле, от людей избавлялись, словно от ненужных домашних питомцев. Китайцы считали их лишним "грузом": всё, что они могли — это потреблять пищу, дышать ценным воздухом и занимать чьё-то жизненное пространство. Все думали, что врачи и медсёстры заслуживают доверия — каждый надеялся, что они сделают всё возможное, чтобы спасти жизнь близкого ему человека, а не станут играть в Бога, решая, кому следует умереть.\\
\par
Они продолжили обсуждение сведений, собранных этим днём. Профессор был очень доволен успехами Робина, а также тем, как ему удалось заслужить доверие Джимми и его товарищей. Не оставалось никаких сомнений, что Робин мог рассчитывать на их поддержу.
\par
Беглецы не стали расспрашивать мальчика, откуда взялись у него такие способности. Было полезно, что вместе с ними Робин стал для Системы гораздо опасней, чем раньше. Завтра наступит день, который она обязательно запомнит, и о котором будет потом жалеть.
\par
Перед тем, как отправиться спать, он снова превратился в орла и полетел к штаб-квартире Системы.
\par
Оказавшись на месте, он позлорадствовал, глядя на тот беспорядок, который вызвал. Робин был раздосадован, когда увидел, что Буси-Ян всё ещё жив — он надеялся, что его враг погиб во время взрыва. Однако ему удалось нанести серьёзный ущерб — в огне погибло много людей Системы, а комплекс значительно пострадал: было снесено всё западное крыло здания.
\par
Буси-Ян был сильно обеспокоен. Многое из того, что произошло, так и осталось для него непонятным. Во-первых, утром кто-то проник в штаб, затем на его корабле случился пожар и, наконец, взорвался генератор и сбежали рабы. И всё это случилось в один день! Буси-Ян был до крайности суеверен. Ему казалась, что один из его богов страшно разгневан, и поэтому он решил, что теперь ему придётся терпеть неудачи во всех делах.
\par
Робин мог убить Буси-Яна прямо здесь и сейчас — так же, как он расправился с отцом Элисон, но, вместо этого, он решил позволить ему немного помучиться. Если то, что произошло сегодня, показалось Буси-Яну плохим, то завтрашний день станет для него абсолютным кошмаром.\\
\par
Робин вернулся к развалинам. Он снова принял человеческий облик и проспал до рассвета.
\par
Проснувшись, он обнаружил, что рядом с ним сидит Билли — один из членов банды Джимми. В его руках был кусок хлеба и ещё что-то, напоминавшее своим внешним видом пожухлые листья салата.
\par
«Держи» - сказал он, предлагая Робину еду,
\par
«Ешь. У нас этого полно — украли у Системы во время последнего набега.»
\par
Робин съел всё, что ему принесли. Затем он поднялся на ноги и пошёл искать Джимми. Вместе, они обсудили предстоящие события текущего дня, которые запланировал Робин.
\par
«А теперь скажи мне, в какое время обычно прибывает корабль?»
\par
«Около полудня.»
\par
«Хорошо. Это значит, что у нас достаточно времени, чтобы расставить всё по местам.»\\
\par
После решения о том, что в отдельности должен делать каждый участник, банда покинула своё убежище и направилась к пристани, куда, как ожидалось, должен был вскоре прибыть корабль.
\par
За всё время, пока они находились в различных укрытиях, ими не было произнесено ни единого слова. Робин снова и снова обдумывал составленный им план, проверяя и заново перепроверяя его — желая убедиться, что в нём нет недостатков. Ему не хотелось, чтобы что-нибудь вдруг пошло вкривь и вкось. Из-за того, что он не сумел скрыть своего нахождения в крепости Системы, многим пришлось умереть, и ему не хотелось отвечать за дальнейшую гибель невиновных людей. Они рассчитывали на него, и надеялись, что он может дать им свободу от этого авторитарного режима.
\par
«Профессор, как Вы считаете, это сработает? Есть что-нибудь такое, что я упустил из виду?» — прошептал Робин.
\par
«Не волнуйся, Робин. Твой план безупречен. Остаётся только надеяться, что остальные хорошо понимают, что должны делать» — ответил профессор.\\
\par
Робин разработал план, который мог нанести сильный удар по Системе и в то же время добиться свободы для этих людей на юге Англии. Пройдёт достаточно много времени, прежде чем Система в других областях осознает, что произошло здесь, но к тому времени будет слишком поздно. Люди станут сильнее и смогут дать решительный отпор любому вторжению.
\par
«Должно быть, близится полдень — солнце почти в зените» — подумал Робин. Он нервничал. Раньше он никогда не чувствовал себя так тревожно, как сейчас. Ему не давала покоя мысль, что он несёт ответственность за жизни многих людей — что все они зависели от него. Люди увидели в нём кого-то вроде Бога после того, как стали свидетелями его «коронного номера» — превращения в орла.\\
\par
Он оказался прав. Вот и он — корабль. Судно выглядело не так величественно, как прежнее, на котором прибыл накануне Буси-Ян. И снова, это был корабль, для управления которым использовалась рабская сила. Глядя на него, Робин вспомнил о других несчастных, которых он уже видел: прикованных к вёслам, полуголодных, со следами на спинах — шрамами, свидетельствующими о жестоком нраве кнута. Робин надеялся, что это вынужденное путешествие — последнее из тех, которые им пришлось совершить.
\par
Осмотревшись по сторонам, Робин заметил, что к причалу приближается Буси-Ян в сопровождении личной охраны. Лицо мужчины было бледным и отстранённым. Он был напуган, растерян, и пытался себя убедить, что больше ничто не может выйти из-под контроля. Другой оплошности просто не может быть! Нет, он не сможет смириться с ещё одной неудачей. Он не сможет объяснить начальству свой четвёртый провал. То, что все они случились во время его пребывания здесь, было слишком большим совпадением. До сих пор у него не бывало такого количества проблем во время столь коротких визитов. Всё это походило на какой-то кошмар, от которого он надеялся избавиться, внезапно проснувшись.\\
\par
Робин приступил к выполнению своей части плана.
\par
Он начал с того, что обманул разум Буси-Яна, нашёптывая ему, что тот скоро умрёт:
\par
«Ты не знаешь, когда и как умрёшь, но, поверь мне, ты умрёшь. Скоро ты будешь освобождён от своих земных обязательств» - шелестел Робин в сознании Буси-Яна.
\par
На лице мужчины отразился ужас. Он прикрикнул на своих телохранителей, чтобы они держались поближе к нему, и внимательно смотрели по сторонам. Он был в ужасе и не мог понять, откуда исходит голос, объяснив его тем, что собственное воображение играет с ним злую шутку. Он сильно нервничал, опасаясь, что дела могут пойти не так, и находился на грани нервного срыва. Какая прекрасная возможность для разума накалить ситуацию до предела! Гладя на то, как реагирует Буси-Ян, Робин пришёл в восторг.
\par
Вскоре после того, как корабль зашёл в док и были выгружены все припасы, была приведена в исполнение остальная часть плана.\\
\par
Робин превратился в орла и влетел на борт корабля. Там он поджёг судно, и рабы снова были освобождены. Система пришла в полный беспорядок. На этот раз Буси-Ян стал свидетелем начала пожара и видел, что на корабль не был допущен ни один человек. Было уже слишком поздно пытаться тушить пламя, вовсю ревущее на палубе, и поэтому, как и другое судно, это постигла участь сгореть и быть затопленным в Соленте.
\par
Как и надеялся Робин, охранники начали спорить между собой. Они так сильно ругались и обвиняли друг друга в катастрофе, что не заметили, как удирали Джимми и его товарищи, прихватив запасы оружия и продовольствия. Это было именно то, в чём они нуждались — что-то такое, с помощью чего они могли бы бороться с Системой.  — Оружие!
\par
Спрятав провизию в своём убежище, вооружившиеся мужчины отправились к дороге, возле которой они могли бы устроить засаду. Вскоре они услышали, как, двигаясь в их направлении, возвращались из порта Буси-Ян и его охрана.\\
\par
Робин посчитал это отличным шансом. Он выбрал идеальное место — оно превосходно подходило для того, чтобы заставить охранников в панике разбежаться в разные стороны. Идея Робина была забавной. Он сосредоточил свой разум на ужасно обезображенном трупе, лежащем посреди дороги. Когда стражники проходили мимо него, мальчик пожелал, чтобы труп встал на ноги, двигался и издавал стоны. Это стало первым ударом по Системе. В страхе, охранники начали расходиться в разные стороны, а Буси-Ян, стоял на одном месте и кричал, чтобы они вернулись к нему. Пока они находились в растерянности, люди Джимми смогли разгромить охрану таким же жестоким способом, какой использовала накануне Система по отношению к бедным невинным народным массам.
\par
Сам Буси-Ян не был убит. Вместо этого банда позволила ему насладиться зрелищем, как один за другим умирали его люди. Его заковали в цепи и вывели на улицу, чтобы нищие могли отомстить ему любыми способами, которые посчитали бы лучшими. Было очень приятно наблюдать, как Буси-Ян просит прощения.
\par
«Не позволяйте им причинить мне боль! Я сделаю всё, о чём вы меня попросите. Пожалуйста… Я не хочу умирать!» - умолял он.
\par
«О, не волнуйся» - успокоил его Робин,
\par
«Ты не умрёшь слишком быстро. Я полагаю, что они придумают какой-нибудь меленный способ истязать тебя, чтобы ты мог вкусить все удовольствия нескончаемой боли.»
\par
Буси-Ян в ужасе уставился на него.
\par
«Любая смерть слишком хороша для тебя» - рассмеялся Робин ему в лицо.
\par
Оставив Буси-Яна на улице объясняться с народом, Робин и остальные вернулись в своё убежище.
\par
Робин испытал облегчение от того, что всё сработало так, как надо, и никто с его стороны не погиб. Он получил множество похвал и слов благодарности.
\par
Внезапно он стал героем — до сих пор ему никогда не оказывали столько внимания, как сейчас. Им досталось много еды, которую можно было раздать всем остальным людям. Затем, они смогут забрать следующую партию припасов, которая должна поступить примерно через пару недель. Им оставалось истребить лишь немногочисленную охрану Системы, и тогда люди могли бы использовать здание штаба как временное укрытие.\\
\par
Джимми и Робин стали хорошими друзьями. Они тесно сотрудничали, следя за тем, чтобы ни один человек не был обойдён вниманием и заботой, и чтобы никто никогда не оставался голодным. Многим казалось, что эта хорошая жизнь была просто сном, а после пробуждения они обнаружат, что Система всё ещё здесь. Впервые за долгое время люди почувствовали себя счастливыми. Они могли свободно ходить по улицам, не опасаясь за свои жизни. В целом, повсюду царила радостная атмосфера, а витавшая в воздухе гнетущая печаль, наконец, исчезла.
\par
Появились планы массовой застройки и реорганизации улиц. Для выполнения тяжёлых работ больше не требовался рабский труд — теперь, вместо него, каждый горожанин вносил свой вклад, делая окрестности более комфортными для жизни людей.\\
\par
Робин был счастлив. Наконец-то он понял, как использовать свои способности на благо людей. Он чувствовал себя ответственным за то, чтобы они никогда ни в чём не нуждались. Он дал им надежду, когда она была нужна им больше всего — вовремя, ещё до того, как они потеряли веру в лучшую жизнь, которую можно было бы назвать достойной после свершившейся ядерной катастрофы.\\
\par
Но однажды к людям пришло печальное осознание — несмотря на события, изменившие их жизни к лучшему, с ними всё ещё оставались последствия взрыва бомбы.
\par
Случились первые роды. Ребёнок родился мёртвым и с первого взгляда было заметно отклонение в его общем развитии. Всем было ясно, что в их мире он не смог бы прожить достаточно долго. Для этого ему требовался специальный уход и лечение в современной, полностью оборудованной больнице.
\par
Слёзы катились по щекам Робина, когда он в отчаянии пытался утешить убитую горем женщину. Если бы он был способен исцелять живых и возвращать жизни мёртвым! Но это было ему не по силам. Если Бог действительно любит своих детей, тогда откуда взялось всё это зло? Эти несчастные люди пережили боль и лишения, они потеряли близких. Они не заслужили такого — постоянно бояться, что больше никогда не смогут иметь свои семьи.
\par
Робин проклинал себя. Это было то, чего он не учёл. Если бы он это сделал, то мог бы, по крайней мере, предупредить их о высокой вероятности того, что дети родятся мёртвыми или будут иметь отклонения.
\par
«Не вини себя, Робин» - утешал его профессор,
\par
«Я никогда не думал, что подобное может случиться. Как и любой человек у вас, мы все здесь были взволнованы возможностью увидеть рождение ребёнка в лучшем, созданном тобой, мире. Ты в этом не виноват. Никто не виноват, за исключением правительств, развязавших войну.
\par
«Профессор Фергер, я не могу ничего с этим поделать, и вряд ли смогу забыть заплаканное лицо убитой горем женщины. Она так сильно хотела ребёнка, что могла неделями, без остановки, говорить о нём. Знаете, её муж и их первенец были сожжены в тот чёртов день, когда сюда прибыл Буси-Ян. Теперь вы понимаете, почему ребёнок был для неё так важен? Я чувствую себя виноватым в том, что не подготовил её к худшему. Люди стали зависеть от меня — они хотят, чтобы я помогал им стать счастливыми… Я чувствую, что подвёл её.»
\par
Робин тяжело вздохнул и откашлялся. Когда он говорил, его голос дрожал.
\par
В расстроенных чувствах, женщина прижалась к нему поближе.
\par
«Извини, Робин. Ребёнок был единственным напоминанием о моей семье. Мне следовало морально подготовиться к такому исходу — до войны мне не раз случалось смотреть кино о взрыве атомной бомбы в Хиросиме.»
\par
Она умолкла, и затем снова горько зарыдала.
\par
Робин с заботой поглаживал её волосы и крепко обнимал, изо всех сил пытаясь вернуть ей веру в себя, чтобы она снова могла почувствовать себя желанной и любимой.
\par
«Тише, тише… В том, что произошло, нет вашей вины. Я знаю, вы любили свою семью. Не думаю, что им понравился бы ваш теперешний несчастный вид. Вы обязаны заботиться о себе — они бы этого очень хотели, я в этом не сомневаюсь».
\par
Через пять минут она успокоилась и вытерла слёзы с глаз.
\par
«Благодарю за слова утешения. Пожалуйста, прости меня за эту слабость. Я не хочу быть твоей обузой — должно быть, у тебя много других, неотложных дел».
\par
«Вовсе нет! Мне бы хотелось, чтобы вы, именно вы, стали счастливой! Неужели вы не понимаете, что все мои помыслы направлены лишь на то, чтобы сделать жизнь каждого человека лучше — уютнее и счастливее? Я не желаю чтобы кто-то печалился, для меня это очень важно!»
\par
«Ты очень добр! Одному Богу известно, как могли бы сложиться наши судьбы, если бы не твоё появление здесь! Теперь иди, продолжай творить свои благие дела... Мне стало намного лучше. Спасибо тебе!»
\par
Поцеловав её на прощание в лоб, Робин ушёл.
\par
Он несколько часов бродил в одиночестве, размышляя о том, что создал. Окружавшие его люди уверовали, что он — Спаситель, посланный сюда, чтобы спасти их от опасной Системы. Они поклонялись земле, по которой он проходил, и не желали замечать в нём ни единого недостатка! Но это было заблуждением! Так не должно быть! Он знал, что не являлся ни Богом, ни святым.\\
\par
Он превратился в орла и взмыл высоко в небо, кружа вокруг центральной части города и внимательно следя за тем, чтобы все горожане были счастливы и довольны жизнью. Для местных жителей он стал олицетворять собой справедливость и порядок, и за ним всегда оставалось последнее слово при вынесении судебных решений в тех редких случаях, когда совершались какие-нибудь преступления.\\
\par
В тот день у Джимми был день рождения — ему исполнилось тридцать два года. Однако ощущение постоянной тревоги и чувство ответственности за каждого члена банды оставили на нём свой заметный отпечаток — в действительности он выглядел гораздо старше пятидесяти. Его лицо стало морщинистым, а волосы поседели.
\par
Трудная жизнь заставила Джимми осознать, насколько легкомысленно он относился к удобствам, которые имел, будучи высокооплачиваемым журналистом до начала войны. Он потерял жену и обоих детей вскоре после того, как была сброшена первая атомная бомба. Если бы у него была возможность вернуть время назад, до начала войны, он потратил бы все свои сбережения на подготовку чего-то похожего на ядерный бункер и постарался бы узнать как можно больше о выживании. Он занимал ту же позицию, что и большинство людей тридцатых годов, зарывавших свои головы в песок и не позволявших себе поверить в близость и неизбежность войны.
\par
В качестве подарка на день рождения Робин решил объявить его новым Лидером.\\
\par
«Робин, нет! Мне этого не нужно! Я уверен, что здесь все согласятся со мной. В любом случае, мне осталось жить не так уж и долго… Разве ты не заметил россыпь пятен на моей коже? Нет никаких сомнений — это рак. Пожалуйста, оставайся Лидером сам! Зачем тебе отдавать кому-то свои полномочия? Ты же не собираешься покинуть нас?»
\par
Наступила тишина. Поскольку ответ так и не прозвучал, Джимми решил повторить вопрос:
\par
«Ты не уйдёшь от нас? Ты не можешь, ведь ты нужен нам! Что нам делать, когда сюда вернётся вооружённая до зубов Система? Чтобы выжить, нам необходимы твои способности. Нам нужны твои знания, чтобы строить новые дома и новые генераторы. Предстоит ещё очень много работы и для этого нам нужен ты.»
\par
Тронутый сказанным Робин потерял дар речи. Он попытался произнести хотя бы слово, но закашлялся и не смог. Что он мог сказать этим встревоженным людям, которые так сильно зависели от него? Он знал, что они хотели услышать его «нет» — что он не собирается покидать их. Возможно, ему следовало рассказать о себе всю правду, и тогда они бы поняли, что когда-нибудь ему придётся уйти. Но у него не хватало духу сказать им обратное: этот день должен был стать радостным событием, а вместо этого они грустили.
\par
«Нет! Конечно, я не оставлю вас! Я всегда буду думать о вас!»
\par
Все развеселились, услышав несколько тщательно подобранных слов Робина. Он изобразил на лице улыбку, не в силах показать им, что чувствует себя несчастным. Он смеялся и шутил со всеми, делая вид, что ему удалось их одурачить.\\
\par
«Профессор Фергер, я не хочу возвращаться! Вы же видите — эти люди полностью зависят от меня. Я знаю — я виноват в этом — я не должен был становиться настолько важным для них. Но что случится, когда я уйду? Вы же слышали — они боятся ответных действий Системы, когда она поймёт, что произошло здесь, в Саутгемптоне. Эти события не за горами, поскольку обратно к ним не вернулись уже девять кораблей снабжения!»
\par
«Робин,» - начал успокаивать его профессор,
\par
«Они больше не нуждаются в тебе. Всё, что им нужно от тебя — это защита. Ты почти полностью, своими руками, создал их новый мир. Они принимают тебя, как должное. Ты даёшь им всё, что нужно, порождая в них лень!»
\par
«Нет, вы ошибаетесь. Они видят во мне не только того, кто предоставит им всё, чего они захотят. Они любят меня как человека. Они любят! Та женщина, вчера, была очень благодарна мне за то, что я потратил часть своего времени, чтобы утешить её. Только не говорите, что они не видят моей эмоциональной поддержки.»
\par
«Ты забываешь о главной цели своего пребывания там. Помни, что в первую очередь мы хотим предотвратить ядерную войну. Если у нас получится, то не случится ни боли, ни невзгод, через которые прошли все эти люди. Они не потеряют своих мужей и детей, а те, кого убила Система, останутся жить.»
\par
«Вы сами произнесли это слово — «ЕСЛИ»! Понимаете? Вы не уверены в том, что мы сможем добиться успеха! А что, если мы не добьёмся? Тогда я брошу этих людей, которые будут страдать без моей помощи и наставлений. Нет, уходить отсюда слишком рискованно. У меня нет уверенности и гарантий, что это самый мудрый шаг.
\par
«Без твоих экстраординарных способностей у нас нет шансов на успех. Ты справишься с заданием раньше, чем мы его обдумаем. Помнится, ты всегда хотел попробовать сделать то, что казалось невозможным. А теперь скажи мне, что случилось с тем Робином, которого мы когда-то знали?»
\par
Тогда всё было по-другому и у меня не было никаких обязанностей. Я не могу найти в себе силы бросить этих беспомощных людей. У меня не будет никаких оправданий, если я повернусь к ним спиной.
\par
«А что насчёт нас? Что насчёт миллиардов других ни в чём не повинных людей планеты? Разве ты не отворачиваешься от нас? Я думал, что ты любишь нас, Робин.»
\par
«Это нечестно! Вы используете эмоциональный шантаж. Вы знаете, что я люблю всех вас… Вы не должны использовать это, чтобы заставить меня вернуться. Вы злоупотребляете моей любовью и доверием к вам.»
\par
«Просто вернись к нам» - воззвал профессор,
\par
«Дай нам хотя бы один шанс пережить катастрофу. Разве ты не понял, что нас нет в Саутгемптоне будущего? Видимо, мы не смогли пережить войну. Может быть, если ты вернёшься, мы сможем продолжить жить и умрём в старости.»
\par
«О, я не знаю, что делать!»
\par
«Подумай вот о чём, Робин. Мы согласны с тем, что всё, частью чего ты являешься, находится в будущем. Тогда, если мыслить логически, можно рассматривать это как сон. Ты должен пробудиться от этого сна и донести до людей те знания, которые приобрёл. Таким образом сон перестанет существовать! Ты обязан помочь нам хотя бы попытаться, раз уж на то пошло. Ты сделаешь это для меня? Я не буду больше ничего говорить — оставлю тебя обдумывать сказанное. Скажи мне, когда примешь решение.»\\

\chapter{ВОЗВРАЩЕНИЕ ДОМОЙ}
\noindent\parП{\scriptsizeРОФЕССОР ВМЕСТЕ СО СВОИМИ СТУДЕНТАМИ СТАЛИ ИСПЫТЫВАТЬ БЕСПОКОЙСТВО, ПОСКОЛЬКУ ОНИ НЕ ПОЛУЧАЛИ ОТ РОБИНА} никаких известий. Всё, что они могли видеть — это как он в облике орла бесцельно кружит над землей.
\par
Задул сильный ветер. Он затягивал в себя дорожную пыль, а также различный мусор, находившийся у него на пути. Ветер сталкивался с препятствиями, и было отчётливо слышно, как его ноша ударяется о стены ближайших построек.
\par
Люди исчезли с улиц и повсюду наступила полная тишина. Пролетая над городом, Робин больше не слышал ни смеха, ни счастливых восклицаний.
\par
Орёл спустился на землю и снова принял свой истинный человеческий облик. Он зашёл в дом — в своё собственное жилище. Это было единственное место, где он мог побыть в одиночестве. Пока не возникнет крайней необходимости, никто не станет беспокоить его, когда он находится здесь.
\par
Он залился слезами, находясь в смятении чувств и не зная, что делать дальше. Глубоко внутри себя, в своём сердце, он знал, что рассуждения профессора были правильными. Но здесь он чувствовал себя своим, он был частью этого мира. Это был его мир — он создал его своими руками. И люди тоже были его. Робин воплотил новую форму правления в то, что мог бы считать сутью абсолютной демократии. Здесь каждый имел право голоса по всем вопросам, будь то возведение новых построек или чья очередь убирать мусор. Любые решения, не исключая самых банальных, тщательно обсуждались всеми заранее. Да, это было именно то, что ему удалось создать. Его мечта рухнет, если ему придётся уйти.
\par
С этим местом у него связано много приятных и счастливых воспоминаний. Здешнее сообщество стало невероятно сплочённым: счастье и горести одного человека разделяли все окружавшие его люди.
\par
«Профессор, я знаю, что вы правы. Я вернусь к вам. Жаль, что больше никогда мне не придётся увидеть такого сообщества, как здесь. Когда я вернусь, мне будет трудно приспособиться к реальному миру. Уйти отсюда — это всё равно, что оставить часть себя позади.»
\par
«Я рад, что ты принял решение. Понимаю, как трудно оно далось, но, поверь мне, я не стал бы просить тебя уйти, если бы чувствовал, что не прав!»
\par
«Мне можно пойти и попрощаться с ними?»
\par
«Нет» - ответил профессор после короткого раздумья.
\par
«Если ты не скажешь им ничего, то они могут подумать, что ты погиб во время полёта. На случай, если ты не заметил… Поправь меня, если я ошибаюсь, но этот ветер выглядит так, будто вскоре за ним последует буря?»
\par
«Хорошо. Думаю, что вы правы. В таком случае, они не подумают, что я трусливо сбежал. Тогда, может быть, они будут вспоминать обо мне с теплотой.»\\
\par
Теперь, когда было решено, что Робин вернётся в свой настоящий мир, всё сводилось к простым вопросам, главным из которых был вопрос «каким образом?» Пройдя сквозь время, он совершил путешествие в этот мир из своей спальни на Девоншир-роуд. Было бы логично вернуться обратно, отправившись на Девоншир-роуд и точно определив положение своего дома. Это было всего лишь идеей, поскольку никто из них не знал ничего о таких путешествиях. Однако это была единственная вещь, над которой им пришлось работать.
\par
Робин бродил по своему временному пристанищу, осматривая его в последний раз, прежде чем покинуть навсегда. Затем, подхватив пальто, он быстро вышел наружу — на его глазах заблестели слёзы. Ветер дул с удвоенной силой, завывая, словно от гнева, и над землёй, вдаль, неслись чёрные грозовые тучи. Грянул гром и заглушил своим раскатом шум проливного дождя. Вслед за ним, небо разрезала молния, осветив своим ярким холодным светом ночной небосвод. Вспышки, следовавшие одна за другой, освещали ему дорогу в кромешной тьме, царившей на пустынных улицах. Разразившаяся буря напоминала ту жуткую ночь, в которую он растерзал мистера Уайтли, отца Элисон.
\par
Накатившие воспоминания оказались полезными — он припоминал как, находясь в облике птицы, возвращался от дома Ника обратно, на Девоншир-роуд, и пытался отыскать таким образом правильный путь.
\par
Поиски дороги назад были похожи на прогулку во сне — как будто некая магнетическая сила влекла его в нужное место. Прибыв туда, он обратился к профессору Фергеру:
\par
«Я тут. Что мне делать дальше? Как вернуться к вам?»
\par
«Не могу сказать… я не знаю. Нам нужно, чтобы ты держал нас в курсе того, что видишь и чувствуешь. Мы можем лишь посоветовать.»
\par
Робин уселся на землю в позе лотоса — скрестив ноги. Мысленно, он вернулся к тому, как выглядела его комната, а затем погрузился в очень глубокий транс.
\par
«Я хочу вернуться в свою спальню и проснуться в своей постели» - повторял Робин снова и снова. Он чувствовал, что его тянет наверх.
\par
«Робин, иди ко мне! Мои руки протянуты к тебе — держись за них крепче, и я смогу привести тебя обратно, к нам.»
\par
«Но я не могу! Я же ничего не вижу!»
\par
«Не видишь? Что это значит? Что мешает тебе увидеть?»
\par
«Разве вы не заметили? Прямо передо мной — яркий белый свет. Его блеск заставляет меня прикрывать глаза — я не могу смотреть прямо на него.»
\par
«Этот белый свет — он вокруг тебя или только прямо перед тобой?»
\par
Робин осмотрелся по сторонам, и снова заговорил:
\par
«Позади меня — там, откуда я пришёл — он есть. Слева и справа от меня, кажется, виднеется множество тёмных троп. Какую из них мне выбрать?»
\par
«Робин! Что бы ты ни делал, ни в коем случае не уходи с пути яркого света. Я не могу объяснить почему, но инстинкт подсказывает мне что ты должен следовать к нему, иначе ты потеряешь нас навсегда. Твой разум заблудится и будет скитаться до конца веков.»
\par
«Но мне не нравится этот путь — от него исходит жар, а свет режет мои глаза. Мне придётся держать глаза открытыми, если я останусь на нём.»
\par
«Робин, прикрой веки, но не смыкай их полностью, и слушай мой голос. Представь, что голос — это верёвка, обвязанная вокруг твоей талии, которая притягивает тебя ко мне. Просто держись за неё.»
\par
Профессор Фергер продолжал взывать к Робину своим ровным, монотонным голосом:
\par
«Робин, иди ко мне.»
\par
Пока Робин шёл по освещённой тропе, он слышал голоса, манившие его идти к ним. Их руки протянулись к нему из глубин тёмных троп. Они издавали гул, и, вдалеке, он услышал сладкоголосое пение ангела, походившее на воплощённую гармонию. Оно было таким притягательным, что ему сразу же захотелось встретиться с существом, обладающим таким прекрасным голосом. Слушать его было необычайно приятно — он успокаивал и расслаблял. Робин почувствовал себя так, словно он плывёт по воздуху; ему показалось, что в мире больше не осталось забот, и что он очутился в раю.
\par
Он начал двигаться в сторону голоса — такого чистого, как вода, стекающая по склону горы с тающей ледяной шапки её вершины. Другие голоса уговаривали его подойти ближе:
\par
«Иди… иди… ты уже почти на месте. Как только придёшь сюда, сможешь наслаждаться красотой этого сладкого чистого пения целую вечность.»
\par
Он последовал за покачивающимися руками, которые, как ему казалось, подводили его всё ближе к пению. Этот голос, подобно голосу мифической сирены, гипнотизировал и притягивал его к себе. Его глаза были широко открыты, но больше не болели от яркого света — боль осталась где-то далеко позади него.
\par
«Робин! Где ты? Что случилось со светом?»
\par
«Не нужно слов, профессор Фергер… просто послушайте это пение — какое оно чистое. Обладательница этого голоса, должно быть, невероятно красива. Мне необходимо увидеть её!» - ответил Робин.
\par
Голос профессора становился всё тише и тише.
\par
«Робин? Робин!» - встревоженно воскликнул профессор Фергер.
\par
«Не позволяй этому пению, этой песне, обмануть тебя. В конце пути, по которому ты сейчас идёшь, никого нет. Это уловка, всего лишь отзвук, издаваемый другими блуждающими душами. Они — зло! Это силы тьмы, которые завладели тобой.»
\par
«За этим пением не может скрываться никакого зла, оно слишком прекрасно.»
\par
«Робин… не испытывай моё терпение. Просто слушай меня. Следуй за моим голосом обратно к яркому свету. Не обращай внимания на пение, если не хочешь потерять нас навсегда. Ты его не интересуешь, у него есть другие причины удерживать тебя там.»
\par
Голос профессора заставил пение прекратиться, оно отступило. Послышался смех ребёнка, потешавшегося над глупостью Робина:
\par
«Ты заблудился! Ты никак не можешь найти нужную дорогу. Не стоит тебе оборачиваться, это бесполезно — ведь ты уже целую вечность бродишь кругами. В каком направлении ты собираешься идти…? Видишь?! Ты и сам не знаешь!»
\par
«Робин!» - прокричал профессор,
\par
«Следуй за моим голосом, просто следуй за ним! Ты сразу поймёшь, когда пойдёшь в правильном направлении, потому что он будет казаться тебе громче. Я буду продолжать говорить с тобой — ты просто слушай и позволь мне направить тебя в нужную сторону.»
\par
Снова послышались голоса заблудших душ:
\par
«На твоём месте мы бы сдались. Бесполезно даже пытаться. Ты уже не сможешь сбежать, раз оказался здесь.»
\par
«Мне плевать на ваши слова! Я буду слушать только профессора! Он мой друг, а вы — нет.»\\
\par
Робин перестал обращать внимание на голоса заблудившихся душ и сосредоточился на речи профессора. Он уже почти добрался до тропинки, но случайно свернул не туда, и голос профессора Фергера снова начал от него отдаляться. Тогда Робин отправился в обратную сторону, пока слова профессора снова не зазвучали громче. Он был уже недалеко — в кромешной мгле можно было различить слабый проблеск. Подойдя ближе, Робин увидел направленный вниз свет, словно луч утреннего солнца, пробившийся сквозь пыльное, затуманенное окно в каком-то старом, заброшенном особняке…
\par
От неожиданности Робин подпрыгнул на месте. Волосы на его затылке встали дыбом, когда он услышал позади себя пронзительный визг. Он оглянулся, но не смог ничего разглядеть. Позади раздавались глухие стуки, похожие на шаги, идущие в том же темпе, что и он сам.
\par
«Ты не должен отвлекаться на это» - призвал профессор.
\par
«Они не смогут навредить тебе. Сконцентрируйся на мне, на моём голосе.»
\par
Робин продолжил идти тем же размеренным шагом, как и прежде. В тумане явственно различался яркий свет, похожий на тот, что обычно бывает виден в конце туннеля. Теперь он был так близко, что мальчику показалось, будто можно протянуть руку и дотронуться до него.
\par
«Смотрите, профессор — я почти у цели! Осталось совсем чуть-чуть. Скоро я вернусь к вам в целости и невредимости — в то место, которому принадлежу.»
\par
«Всё правильно, Робин. Продолжай идти, пока не дойдёшь до тропы. Не отвлекайся на прочие мысли, сконцентрируйся на моём голосе.»\\
\par
Наконец он оказался на тропинке. Яркий свет заставил его на мгновение прищуриться, пока глаза привыкали к его блеску. Робин посмотрел себе под ноги и увидел там самую странную картину на свете. Вокруг клубился тонкий бледный туман. Когда в нём появился просвет, Робин заметил, что под его ногами нет никакой тверди. Он шёл через пространство, в тёмно-синем и фиолетовом море, которое, казалось, не имело границ, уходя в бесконечность.
\par
«Сколько ещё мне осталось пройти? Я уже близко?» - спросил Робин, и его голос выдавал сильную усталость.
\par
Последние недели стали огромным испытанием для его психических сил. Он был утомлён — психически и физически. Им было потрачено немало энергии, он использовал этот ресурс по максимуму, чтобы обеспечить своё выживание. Его шаг становился всё медленнее, а ноги волочились — путь домой казался теперь бесконечным.
\par
«Я больше не могу идти, я слишком устал. Я хочу спать — хочу лечь на мягкую постель, опустить на неё свою голову и хорошенько отдохнуть, проспав века.»
\par
«Нет, ты не хочешь. Превратись в орла и тогда ты сможешь быстро преодолеть остаток своего пути.» 
\par
«Робин снова послушался совета профессора. Он превратился в орла и позволил крыльям нести себя, а ногам дал заслуженный отдых.»
\par
Голос профессора стал очень громким, мальчик уверенно приближался к границе бесконечного времени и своего настоящего мира — тысяча девятьсот восемьдесят седьмого года. Он стал ускоряться, теряя контроль над своими действиями; его неудержимо тянуло к профессору.
\par
«Я уже почти на месте, собираюсь снова превратиться в человека.»
\par
Внезапно в спальне Робина произошла яркая вспышка белого света и мощнейший грохот — путешествующая сущность вывалилась из одного временного промежутка и стремительно ворвалась в другой. С огромной силой он упал на кровать и тут же потерял сознание.\\
\par
Профессор позвал врачей, чтобы тщательно обследовать Робина. С мальчиком всё было в порядке. Они убрали капельницу и составили для Робина специальный диетический лист. Он проспал весь день и всю ночь, а на следующий день проснулся только ближе к обеду. Робин принял ванну — первую за месяц, в которой так отчаянно нуждался. Затем съел лёгкоусвояемую пищу — варёную белую рыбу и пару отварных картофелин.
\par
Все баловали его, дарили любовь и ласку и заботились о том, чтобы у него было всё, о чём он просил. Он ждал, когда же профессор затронет в разговоре вопросы, касавшиеся его жизни в другом мире. Лишь вечером профессор решил, что им можно начать обсуждать, что происходило с Робином в мире будущего. Они говорили несколько часов, в мельчайших подробностях описывая Систему, её роль в управлении британским народом и о том, каким был мир до того, как Робин уничтожил этот руководящий орган, совершив не что иное, как революцию.
\par
Профессор Фергер достал досье с записями, которые он сделал. В них подробно описывались передвижения Робина, а также приводились выдержки из документов и газетных статей, которые им попадались. Его почерк был неразборчивым. Большинство штрихов казались одинаковыми, и тот факт, что он записывал свои наблюдения на немецком, не помогал делу — студентам было ещё труднее их читать.
\par
«Мы можем начать с того, когда Робин впервые рассказал нам о том, что увидел вокруг?» - спросил профессор Фергер.
\par
«Я думаю, что, начав с самого начала, мы сможем добавить любые дополнительные детали, на которые не обратили внимания прежде.»
\par
«Это неплохая идея» - ответил Робин,
\par
«Но должны ли мы пройти через всё это сегодня вечером? Я всё ещё очень устал и чувствую слабость. Буду признателен, если мы закончим пораньше. Мне кажется, что у нас нет причин для спешки. Если мы хотим это сделать, то нам следует потратить немного времени и убедиться, что всё сделано хорошо. Иначе всё пропадёт впустую.»
\par
«Хорошо, мы всего лишь обсудим ту область, откуда ты впервые сообщил нам о том, где находишься» - согласился профессор,
\par
«Я думаю, что первое, с чего нам стоит начать — это получить сведения о послевоенном Саутгемптоне.»
\par
«Я мало что могу добавить к тому, что вы уже видели сами. Когда я впервые побывал там, это место ассоциировалось у меня со смертью. Оно было настолько убогим, что я не отправил бы туда даже своего злейшего врага. Большинство многоэтажных бетонных домов представляли собой полуразвалившиеся останки. Я не ходил туда, в руины, после наступления темноты — мне было жутко. В этом месте обитали призраки, я чувствовал печаль потерянных душ — блуждающих, которым некуда было пойти.»
\par
«Город был пустым, а местность - бесплодной. Там ничего не росло, кроме мелких растений… таких, как мхи. Когда я в первый раз очутился там, мне и в голову не пришло, что причина отсутствия растений на самом деле кроется в загрязнении почвы радиоактивными осадками. Однако закаты и рассветы выглядели очень эффектно. Никогда ещё я не видел ничего столь же впечатляющего: небо пылало ярко-красным, сливающимся с остальными цветами спектра. Отражение неба медленно переливалось на поверхности моря и исчезало в белой пене разбивающихся о берег волн. Само море в течение дня, казалось, было слегка прикрыто тонким покровом жёлто-серого тумана. Не знаю почему, но было именно так.»
\par
«Самое страшное было по ночам, ведь они были такими холодными. Мне было трудно спать. Каждое утро я просыпался с болью в суставах и скованностью во всём теле. Днём солнце почти всегда было закрыто пасмурным небом…»
\par
«Завтра, или когда у нас будет свободное время, я дам вам более подробное описание всех значимых мест, в которых я побывал, и расскажу, где они расположены.»
\par
«Хорошо. Это может может оказаться весьма полезным, Робин» - ответил профессор Фергер. Он торопливо составлял записи в своей обычной неряшливой манере.
\par
Робин зевнул. Он надеялся, что профессор поймёт его тонкий намёк и предложит остановиться прямо сейчас, а продолжить — завтра. К великому огорчению Робина, профессор этого так и не сделал. "Бабушкины" часы пробили одиннадцать раз, а затем продолжили свой ровный и монотонный ход. До этого момента Робин никогда не замечал, насколько громко они тикали. Все его мысли были поглощены мечтами о том, чтобы поспать в удобной постели, пристроив на подушке свою усталую голову. В тот момент ему хотелось этого сильнее всего остального. Чем больше он думал об этом, тем тяжелее становились его веки. Медленно, но верно, они начали смыкаться, голоса становились всё тише, пока окончательно не перестали слышаться…
\par
Следующим, что помнил Робин, было то, как его легонько встряхнули.
\par
«Робин, проснись. Иди в свою кроватку» - тихо прошептала Элисон ему на ухо.
\par
«Я не хочу. Дайте мне поспать здесь.»
\par
Робин закрыл глаза и попытался снова уснуть, но это было бесполезно. Элисон продолжала говорить:
\par
«Пойдём, Робин. Я не позволю тебе спать здесь. Если тебе нужен хороший отдых, то перебирайся в спальню. Там тебя никто не потревожит и ты сможешь спокойно поспать до утра.»
\par
"Спящая красавица" вяло зашевелилась, потянулась и осторожно открыла усталые глаза.
\par
«Ох, как я устал! Я не хочу вставать, но, похоже, ты не оставляешь мне выбора. Спокойной ночи, Элисон!»
\par
Робин обвил её руками и с нежностью поцеловал в щёку.\\
\par
Гарри, Элисон и Мэтью сидели в гостиной и пили чай в ожидании, когда проснётся Робин. На часах было почти одиннадцать.
\par
«Итак… Как долго ты пытался скрыть тот факт, что встречаешься с Элисон?» - саркастически спросил Мэтью.
\par
Гарри поперхнулся и слегка расплескал свой чай. Элисон удивлённо посмотрела на Гарри — она явно не ожидала подобного вопроса.
\par
«О чём ты, Мэтью? С чего ты взял, что между мной и Элисон что-то есть?»
\par
«Не пытайся казаться невинным, Гарри — я не настолько глуп! Помни, я знаю тебя гораздо дольше, чем Элисон. Ты хочешь, чтобы я рассказал ей о нашем уговоре, который состоялся не так давно?»
\par
«Мэтью, ты просто завидуешь! Всё потому, что это я встречаюсь с ней, а не ты.»
\par
Элисон надоело слушать, как они вздорят, поэтому она встала и вышла из комнаты, оставив их обоих наедине друг с другом.
\par
«Нет, Гарри! Тебе не нужно идти вслед за мной! Ты должен уладить свои разногласия с Мэтью!»
\par
Как только Элисон ушла, Гарри закричал на Мэтью:
\par
«Ну и чего ты хотел добиться? Тебе хорошо известно, что Элисон слишком близко принимает к сердцу каждое сказанное слово и не выносит, когда кто-то ссорится! Одному Богу известно, о чём она подумала, когда ты упомянул об уговоре.»
\par
«Гарри, послушай меня внимательно и не перебивай! И не пытайся коверкать мои слова! Мы с тобой оба прекрасно знаем, что ты не способен на продолжительные отношения с кем-либо. И я подозреваю, что до сих пор у вас не было тесного физического контакта, а ведь именно он привлекает тебя в отношениях сильнее всего остального! Но ты не получишь ничего такого ещё очень долго! Было бы лучше, если бы вы оба посмотрели правде в глаза и закончили всё прямо сейчас! Просто будь с ней честен, и тогда ты избавишь её от будущих страданий!»
\par
«Ладно, Мэтью. Ты сказал достаточно! Но откуда такая уверенность в собственной правоте? Почему ты решил, что я не могу относиться к ней серьёзно — не так, как это было с моими прежними подружками? Твои предположения абсолютно ничем не обоснованы!»
\par
«Не пытайся обманывать себя! Ты знаешь, что я прав, просто не хочешь в этом признаваться!»
\par
«Ты о чём вообще? Для меня совершенно ясен ход твоих мыслей! Ты ревнуешь и не можешь смириться с мыслью о том, что Элисон выбрала меня, а не тебя! Боже! Откуда в тебе столько ребячества?»
\par
«Гарри, ты опять лукавишь! Я знал, что честным путём тебя не переубедить! В таком случае, кто тогда держал пари, что начнёт встречаться с Элисон до конца семестра? Извини, но мне пришлось вспомнить об этом, иначе ты не стал бы слушать меня. Так что поверь моим словам! Если вы не закончите свои отношения как можно раньше, то в конце концов ты заставишь её страдать!»
\par
«Чего вы так раскричались?» - спросил внезапно появившийся Робин, с удивлением осматривая гостиную.
\par
«А где Элисон? Мне показалось, что сегодня она собиралась быть тут.»
\par
«Её нет» - быстро ответил Мэтью,
\par
«Она была здесь недавно, но ушла, пообещав скоро вернуться. Она просила нас приготовить тебе завтрак, если ты проснёшься до её возвращения. Что будешь есть?»
\par
Ему не хотелось говорить с Робином об истинной причине ухода Элисон.
\par
«Нет, я знаю, что ты говоришь мне неправду» - возразил Робин,
\par
«Это написано на лицах у вас обоих. Что случилось? Вы расстроили её, не так ли?»
\par
Гарри и Мэтью притихли. Почему Робин не поверил их словам? Может быть, они сглупили, полагая, что смогут одурачить его? Для них было непросто осознавать, что они не могут скрывать от Робина что-либо, и что мальчик обязательно распознает обман.
\par
«Знаете ли, я ведь не дурак. Меня не было здесь какое-то время, но я прекрасно осведомлён о том, что происходит между Элисон и Гарри, и я знал, что они встречаются ещё до того, как посетил мир будущего.»
\par
Робин ненадолго умолк.
\par
«Гарри, мне не нравятся твои свидания с Элисон. Ваши отношения закончатся тем, что ей будет очень больно. Я всегда считал, что в твоих действиях больше здравого смысла…»
\par
Робин замолчал снова. Гарри и Мэтью были ошеломлены этим внезапным заявлением. Гарри выглядел взволнованным — это было несправедливо. Его осуждали, причём очень несправедливо, по прошлым «заслугам». Что эти двое могли знать об его истинных чувствах к Элисон?
\par
«Я не верю своим ушам! Почему вы не одобряете то, что я — именно я из всех людей — встречаюсь с Элисон? Вы даже не представляете, как я уважаю её. Никогда прежде я не испытывал ни к кому таких сильных чувств, как к ней. Я люблю её. Пожалуйста, поверьте — я люблю её и не хочу причинять ей боль.»
\par
«Ты увлечён ею» - ответил Робин,
\par
«Но ты не любишь её в истинном смысле этого слова.»
\par
Мэтью кивнул, соглашаясь с ним. Мальчик пристально посмотрел в глаза Гарри и неспешно заговорил:
\par
«Скажи правду. В глубине души ты знаешь сам, что не любишь Элисон. На самом деле тебе хочется близости, хочется быть её первым, потому что тебе известно, что она девственна. Большинство мужчин думает точно так же, как ты, хотя многие не отдают себе в этом отчёта. Они находят нечто привлекательное в свиданиях с девственницами. Гарри, я прав! Совершенно бесполезно пытаться отрицать всё это. Я не собираюсь стоять в сторонке и безучастно смотреть, как ты доводишь Элисон до слёз, ведь я очень сильно её люблю! Я не позволю этого ни тебе, ни кому-либо ещё! Никто и никогда не навредит ей, пока я буду присматривать за ней и защищать!»
\par
«Вы слишком придирчивы, понимаете?! Даже Робин завидует моему счастью.»
\par
«Хватит, Гарри! Ты окончательно запутался в том, что значит быть влюблённым и что значит любить кого-то, потому что ты заботишься о нём, как о друге.».
\par
Робин решительно отверг все обвинения Гарри относительно того, почему он не хотел, чтобы Гарри встречался с Элисон. Мэтью молчал — Робин хорошо постарался, заставив Гарри понять, что к чему. Определённо, на этот раз бестактность Робина сослужила добрую службу. Просто удивительно, что мальчик его возраста так хорошо понимал человеческую природу. Его доводы не являлись беспочвенными — в его словах было много правды, и Гарри понял это. Однако он был упрям и не хотел этого признавать. Они были правы, он не был подходящим человеком для Элисон — она заслуживала кого-то более чуткого и заботливого. Гарри не выдержал и расплакался.
\par
«Я… я...»
\par
Его голос затих, когда он всхлипнул. Его лицо было плотно закрыто руками, ему было слишком стыдно поднять глаза. Гарри считал слёзы проявлением слабости и, одновременно, эффективным способом завоевать сочувствие окружающих.
\par
Мэтью поднялся на ноги и уже был готов подойти к Гарри, но тут к нему обратился Робин:
\par
«Мэтью, оставь его в покое. Нам лучше пойти на кухню, пока он не придёт в себя».
\par
Мальчик встал и выпроводил Мэтью из гостиной, оставив Гарри всхлипывать в одиночестве.
\par
«Мы не должны оставлять его плакать наедине с самим собой. Всем нам иногда нужно, чтобы кто-то был рядом, когда мы расстроены. Разве тебе не кажется, что ты был к нему слишком строг? - прокомментировал Мэтью, обжаривая для Робина несколько пышек. Он открыл холодильник и поискал сыр и масло.
\par
«Нет, мне не кажется. Это для его же блага и, что гораздо важнее, мы спасём Элисон от новых душевных травм. Она и без них сторонится мужчин. Как ты думаешь, какие будут последствия, если она обожжётся, приблизившись слишком близко к Гарри? Это сломило бы её окончательно.»
\par
«Наверное, ты прав. Хочешь, я положу кусочек сыра на одну из пышек?»
\par
Робин кивнул. Мэтью аккуратно нарезал сыр. Он намазал маслом остальные пышки, а ту, что была с сыром, положил обратно в печь, чтобы слегка расплавить его. Он подал Робину завтрак с чашкой крепкого чёрного кофе.
\par
«Вернусь через минуту, мне нужно сходить в туалет.»
\par
«Мэтью, ты уверен, что ты идёшь в туалет, а не ищешь повод увидеться с Гарри? Оставь его в покое — ему нужно время, чтобы обдумать то, что мы сказали ему.»
\par
Пока Мэтью ходил в туалет, в дверь позвонили. Робин поставил тарелку на стол, вытер салфеткой рот и пальцы, и открыл дверь.
\par
«Привет, Элисон. Я думал, что у тебя есть ключи.»
\par
«Я забыла их здесь этим утром.»
\par
«Проходи на кухню — мы поболтаем с тобой, пока я завтракаю.»
\par
Робин взял её за руку и крепко сжал её ладонь в своей. Он слегка приобнял её и чмокнул в щёку. Они прошли на кухню и сели.
\par
«Элисон, что случилось? Не стесняйся, расскажи мне. Помни, я стараюсь заботиться о тебе, и мне совсем не нравится видеть тебя расстроенной. И пожалуйста, не обращай внимания на то, что я жую.»
\par
«Ох, я не знаю, что мне делать. Причина, по которой меня не оказалось здесь этим утром, когда ты проснулся, в том, что Мэтью и Гарри поспорили из-за меня. Мэтью сказал, что Гарри, в конечном счёте, только навредит мне. Это странно, но за последнюю неделю, или около того, мой внутренний голос подсказывал мне, что нам следует прекратить эти отношения, прежде чем они зайдут далеко. У меня просто не хватает духу сказать обо всём Гарри. Ему будет больно, если я скажу ему. Что ты думаешь, Робин?»
\par
Робин поставил тарелку в раковину и вымыл руки, вытерев их кухонным полотенцем. Он выждал, подумав, как лучше всего сказать ей о своих чувствах. Глаза Элисон остекленели. Она была явно расстроена, и он не хотел ухудшать её положение — или, если подумать, ухудшать его для Гарри.
\par
«Элисон, сегодня утром я говорил о тебе с Гарри. Он любит тебя, но, пойми его правильно, я думаю, он знает, что он не тот человек, который подходит тебе. Когда только ты вошла, я решил привести тебя сюда, чтобы ты не ходила в гостиную. Видишь ли, Гарри расплакался. Мы с Мэтью оставили его там, чтобы он не смущался.»
\par
«Я могу сходить и навестить его?»
\par
«Нет, не нужно. Пусть он сначала придёт в себя. Лучше подождать, пока он не заговорит первым.»
\par
«Интересно, куда запропастился Мэтью? Он сказал, что собирается в туалет…»
\par
Робин сделал паузу.
\par
«Убирает комнату наверху — это, конечно, отговорка. Вероятно, он услышал твой голос и, почувствовав себя виноватым, решил таким образом избежать встречи с тобой. Подожди, я пойду и попрошу его спуститься вниз.»
\par
Робин покинул кухню. Элисон поднялась и подошла к раковине, чтобы вымыть тарелки и чашки, оставленные, видимо, ещё со вчерашнего вечера. Было бы чудом, если бы накануне они вымыли за собой посуду. Ей стало грустно от того, чем всё закончилось, но, с другой стороны, она почувствовала облегчение, как будто с её плеч спала тяжёлая ноша.
\par
Оставалось только прояснить всю эту печальную историю с участием Гарри.
\par
Она не могла не чувствовать, что Робин и Мэтью были слишком строги к нему. Тем не менее, всё, что они сделали, это воплотили в жизнь то, что она сама, в конечном итоге, собиралась сделать. Конечно, в подходящее для этого время. Но это произошло сейчас, и ей нужно было морально готовиться к встрече с Гарри.
\par
Она никогда не задумывалась о том, что он может плакать. Это было странно, ведь он всегда казался ей таким сильным. Вот чего она так боялась – она молилась, чтобы он не сломался на её глазах,  разразившись потоком слез. Она не сможет справиться с этим и будет чувствовать себя виноватой.
\par
Как и предполагалось, Робин нашел Мэтью в своей спальне, заправлявшим постель.
\par
«Элисон здесь. Ты не хочешь спуститься к нам? Мне показалось, ты говорил, что собираешься только сходить в туалет.»
\par
«Ох... я не знал, что Элисон здесь. Я буду через минуту.»
\par
«Что случилось, Мэтью? Почему ты не смотришь на меня? Ты сердишься? Я не хотел быть слишком грубым с ним этим утром — честное слово, я не хотел.»
\par
«Нет, это не из-за тебя. До меня только сейчас дошло, почему я так кричал на него. Это было оттого, что я ревновал. Для меня была невыносима мысль о том, что Гарри так повезло встречаться с Элисон. Я думал не об Элисон, а о себе. Мне чертовски стыдно. Возможно, мы ошибались — откуда нам знать, что Элисон и Гарри не подходят друг другу. Гарри вполне мог быть искренним в своих чувствах по отношению к ней.»
\par
«Не говори так, Мэтью. Ладно, ты был неправ, но в то же время это к лучшему, поверь мне.»\\
\par
Мэтью закончил заправлять одеяла и последовал вслед за Робином. Когда они спустились вниз, Элисон уже не было. Помещение кухни сверкало чистотой.
\par
«Мы подождем их здесь. Они, должно быть, разговаривают и выйдут, когда закончат.» — сказал Мэтью, наливая воду в электрический чайник. Он достал чашки, чтобы приготовить по чаю для каждого. Они сидели молча и никто не проронил ни слова. Отчётливо слышалось тиканье настенных часов. Чайник зашипел, свидетельствуя, что в нём закипела вода.
\par
«Почему они задерживаются?» – спросил Робин. Мэтью пожал плечами в ответ. Он закончил заваривать напиток и выставил посуду на поднос.
\par
«Ладно, они, наверное, уже закончили обсуждать разрыв своих отношений. Самое время внести чай.»
\par
Робин встал с табуретки и последовал за Мэтью в гостиную. Мэтью постучал в дверь, осторожно удерживая поднос свободной рукой.
\par
«Можно нам войти? Я заварил свежий чай.»
\par
«Да, Мэтью, входи», — ответил Гарри.
\par
Элисон улыбнулась Робину, как бы говоря, что он был прав, и что она стала чуточку счастливее теперь, когда вопрос, наконец-то, прояснился. Мэтью поставил поднос на маленький журнальный столик, сделанный из соснового дерева, и раздал чашки.
\par
«Вам не кажется, что пора позвонить профессору?» - спросил Гарри,
\par
«Он просил связаться с ним, когда Робин проснётся и будет готов продолжить работу над записями.»
\par
Гарри отпил чай, осторожно дуя на него, чтобы не обжечь язык.
\par
Элисон встала и подошла к телефону:
\par
«Я позвоню ему и спрошу, хочет ли он, чтобы мы поехали в университет, или он собирается приехать сюда сам.»\\
\par
«Робин, как ты вообще попал туда, в ту послевоенную эпоху?» - спросил Гарри,
\par
«Что ты почувствовал, когда понял, что твой разум находится там?»
\par
Наблюдая за Робином, он много раз задавал себе этот вопрос. Путешествия во времени захватывали его воображение. Об этом можно было прочитать только в художественных книгах, и он никогда не думал, что такое может произойти на самом деле.
\par
«Я точно не знаю, как это произошло. Сначала я просто хотел погрузить себя в лёгкий транс, чтобы узнать, смогу ли я сдать экзамены. Я ни секунды не думал о том, что когда-нибудь мне придётся путешествовать во времени. Я сильно испугался, когда очутился там в первый раз. Я не знал, где, чёрт возьми, был. Я думал, что мне приснился кошмар, но не мог от него отделаться. Хотя моё тело было там, его там не было — если вы понимаете о чём я. Я не знал, куда попал, и не догадывался, переместился ли я назад или вперёд во времени. Моей главной задачей стало выживание и мне пришлось приложить немало усилий, чтобы сохранить рассудок и самообладание. Я должен был постоянно следить за тем, чтобы не попасть в засаду банды головорезов, или, что гораздо хуже, в лапы Системы. Я слышал рассказы о том, как она пытала несчастных слабых людей — никого из тех, кто был схвачен, больше никогда не видели.»
\par
Вздрогнув, Робин прервал свой рассказ. По его спине забегали мурашки а по по щеке покатилась одинокая слеза. Он достал из кармана белый платок, вытер слезу и шумно высморкался. Только сейчас он понял, как ему повезло, что он смог вернуться в свой настоящий дом. А ведь когда-то он всерьёз думал о том, чтобы остаться там навсегда.\\
\par
Элисон возвратилась и, обняв своего юного друга, спросила:
\par
«Что случилось, Робин? Как твоё самочувствие? Ничего не болит?»
\par
«Нет, со мной всё в порядке, правда. Я просто подумал, как мне повезло, что я снова здесь, среди людей, которые любят и заботятся обо мне.»
\par
«Что сказал профессор? - спросил Мэтью, пытаясь по-быстрому переменить тему, поскольку было понятно, что она расстроила Робина.»
\par
«Он будет здесь примерно через полчаса — когда я позвонила, у него была лекция. Один из преподавателей заболел и ему пришлось вести чужое занятие. Похоже, он не слишком доволен этим.»
\par
«Я собираюсь сходить в магазин и купить немного табака для своей трубки и сигареты», - сообщил Мэтью.
\par
«Скоро вернусь. Если кому-нибудь из вас что-то нужно, говорите сейчас.»
\par
Он взял куртку и надел её, роясь в карманах в поисках ключей от машины.
\par
«Что ты ищешь?» - спросил Гарри.
\par
«О, чёрт! Не могу вспомнить, куда я положил ключи.»
\par
«Зачем они тебе? Ты что, собрался ехать в магазин на углу улицы на машине? Он всего в пятистах метрах отсюда» - заметила Элисон.
\par
Промолчав, Мэтью вышел из комнаты и захлопнул за собой входную дверь.
\par
Интересно, как профессор планирует предотвратить ядерную войну? По-моему, он слишком самонадеян, и, если у него получится, это будет чудом. Не думаю, что кто-то воспримет его всерьёз. Нас обзовут кучкой сумасшедших, особенно если учесть, что скоро выборы. Люди скажут, что мы — делегаты от партии безумцев.
\par
«Да, Гарри. Мне тоже интересно, как он собирается донести до людей, что мы — не кучка стремящихся к саморекламе психов и не какие-нибудь паникёры» - ответила Элисон.
\par
Робин сидел молча, думая о людях, с которыми ему пришлось расстаться. Он чувствовал себя виноватым, потому что знал: рано или поздно Система начнёт догадываться, что происходит с её продовольственными кораблями. Тогда-то и потребуется его помощь, чтобы противостоять воинам Системы. Но, как заметил профессор, если им удастся предотвратить войну, то у Системы больше не будет этой зловещей власти над людьми. Он выбросил из головы тревожащее его чувство вины.\\
\par
«Если вы не будете против, мне бы очень хотелось сходить вместе с вами в кино и посмотреть какой-нибудь фильм. Что посоветуете? Я даже не представляю, что сейчас показывают на канале "Эй-Би-Си" или в кинотеатрах "Одеон".»
\par
«Согласна. Думаю, нам было бы полезно сделать небольшой перерыв и посмотреть "Комнату с видом", - ответила Элисон. Она уже видела этот фильм и была от него в восторге.
\par
«А после кино мы зайдём перекусить в "Бургер Кинг". Если вы, конечно, хотите этого» - добавила она поспешно. Давненько они, все вместе, не выходили на улицу и не развлекались.
\par
«Отличная идея, Элисон. Уверен, всем нам будет очень приятно. Мы подождём, пока не вернётся Мэтью и проверим, всё ли у него в порядке. Возможно, у него другие планы на этот вечер.»

\chapter{УЖАСЫ ВОЗЛЕ КАМИНА}
\noindent\par«Д{\scriptsizeОБРОЕ УТРО, РОБИН!», - ВОСКЛИКНУЛ ПРОФЕССОР} Фергер.
\par
«Как твоё самочувствие? Ты выглядишь намного лучше, чем вчера вечером. Хорошо поспал?»
\par
Он вошёл через парадную дверь, и Робин закрыл её за ним. Мальчик обратил внимание на необычную походку профессора и посчитал это зрелище немного комичным. Вежливо выражаясь, профессор Фергер был "хорошо сложен". Менее вежливо, но зато более точно, можно было бы сказать, что их немецкий друг любит вкусно и плотно поесть. Он был невероятным сладкоежкой, и все его студенты хорошо знали, что, если угостить профессора пирожными со свежим заварным кремом, то можно выпросить у него всё, что угодно. Его фигуру можно было описать лишь одним словом — "обширная". При этом, она вполне устраивала его. Его ухоженная внешность прекрасно сочеталась с простой и приятной улыбкой. Губы Робина тронула едва заметная усмешка, когда он подумал о том, что профессору, должно быть, не удаётся самостоятельно завязать шнурки на своих ботинках. Более того, он гадал, как давно профессор не видел своих ног. Возможно ли, что профессор Фергер не снимал носки и ботинки перед тем, как отправиться спать?\\
\par
Впустив профессора Фергера, Робин не последовал за ним в гостиную, где сидели остальные студенты. Вместо этого он отправился на кухню и налил большой стакан сока из чёрной смородины, а также выложил на тарелку целую пачку шоколадного печенья.
\par
Он передал эти яства профессору и на лице последнего засияла улыбка Чеширского кота.
\par
«Спасибо тебе, мальчик. Из-за того, что этим утром мне пришлось вести незапланированные занятия, у меня не было времени даже на еду» - произнёс он с наполовину заполненным ртом, счастливо жуя.
\par
«Как только я закончу есть, мы сразу продолжим начатую накануне работу. Итак, чем вы занимались сегодня утром, пока я был в университете?»
\par
«Почти ничем, профессор Фергер», - ответила ему Элисон.
\par
«Мы ждали, когда проснётся Робин. Мы решили, что будет лучше, если ему позволят поспать подольше, учитывая его измотанное состояние прошедшей ночью. Не волнуйтесь, мы позаботились о том, чтобы он хорошо позавтракал.»
\par
Она ответила почти моментально, опасаясь, что кто-нибудь из присутствующих наговорит лишнего.
\par
«Хорошо. Значит у нас будет много работы. Надеюсь, ни у кого из вас не запланировано никаких мероприятий на сегодняшний вечер.»
\par
«Ну… Ээээ», - начал было Робин.
\par
«Отлично! Я рад, что не запланировано», - перебил его профессор,
\par
«Сейчас я достану папку с моими записями и мы сможем продолжить.»
\par
Он доел печенье и допил остатки сока, прежде чем начать копаться в своём старом, невзрачном портфеле из коричневой кожи.
\par
«Я нашёл их! В какой-то ужасный момент мне показалось, что я оставил их дома.»
\par
«Какая жалость, что он этого не сделал», - сокрушённо подумал Робин.
\par
«Итак, на чём мы остановились перед тем, как Робин уснул?» - пробормотал профессор, роясь в свои бумагах; его очки неспешно сползали с носа.
\par
«Мы закончили на обсуждении Системы и людей, населявших тот мир. Робину оставалось лишь нарисовать карту Саутгемптона» - ответил Гарри.
\par
«Нет, Гарри! Я думаю, что ты ошибаешься. Судя по моим записям, мы закончили на окружающей среде. Полагаю, что у меня имеется достаточно информации по этой теме.»
\par
«М-м... Профессор?», - начал Робин,
\par
«Какие ещё темы из вашего списка вы хотели бы обсудить? Если вы зачитаете их, то я уверен, что будет намного быстрее, если я просто опишу их, вместо того, чтобы вы задавали вопросы, а я отвечал на них.»
\par
Ему действительно хотелось пойти в кино и в таком случае у них были хорошие шансы успеть на последний сеанс, который должен был начаться около восьми часов.
\par
«Это кажется вполне разумным, Робин. Ну, ещё есть три основные темы — люди, которые там жили, то есть народная масса, их элита и, наконец, Система.»
\par
«Хорошо. Люди — они представляли собой довольно жалкое зрелище. Я мало что могу сказать о них такого, чего вы не видели сами. Я скажу вам, что меня удивило: Великобритания всегда славилась как страна, любящая животных. Видеть, как убивали собак и кошек на самом деле… Интересно, прибегали ли те немногочисленные группы, протестовавшие против жестокого обращения с животными, к этим примитивным, но необходимым средствам? Или они голодали? Серьёзно, когда я попал туда в первый раз, мне стало дурно от того, в каких унизительных условиях приходилось существовать людям. Большую часть времени они бродили полуголодные. Было невыносимо видеть, как они страдают от… ох, можно только догадываться, какими болезнями они могли заразиться.»
\par
Робин выждал. Наступила полная тишина — все с полным вниманием ждали, когда он продолжит рассказ.
\par
«Вонь, исходившая от этих людей, была такой тошнотворной… Но чего ещё можно ожидать, если им было негде помыться и сходить в туалет. Думаю, там было огромное количество болезней, вызванных отсутствием гигиены. Они были бездомными и жили одним днём. Мне было досадно, когда я думал о том, что многие из них вложили огромные усилия, чтобы построить себе уютные дома. А потом случилась война, и не успели они оглянуться, как исчезло всё — всё, ради чего они трудились. Интересно, что случилось бы с обществом, если бы оно увидело то же, что совсем недавно видел я.»
\par
«Матери не могли сделать ничего, чтобы предотвратить смерть детей. Вы не можете даже представить, как это было ужасно, когда их дети умирали там, у них на руках.
\par
Мальчик снова сделал паузу, раздумывая о тяжёлом положении этих бедных, несчастных людей. Он глубоко вздохнул, изо всех сил стараясь не позволять слезам подступить к глазам.
\par
«Да, если подумать, то я, наверное, был одним из немногих детей, живших там. Помните, что я говорил о странных людях, бродящих по окрестностям? Так вот, я имел в виду, что там были люди, которые потеряли большую часть своих волос и имели на коже ужасные пятна. Их лица были бледными и не выражали признаков жизни. Прямо как зомби, или, буквально, живые мертвецы. Это было похоже на кошмарный сон, на фильм ужасов, в котором я играл главную роль.»
\par
«О, мой несчастный ребёнок», - сочувственно сказала Элисон,
\par
«Должно быть, для тебя это было невыносимо. Обещаю, что никому не позволю снова подвергнуть тебя чему-нибудь, столь же ужасному.»
\par
«Этого хватит. Думаю, что теперь у меня имеется достаточно подробный отчёт о населявших тот мир людях. Не возражаешь, если мы перейдём к «элите»?» - спросил Робина профессор Фергер. Он видел, как негативно отражается его "допрос" на общем самочувствии мальчика. В Саутгемптоне будущего он находился под огромным давлением, и теперь его просили вспомнить как можно больше, пока воспоминания всё ещё были относительно свежими. У многих людей случился бы нервный срыв после таких же суровых испытаний, какие выпали на его долю.
\par
«С мной всё в порядке» - поспешно сказал Робин, увидев на лице профессора беспокойство,
\par
«Я бы предпочёл, чтобы мы закончили всё это сейчас, и тогда я смогу расслабиться.»
\par
«Хорошо, это меня устраивает», - ответил профессор Фергер.
\par
«Элита… Они были достаточно храбрыми. Необходимо отдать должное их смелости и дальновидности.»
\par
«Дальновидности? Что ты имеешь в виду?» - спросил профессор, поднимая взгляд поверх очков,
\par
«Я что-то пропустил или неправильно понял?»
\par
«Да, именно дальновидности. Возможно, я не рассказывал об этом раньше, но однажды в разговоре Джимми поведал мне о том, что после первого объявления о блокировании русскими Ормузского пролива, группа людей стала готовиться к наступлению ядерной войны, значительно преуспев в этом деле.»
\par
«Ты хочешь сказать, что в Саутгемптоне нашлась горстка людей, не доверявших правительству и средствам массовой информации, и призывавшая граждан начинать готовиться к войне?»
\par
Профессор откинулся на спинку стула.
\par
«Прекрасно! Оказывается, общественность не настолько глупа, как это представлялось мне до сих пор.»
\par
«Я хотел бы продолжить», - произнёс Робин,
\par
«На чём я остановился? Вспомнил, речь шла об "элите". Могу рассказать о них немногое. Они были достаточно хорошо организованы. В основном они промышляли воровством еды и предметов первой необходимости у Системы, забирая всё нужное им во время выгрузки кораблей. Они были рационально мыслящими людьми, верили в свои силы и не разделяли точку зрения большинства, полагавшего, что после атомных бомбардировок не стоит даже пытаться выживать. Они готовились и они приняли все необходимые меры предосторожности.»\\
\par
Через некоторое время профессор объявил, что вполне удовлетворён, и предложил перейти к заключительной части.
\par
«А, теперь Система. Как вы знаете, Система была там органом управления. Она главенствовала над людьми около трёх лет, и за всё это время в их жизни не произошло никаких изменений в лучшую сторону. Нет ничего удивительного в том, что люди не испытывали к ней ни малейшей приязни. Таков главный закон любого цивилизованного сообщества — никто не вправе рассчитывать на уважение и сотрудничество, если не желает проявлять уважения к другим. Всё, что они совершали — это массовый террор и физические расправы, гарантируя себе, что население будет оставаться слабым и неспособным на восстания или революцию..»\\
\par
«Робин повернулся к Элисон и внимательно всмотрелся в её глаза. Её что-то беспокоило.. Но что именно? Он никак не мог понять этого.»
\par
«Профессор, вы не станете возражать против небольшого перерыва? Сейчас половина пятого, и мы сидим здесь уже около трёх с половиной часов.»
\par
Кивнув, профессор согласился на просьбу Робина и печально посмотрел на свой опустевший стакан и тарелку.
\par
«Элисон, пожалуйста, помоги мне заварить для всех чаю и приготовить какую-нибудь снедь. Они, должно быть, умирают с голоду, потому что ничего не поели во время обеда.»
\par
Элисон вместе с Робином покинули гостиную, оставив Гарри и Мэтью развлекать профессора Фергера.
\par
Когда они пришли на кухню, Робин включил электрический чайник. Он развернулся лицом к Элисон и мягко, но настойчиво усадил её на стул.
\par
«Эли, что случилось? Тебя расстроил утренний эпизод с Гарри? Не молчи, ты ведь знаешь, что со мной можно поговорить.»
\par
«Нет, это не имеет к нему никакого отношения. Напротив, я вполне довольна тем, что между нами, наконец-то, расставлены все точки.»
\par
«В таком случае, что это?»
\par
«Меня расстроил твой рассказ о Системе. Он напомнил мне о кровавой расправе, и, в особенности, о варварском способе, которым один из стражей решил убить ту женщину и её ещё не родившееся дитя. Я понимаю, что веду себя глупо, расстраиваясь из-за того, что ещё даже не произошло, но ничего не могу поделать с собой.»
\par
«Элисон, не стоит стыдиться своей излишней эмоциональности. Нужно быть предельно чёрствым и бесчувственным человеком, чтобы на тебя никак не повлияло подобное зрелище. Если бы в тот момент в моём желудке находилось что-то иное, кроме пустоты, то меня бы непременно стошнило. Одному Богу известно, как сильно мне хочется повернуть время вспять и больше не позволять тебе становиться свидетелем подобных зрелищ. Прости меня, Элисон, и пожалуйста, больше не волнуйся. Надеюсь, нам как-нибудь удастся предотвратить это всё.»
\par
Робин подошёл к Элисон и крепко обнял, стараясь подарить ей ощущение спокойствия и безмятежности.
\par
Со мной всё в порядке. Смотри, вода уже закипела — пора заваривать чай. Чем скорее мы закончим, тем лучше: если мы поторопимся, то успеем на последний сеанс в кинотеатре.
\par
Робин посмотрел ей в глаза, внимательно заглядывая по очереди в каждый из них.
\par
«Ты уверена, что у тебя всё в порядке? Я здесь, чтобы выслушать тебя, если ты нуждаешься в моей помощи. Я твой друг, и для меня очень важно, чтобы ты была счастлива. Поверь, всё будет хорошо.»
\par
Элисон кивнула. Слёзы, блестевшие на глазах Робина, глубоко ранили её душу и вызвали в ней мощнейший прилив чувствительности.
\par
«Робин.. После своего возвращения оттуда ты стал другим… Будто бы повзрослел лет на десять и стал более восприимчивым к тому, что чувствуют люди. Более того, как мне кажется, у тебя есть ответы на их вопросы.»
\par
«Находясь в том месте, мне пришлось повзрослеть. Я научился помогать самому себе и больше не зависеть от остальных, как это было здесь. Там я не мог думать лишь о себе и ждать милостей от других. Когда я сверг власть Системы в этом районе, на меня внезапно свалилась ответственность — быть отцом и матерью для всех этих беспомощных мужчин и женщин. Они ожидали, что я решу все их проблемы и буду следить за тем, чтобы им было удобно жить, чтобы у них всегда были кров и пища. Они были словно малые дети, физическая и эмоциональная жизнь которых полностью зависит от родителей. Понимаете, я совсем не хотел меняться или взрослеть раньше, чем нужно, но у меня не осталось другого выбора.»
\par
«Я могу понять это, Робин, но мне очень жаль, что в твоём юном возрасте тебе пришлось лицезреть такие ужасные зверства. Ни один заботливый родитель не позволил бы своему ребёнку пережить такое даже в кино, не говоря о реальной жизни. Ладно, нужно идти, нас уже заждались.\\
\par
Пока Элисон готовила чай, Робин достал тарелки. В шкафу, в большой пластиковой коробке с этикеткой "Таппервэр", он обнаружил несколько свежих колбасных рулетов, упаковку печенья с мюсли и большой нетронутый шварцвальдский торт со взбитыми сливками.
\par
«Этого должно быть достаточно, чтобы продержаться до нашего вечернего похода в "Бургер Кинг" - заметил он.
\par
Они отнесли подносы с едой всем остальным. Лицо их милого старого друга озарила широкая улыбка, когда он разглядел на подносе десерт.
\par
«Я продолжу, пока мы едим» - сказал Робин, попивая чай и пережёвывая печенье с мюсли. Прошло немного времени, прежде чем он что-то сказал.
\par
«Мне кажется, что мы обсудили все темы, которые вы перечислили в самом начале. Что касается будущего мира, то я не думаю, что нам нужно говорить о чём-то ещё. События должны быть понятны — в конце концов, вы сами были свидетелями всего, что происходило. Вы же сделали копию той газетной статьи, которую я для вас прочитал?»
\par
«Всё правильно — ты рассказал всё, о чём я просил. Отвечаю на твой вопрос: да, я дословно переписал весь её текст.»
\par
Профессор ненадолго замолчал, роясь среди бумаг в своём портфеле. Он достал лист, на котором было нацарапано несколько его пометок. Никто из его студентов так и не смог разобрать ни одной буквы, а заголовок был определён ими исключительно по положению вверху страницы. Поэтому они поверили профессору на слово и согласились, что на бумаге действительно были подробности той газетной статьи.
\par
«Хорошо. В котором часу нам зайти к вам завтра?» - спросил профессора Мэтью,
\par
«Если я правильно понял, на сегодня все вопросы закончились? По крайней мере те, что требуют немедленных ответов, и не могут подождать до завтра.»
\par
Он продолжил речь, приводя доводы в пользу остановки. Никто из студентов не проронил ни слова. Они сидели и молча смотрели на профессора в восемь глаз, ожидая, что он заговорит первым и, таким образом, станет проигравшей стороной в этом крошечном психологическом противостоянии.
\par
Так прошло несколько минут.
\par
Наконец, нервы профессора сдали и всем стало ясно, какая из двух сторон оказалась слабее. Правда была в том, что он не смог придумать подходящего предлога, чтобы и дальше задерживать их.
\par
«Ладно, я согласен — нам больше нечего обсуждать. Ожидаю увидеть вас всех завтра, ровно в девять часов, в моём университетском кабинете. Постарайтесь не излишествовать этим вечером, а завтра — проснуться вовремя.»
\par
Последняя его реплика была адресована Гарри.
\par
Он закончил складывать свои записи в портфель и надел колпачок на свою дорогую перьевую ручку. Все важные записи он делал именно ей, из-за чего на бумаге и на его рубашке постоянно оставались кляксы. На боковой стороне безымянного пальца его левой руки постоянно виднелось чернильное пятно — явный признак человека, которой очень сильно любит писать. Студенты не раз пытались убедить профессора пользоваться печатной машинкой, упирая на то, что она сэкономит время. Они не говорили ему всей правды: ни у кого из них не хватало смелости сказать, что его почерк выглядит хуже, чем у ребёнка младшего школьного возраста, а расшифровка сделанных им "упорядоченных" записей — это сущий кошмар.
\par
«Постарайтесь не забыть — я рассчитываю увидеть вас завтра, ровно в девять часов утра. Желаю вам хорошо провести время. И пожалуйста, не утомляйте Робина. Спокойной ночи!»
\par
Профессор вышел из дома. Хотя он старательно изображал беспечность, на его душе было неспокойно.

\chapter{ПРЕСЛЕДОВАНИЕ}
\noindent\parП{\scriptsizeРОФЕССОР ВЕРНУЛСЯ К СЕБЕ ДОМОЙ ПОСЛЕ ТОГО, КАК ЗАЕХАЛ В ПАБ}, и "опрокинул" там несколько рюмочек виски. Депрессия, которую он чувствовал прежде, всё сильнее укоренялась в его сознании.
\par
Тем утром он задержался в университете вовсе не для того, чтобы заместить другого преподавателя — это было его ложным алиби.
\par
Утром он был занят совсем другими делами.
\par
В течение нескольких часов он пытался дозвониться по телефону до Британского премьер-министра, однако ему не повезло и все усилия оказались напрасны.
\par
Единственным человеком, с которым ему было позволено говорить, оказался министерский секретарь, но даже он не смог оказать требуемой помощи, сообщив только, что «в настоящее время премьер-министр находится на важном совещании.»
\par
Профессор изо всех сил старался втолковать, что война, о которой он только что сообщил, должна начаться менее, чем через год, и что в этом вопросе он чрезвычайно серьёзен, однако к нему отнеслись как к умалишённому, пытающемуся привлечь к себе внимание представителей высшего правящего сословия. Чем больше он размышлял о своём утреннем унижении, тем сильнее охватывали его мрачные предчувствия.
\par
«Если нет никакой возможности достучаться до собственного правительства, значит не осталось никакой надежды на то, чтобы предотвратить приближающуюся катастрофу» - с горечью думал профессор,
\par
«На всём белом свете не найдётся ни одного человека, который воспримет меня всерьёз. Что же мне делать? Я не способен предотвратить взрывы атомных бомб!»
\par
Сейчас он находился поблизости от своего дома, на расстоянии всего одной улицы.\\
\par
Часы показывали половину одиннадцатого. Профессор Фергер молча сидел в своём кресле, в гостиной, напротив электрического камина. Он бессмысленно смотрел в неизвестность, в полную темноту. Его глаза опухли, налились кровью и сильно болели.
\par
Он был в растерянности — ему казалось, что он находится в тесной маленькой комнатке и чувствовал, как быстро сужаются её стены. Он не должен взваливать на себя такой груз — ему не следует в одиночку нести ответственность за спасение мира, ведь это не его работа. Этим должны заниматься политики, по этой причине за них отдают голоса. Именно они ответственны за благополучие своих подданных.
\par
Чем больше профессор сидел в темноте и размышлял, тем сильнее чувствовал, как сжимаются на его шее оковы, а стены комнаты становятся всё ближе и ближе. Воображение подсказывало ему, что теперь он станет жертвой преследования властей, хотя он всего лишь пытался делать то, что считал правильным.\\
\par
Несмотря на то, что накануне шёл нескончаемый дождь, этим днём выдалось чудесное апрельское утро.
\par
Мэтью выглянул из окна своего дома. На часах было шесть утра. Отовсюду слышалось весёлое воробьиное чириканье, а на чистом, безоблачном небе ярко сияло взошедшее солнце, отражавшееся ослепительными искрами в капельках росы на листьях и головках прекрасных жёлто-фиолетовых цветов, росших неподалёку от его дома. Потребуется ещё немного времени, чтобы скрыть уродство той кирпичной стены, что расположена в конце их сада — оно спрячется за густыми зарослями "русского" плюща. Единственное, что вечно портило сад — это газон, требующий постоянного и тщательного ухода. Однако за такую скучную и утомительную работу никому не хотелось браться, и он часто оставался в запущенном состоянии.
\par
«Сегодня чудесный денёк, и у меня не запланировано никаких дел, поэтому стоит заняться стрижкой травы» - подумал Мэтью.
\par
Умывшись и переодевшись, он вышел в сад, прихватив с собой стакан свежевыжатого апельсинового сока.
\par
Хорошо, но где искать газонокосилку?
\par
Погрузившись в размышления, он нахмурился. Дело было в том, что он давно не пачкал своих ухоженных рук и забыл о таких пустяках, как например, место, в котором хранилась газонокосилка. После продолжительных поисков он нашёл её в маленьком чуланчике возле кухни.\\
\par
Была половина седьмого, когда Мэтью закончил своё сегодняшнее доброе дело. Работа была тяжёлой, особенно для человека, не привыкшего к таким огромным тратам энергии по утрам, а в его случае — в любое время дня. Тыльной стороной ладони он вытер пот со лба и облегчённо вздохнул, осматривая свою непростую работу. Обтерев липкие руки о штаны, он порадовался, и вернул газонокосилку туда, где нашёл, а после — поставил разогреваться чайник. Ему пришлось выпить чашку крепкого чёрного кофе, чтобы прийти в себя после такого необычного для него поведения.
\par
Мэтью тихонько поднялся по лестнице, тщательно стараясь не уронить поднос с завтраком, приготовленным специально для Робина. Поскольку обе его руки были заняты, он решил слегка подтолкнуть ногой дверь, преграждавшую его путь.
\par
«Робин, я принёс твой завтрак! Пора вставать. Уже почти восемь часов, а к девяти нам нужно быть в университете.»
\par
Он поставил поднос на столик и широко распахнул дверь в комнату, после чего подошёл к окну и раздёрнул шторы, позволив солнцу залить всю комнату своим ярким светом.
\par
Ему пришлось слегка встряхнуть Робина.
\par
«Подъём! Завтрак уже остывает. На часах — две минуты восьмого» - произнёс Мэтью, с любовью рассматривая свои старинные золотые часы. Одеяло зашевелилось, и из-под него медленно выглянуло заспанное лицо. Осторожно открыв один глаз и узрев им перед собой бодрствующего и полного сил Мэтью, оно издало преисполненный скорбью стон и тут же снова скрылось под одеялом.
\par
«Робин, если ты сейчас же не встанешь, я сорву с тебя это чёртово одеяло, и тогда ты замёрзнешь и, может быть, заболеешь» - сообщил Мэтью строгим тоном. Шансы на то, что угроза будет исполнена, были достаточно высоки, поэтому Робин без лишних слов выскочил из своего укрытия и уселся на краешек кровати. Он всегда предпочитал тепло, нежели холод, и малейшие, даже самые незначительные, изменения погоды могли легко заставить его дрожать.
\par
«Завтрак — первое услышанное мной этим утром слово! Что у нас сегодня вкусненького? Что бы это ни было, пахнет оно замечательно! М-м-м-м, я просто умираю от голода!»
\par
Робин глубоко вдохнул, наслаждаясь ароматом еды, столь приятной для его живота.
\par
«Я принёс чёрный кофе, обжаренный тост, кусок чайного пирога и две пышки, истекающие топлёным маслом. Как это звучит? Неплохо?»
\par
«Спасибо, Мэтью! Это музыка для моих ушей. Надеюсь, что в намазанном на тост мармеладе не окажется противных мелких комочков.»
\par
«Я тщательно проверил — их там нет. А теперь ешь и быстренько одевайся, иначе мы опоздаем на встречу с профессором.»
\par
Мэтью передал поднос с завтраком, и приступил к поискам одежды для Робина.
\par
«Вот твои вещи. Я встречу тебя внизу, когда ты будешь готов. Приятного аппетита!»
\par
Робин уплетал завтрак за обе щеки. Как обычно бывало в таких случаях, весь его рот и часть подбородка были испачканы мармеладом, а в комнате отчётливо слышалось смачное чавканье.
\par
Он обратил своё внимание на отрытое окно. Это казалось странным, ведь прежде Мэтью никогда не оставлял его открытым, зная как болезненно переносит сквозняки его юный друг. Мальчик поднялся с кровати и выглянул на улицу. Именно тогда он и заметил, как преобразилась лужайка.
\par
«Боже! Кто-то покосил её. Но не мог же Мэтью… Если это всё-таки он, то с ним, должно быть, что-то стряслось… И этот завтрак в постель…»
\par
Робин вышел из своей спальни и с верхних ступеней окликнул Мэтью:
\par
«Мэтью! Что случилось с лужайкой? Неужели ты стал лунатиком и этой ночью тебе приснился кошмар, в котором ты занимался какими-то мрачными потусторонними делами… Например, косил траву возле нашего дома?»
\par
Мэтью в спешке стал подниматься наверх, уронив на лестницу свою зажигалку:
\par
«Что ты имеешь в виду, говоря о том, что я, наверное, ходил прошедшей ночью во сне? Я всегда готов помогать!»
\par
«Мэтью, не пытайся выкручиваться. Лучше признайся, что твоё любимое занятие — ничего не делать. Лень — это твоя визитная карточка. В последний раз, когда нужно было подстричь газон, кто сказал, что внезапно ему стало плохо и отправился искать врача? Ну давай, скажи правду — ты ненавидишь работать.»
\par
«Хватит нудеть! Я не ходил во сне. Я рано проснулся и решил покосить газон. Одевайся живее, нам пора уходить.»
\par
Мэтью оставил Робина в одиночестве, чтобы вернуться на кухню, и в тишине и спокойствии доесть свой тост с сыром.
\par
«Вот это наглость. Чудик обозвал меня лентяем» - подумал Мэтью, но вслух этого, конечно, не произнёс.\\
\par
К тому времени, когда Мэтью вместе с Робином прибыли в университет, Элисон и Гарри уже находились в кабинете у профессора Фергера.
\par
«Вы опоздали на две минуты. До этого момента обычно опаздывали мы. Не переживайте, профессора ещё нет. Наверное, сегодня национальный праздник опозданий для наиболее пунктуальных и ответственных членов общества» - ехидно заметил Гарри, ненадолго оторвавшись от своей книги.
\par
Мэтью и Робин нашли для себя пару стульев и придвинули их поближе к своим друзьям, после чего между ними завязалась непринуждённая беседа о прошедшей накануне вечеринке и о том, насколько интересным оказался фильм. Посмеявшись над тем, как ловко удалось провести профессора и выиграть для себя свободный вечер, они наконец-то обратили внимание на время. Часы показывали половину десятого, но от профессора до сих пор не было никаких известий.
\par
«Возможно, в этом нет ничего страшного. Он мог задержаться в пути из Уинчестера — сейчас там ведутся дорожные работы. Я уверена, что скоро он будет здесь» - произнесла Элисон, потягивая горячий шоколад из картонного стаканчика. Это был единственный приличный напиток, который можно было купить у торгового автомата, расположенного в конце университетского коридора.
\par
«Попробуйте угадать, что этим утром сделал Мэтью.»
\par
«Удиви нас, Робин» - ответил Гарри.
\par
«Минуточку внимания…  Звучит барабанная дробь… Он скосил траву на газоне!»
\par
«Ты имеешь в виду… Не может быть! Он и правда сделал это? Ужас! Шок! Мэтью, что с тобой не так? Разве это не противоречит твоим убеждениям? Дружище, ты слишком жесток к себе!»
\par
Гарри затих, а затем громко расхохотался, а все остальные последовали его примеру. Мэтью остался сидеть в прежней позе, скрестив на груди руки и исподлобья поглядывая на товарищей.
\par
«Мне плевать, если вы решили изгнать из меня всех святых. Просто сделайте вид, что меня здесь нет. Я тихонько посижу здесь, а когда вы решите, что окончательно пришли в себя, мы наконец-то сможем выяснить, куда запропастился профессор. Сейчас четверть одиннадцатого и он давно уже должен был с нами связаться.»
\par
После этих слов все приступы безудержного смеха и корч были моментально окончены. Вмиг посерьёзневший Гарри поднялся со своего стула и направился в офис главного управляющего отделением.\\
\par
«Сегодня профессор Лайон не получал никаких известий от профессора Фергера и ничего не знает о его планах на это утро» - вернувшись, сообщил Гарри остальным студентам.
\par
«Я думаю, нам нужно отправиться к нему домой и убедиться, что с ним всё в порядке. Робин, ты, кажется, уже бывал в его доме прежде, и должен помнить, как туда добраться» - сказала Элисон, повернувшись в сторону мальчика.
\par
«Да, я бывал. Его адрес должен быть записан на клочке бумаги — он в кармане моего жакета.»
\par
Он порылся в карманах, вывалив наружу кучку разнообразного хлама, и наконец, извлёк скомканный листок.
\par
«Вот он. Думаю, нам лучше не медлить.»\\
\par	
Примерно через полчаса они подъехали к особняку профессора. Его машина стояла припаркованной возле дома, а окна жилища были занавешены. Бутылки с молоком, аккуратно расставленные молочником этим утром, находились на ступенях в нетронутом состоянии.
\par
«Это странно. Всё выглядит так, словно он ещё не проснулся» - отметил Мэтью, пытаясь осмотреть содержимое почтового ящика через имеющиеся в нём щели.
\par
«Здесь слишком темно, но я вижу несколько писем. Вы слышите, как скулит его собака?»
\par
«Попробуй открыть входную дверь, Мэтью.»
\par
В голосе Гарри отчётливо слышалось беспокойство. Несмотря на то, что он частенько подшучивал над профессором, на самом деле старик значил для него очень много.
\par
«Не могу — она заперта. Я нажал на кнопку звонка, но, кажется, он не торопится открывать нам дверь. Нужно поискать другой способ попасть внутрь.»
\par
Они обошли дом вокруг и обнаружили, что ведущая на кухню дверь, расположенная с обратной стороны здания, оказалась незапертой. Внезапно вынырнувшая из глубины профессорского дома собака принялась яростно облаивать их, протестуя против вторжения незваных гостей.
\par
Разделившись, они отправились на поиски профессора Фергера.

\chapter{ГОРЕ}
\noindent\par«Г{\scriptsizeАРРИ, МЭТЬЮ! ЗДЕСЬ НИКОГО НЕТ!  КТО-НИБУДЬ ИЗ ВАС НАШЁЛ ХОТЬ ЧТО-ТО?}» - прокричала Элисон в темноту.
\par
«Постель аккуратно заправлена. Его не было здесь прошедшей ночью» - откликнулся Мэтью.
\par
«Если его не было, то что, в таком случае, делает здесь его машина? Наверное, ты случайно не заметил, но она находится на подъездной дорожке, прямо напротив твоей» - саркастически заметил Гарри. Он не хотел грубить Мэтью, но был сильно обеспокоен отсутствием профессора, и его нервы сдавали.
\par
«Прости меня, Мэтью, я не хотел...»
\par
Он замолчал и попытался успокоиться.
\par
«Всё в порядке, Гарри. Я понимаю… Ничего плохого не случилось. Давайте поищем Робина. Думаю, можно с уверенностью утверждать, что профессора Фергера здесь нет. Возможно, он уехал в Лондон. Он говорил, что когда-нибудь, в не слишком отдалённом будущем, попытается связаться с премьер-министром.»
\par
После этих слов Гарри и Элисон направились вниз по лестнице вслед за Мэтью.
\par
«Робин! Робин! Где ты находишься?» - прокричала Элисон.
\par
Ответа не последовало.
\par
«Куда снова пропал этот мальчик? Неужели он не понимает, что сейчас не время играть с нами в прятки» - ворчала она себе под нос.
\par
«Ты что-то услышала, Элисон?»
\par
«Нет, Гарри, ничего. Наверное, он в столовой или гостиной.»
\par
Гарри и Мэтью вдвоём отправились в столовую, а Элисон решила осмотреть гостиную.
\par
Подойдя к прикрытой двери она услышала тихие всхлипы. Тихонько отворив её, вошла внутрь — Робин был здесь. Он склонился над профессором, сидевшим в кресле в какой-то нелепой и неестественной позе, и плакал. Комната была окутана полумраком и только изредка освещалась сполохами работающего на полную мощность электрического камина.
\par
Элисон растерялась. Она не знала, как ей следует поступить — подойти к мальчику, взять его за руки и попытаться утешить или оставить в покое, выйдя из комнаты и подождав, когда он придёт в себя.
\par
Она покинула комнату точно так же, как и вошла в неё —  оставшись незамеченной рыдающим ребёнком. Плотно прикрыв дверь, быстрым шагом прошла в столовую и позвала Гарри и Мэтью.
\par
«Что случилось, Элисон?» - громко спросил Гарри.
\par
«Т-с-с. Говори потише.»
\par
«Что случилось? Почему ты требуешь тишины?» - повторил свой вопрос Гарри, на этот раз шёпотом.
\par
«О Господи!» - тихо сказала Элисон, и её дыхание участилось.
\par
«Я не представляю, что делать с Робином.»
\par
«Почему? Что с ним случилось? Он снова ввёл себя в транс?» - спросил Мэтью.
\par
«Он плачет! Я впервые вижу его таким. Он в гостиной, возле профессора, и его трясёт от горя. Понимаете, мне кажется, что профессор мёртв. Робин не знает, что я его видела.»
\par
«Боже! Что нам делать? Наверное, нужно вызвать полицию. Элисон, ты должна пойти к Робину и попытаться утешить его» - заикаясь, промямлил Гарри. Он был в панике. Никто из них не подготовился к такому маловероятному исходу событий.
\par
Мэтью вышел в прихожую и набрал номер полицейского участка, не имея никакого представления о том, что будет говорить. О чём он мог рассказать? О том, как несколько студентов вломились в чужой дом и обнаружили мёртвого старика, сидящего в своём кресле возле камина? А о чём ещё? -Он не знал.\\
\par
Не произнося ни звука, Элисон вошла в гостиную, подошла к Робину и положила руки на его плечи.
\par
Он обернулся, вытер рукавом жакета своё заплаканное лицо и упал в её объятия, снова зарыдав.
\par
«Робин… Милый… Робин, не плачь. Пожалуйста, не нужно… Иначе я заплачу вместе с тобой.»
\par
Элисон трудно давались слова, ей приходилось делать длинные паузы, чтобы откашляться, восстановить дыхание и не позволить себе раскиснуть. Она вытерла слёзы с глаз.
\par
«Робин, скоро здесь будет полиция. Ты знаешь, почему он…? Что стало причиной его…? Смерти… Робин, поверь, я знаю, что ты сейчас чувствуешь. Обними меня крепче, прижмись. Помни, я всегда буду рядом с тобой.»
\par
Она ласкала его, с нежностью поглаживая по спине и затылку, и отчаянно старалась, чтобы он почувствовал себя в безопасности.
\par
«Похоже, что он убил себя, Элисон. В его ладонях были зажаты эти письма и...»
\par
Его голос снова затих, а по раскрасневшимся щекам в два ручья потекли слёзы.
\par
«Могу я взглянуть на них? Где они?»
\par
Робин протянул ей два влажных конверта — они были насквозь пропитаны слезами мальчика.
\par
«Первый из них мы можем отдать полицейским. Второй — лично для нас.»
\par
Элисон открыла первый конверт, и там было написано:
\begin{quote}
The Mind, The Human Race.\\
It's time it died!\\
Why should it suffer when it can no longer be tried?\\
Every day being tested - can it be done?\\
Mind over matter, can that war be won?\\
From a race of scurrying, useless life,\\
Full of pain, pressure and strife,\\
To be free of the chains that burn like ice,\\
To leave this world would suffice." \footnote[1]{Дословно: "Разум. Человечество. Настало время умирать. Для чего им страдать, если их больше нельзя испытывать? Мы проверяем каждый день - возможно ли это? Разум над материей - возможна ли победа в такой войне? Чтобы спастись от гонки суетной и бесполезной жизни, исполненной боли, давления и борьбы, освободиться от обжигающих, словно лёд, оков, достаточно покинуть этот мир."}
\end{quote}
\noindent\parОна аккуратно сложила листок с предсмертным стихотворением и убрала его обратно в конверт. Достав из кармана носовой платок, вытерла навернувшиеся на глаза слёзы и высморкалась.
\par
«Кажется, он был в чрезвычайно подавленном состоянии, когда писал эти строки. Но почему? В них нет никакого смысла.»
\par
«Элисон, ты найдешь объяснения во втором, предназначенном для нас, письме. Только не читай его вслух — мне бы этого не хотелось.»
\par
Робин снял свой жакет и накинул его на голову и плечи профессора, после чего вышел из комнаты, чтобы поискать Мэтью и Гарри.
\par
«Элисон сказала, что вы звонили в полицию. Когда они будут здесь, отдайте им этот конверт. В нём что-то вроде прощального стихотворения. Нужно забрать отсюда его собаку, иначе её усыпят. Пожалуйста, Мэтью, скажи, что мы можем взять её себе.»
\par
«Да, Робин» - ответил Мэтью,
\par
«Конечно, она будет жить у нас. Полиция должна прибыть с минуты на минуту.»
\par
«Я хочу домой. У меня нет желания разговаривать с полицейскими. Ты можешь отвезти меня домой, Мэтью? Разве Гарри не может поговорить с ними вместо нас?»
\par
«Робин, тебе придётся остаться и ответить на все их вопросы, ведь это ты обнаружил тело профессора Фергера» - прервал его Гарри.
\par
«Хорошо. Но если мне придётся остаться, то вы должны сказать, что не имеете понятия о причинах его самоубийства. Я пойду к Элисон, чтобы предупредить её».
\par
Робин вернулся в гостиную, подошёл к Элисон и крепко обнял.
\par
«Спасибо за слова утешения, я в них очень нуждался. Когда сюда приедет полиция, все мы должны настаивать, что никто из нас не знает о том, почему он решил покончить с собой... Элисон, то письмо… Ты не могла бы вернуть его мне?».
\par
В тот момент, когда Элисон передавала ему письмо, прозвенел дверной звонок. Дверь открыл Гарри. Встретив полицейских, он позволил им войти в дом и затем проводил в гостиную.
\par
Сперва осмотрели тело покойного. В одном из карманов его одежды был обнаружен шприц и пустой инсулиновый флакон — внезапно выяснилось, что профессор был диабетиком, чего до этой минуты не знал ни один из его студентов.
\par
Чтобы опросить каждого из присутствующих, полиции потребовалось несколько часов, после чего всех отвезли в участок для дачи официальных показаний. Тело профессора Фергера перевезли в морг.
\par
В конце всех злоключений Робину было позволено забрать себе собаку профессора.\\
\par
Они вернулись в дом к Мэтью.
\par
Покинув остальных, Робин поднялся наверх, в свою комнату, и снова тихонько заплакал. Дрожащими руками он взял в руки адресованное им письмо и, прилагая усилия, чтобы совладать с пальцами, вытащил его из конверта.
\par
Его глаза застлала мутная пелена. Текст послания был немного смазан, а часть чернил внизу страницы сильно растеклась из-за обилия пролитых на неё слёз. Робин начал заново его перечитывать, хотя этого и не требовалось — написанное он выучил наизусть:\\
\par
«Моим любимым студентам.\\
\par
Я знаю, что подвёл вас, и не хочу, чтобы вы расстраивались из-за меня. Вчера утром я не заменял другого преподавателя — простите меня за этот обман. Правда заключается в том, что я пытался донести до британского премьер-министра ставшие мне известными факты, но потерпел крах.
\par
Не теряйте надежды и веры в свои силы. Я всегда буду с вами, буду присматривать за каждым из вас и оберегать. Искренне верю, что когда-нибудь мы обязательно воссоединимся.
\par
Мне было необычайно приятно работать с вами все эти последние годы. Продолжайте так же хорошо трудиться и впредь, и не позволяйте пропасть тому, чего мы сумели достигнуть.
\par
Возможно, вам удастся найти способ предотвратить надвигающуюся ядерную катастрофу.
\par
Элисон, Гарри, Мэтью! Пожалуйста, сделайте всё возможное, чтобы уберечь Робина — он очень особенный мальчик, и ему требуется много внимания и заботы.
\par
Мне кажется, это всё, что я хотел до вас донести. Поверьте, мне искренне жаль, что всё закончилось именно так. Я больше не мог продолжать нашу дальнейшую работу, поскольку был уверен, что после вчерашнего дня меня обязательно найдут и заставят замолчать.\\
\par
Люблю вас всех!\\
\par
Искренне ваш,\\
\par
Профессор Фергер».\\
\par
Робин поместил письмо в конверт и убрал его в верхний ящик своего письменного стола.
\par
Какое-то время он тихо сидел на своей кровати, прислушиваясь к тому, что происходило в доме. Затем, убедившись, что никто не зайдёт в его комнату, покопался во внутреннем кармане своего жакета и извлёк из него ещё один конверт, на котором было написано его имя. Об этом письме он не сообщил никому.
\par
Профессор, должно быть, понимал, что его смерть заставит страдать Робина сильнее всех остальных студентов, и такое трепетное отношение растрогало чувства юноши. В своём письме к нему профессор во всех подробностях объяснял причины своего добровольного ухода из жизни.
\par
Когда Робин стал одним из членов исследовательской группы по изучению аномальных явлений в Саутгемптонском университете, именно профессор взял на себя всю ответственность за его дальнейшую судьбу. Профессор Фергер относился к нему как к своему собственному сыну. Из всех знакомых людей, профессор был самым близким для Робина человеком. Иногда ему казалось, что ни один человек на планете не в силах понять его лучше, чем профессор. Среди прочих он чувствовал себя чужаком, белой вороной, как будто не находился с ними на одной ступени социальной лестницы, а располагался ниже, значительно ниже. С профессором всё было наоборот — в его обществе Робин никогда не чувствовал себя изгоем.
\par
Он открыл конверт, извлёк из него письмо, и начал читать. Это было всё, что осталось после профессора, и он ни за что не позволил бы ни одному человеку забрать у него это сокровище.\\
\par
«Дорогой Робин,\\
\par
Каким-то образом я понял, что именно ты первым найдешь меня, поэтому не увидел ничего плохого в том, чтобы написать лично тебе. Другое письмо немного запутано. Я не знал, как выразить свои мысли, чтобы они стали понятными для остальных, ведь они не понимают меня так, как ты. Робин, я люблю тебя! Пожалуйста, не забывай об этом. Когда я писал, что всегда буду с тобой, то имел в виду именно это — не забывай обо мне и тогда мы навсегда останемся вместе! Никто на Земле не понимает тебя так хорошо, как я. Оставляю тебе на память свою любовь и свою рассудительность — они помогут тебе стать взрослее.
\par
Изо всех сил я старался донести до правительства, что всё сказанное мной — чистая правда. Но они стали издеваться, обращаясь со мной как с психом. Им стало известно моё имя, и мне пришлось уйти из жизни, чтобы избавить от преследования себя и вас. Но я не сказал им ни слова ни о тебе, ни о других.
\par
Ты должен понять, что после сегодняшнего провала мы не смогли бы избежать преследования. В результате они приложили бы все усилия, чтобы выяснить, как я пришёл к такому выводу, и в конце концов, узнали бы обо всех вас. Ты, как и все остальные, подвергся бы преследованиям.
\par
Пожалуйста, не теряй надежды! Всеми своими силами ты должен хотя бы попытаться предотвратить приближающуюся катастрофу. Крайне важно, чтобы никто не узнал о твоих особых талантах. Люди не станут заботиться о тебе, и даже не приютят, как это сделал я. Для них ты станешь прославленной морской свинкой, а вовсе не человеком, имеющим право чувствовать и рассуждать.
\par
За последние несколько лет изучения парапсихологии и связанных с ней способностей разума я собрал обширную коллекцию файлов, в которых содержится множество полезной информации. По возможности, забери их себе. Вероятно, некоторая информация окажется для тебя полезной. Среди них также есть множество заметок о механизмах загрузки и применения ядерного оружия. Они хранятся под половицами, возле шкафа в моей спальне.
\par
Я знаю, сейчас ты думаешь о том, что я подвёл всех вас, но, пойми, я не мог поступить иначе и продолжать исследования делая вид, будто всё идёт своим чередом. Ты должен понять — я пытался, но у меня не получилось. Честное слово.
\par
Будь терпелив по отношению к остальным студентам. Они заботятся о тебе, и они будут защищать тебя от любой агрессии внешнего мира до тех пор, пока ты не станешь достаточно сильным, чтобы суметь позаботиться о себе.
\par
Я чувствую, как сказывается на мне передозировка инсулина и вряд ли смогу написать что-то ещё.
\par
Помни, ты не должен терять надежды теперь, когда меня нет рядом. Верю в твои силы!\\
\par
\par
Пожалуйста, не забудь,\\
\par
Я люблю тебя и всегда буду с тобой,\\
\par
Искренне преданный тебе,\\
\par
Джозеф Фергер.»\\
\par
Внизу страницы почерк стал совсем неразборчивым. Его рука, должно быть, сильно дрожала.
\par
«Профессор, я спасу мир. Верьте мне» - прошептал Робин самому себе,
\par
«Если бы вы не просили об этом, то я бы не стал беспокоиться — им не нужно жить, они заслужили смерть.»
\par
По его щекам потекли слёзы. Он тяжело вздохнул и продолжил:
\par
«Но когда я выполню эту задачу — обязательно отомщу всем этим так называемым политикам. Они дорого заплатят за то, что довели вас до самоубийства» - поклялся Робин.
\par
Он вложил письмо обратно в конверт, положил его в верхний ящик своего секретера и запер его, чтобы никто не смог посмотреть его личные записи.
\par
Он вышел из спальни и спустился вниз, чтобы узнать, чем заняты остальные.
\par
«Как ваши дела? Извините, что я так долго был наверху, но мне нужно было побыть какое-то время одному. Как себя чувствует Симпсон?»
\par
Гарри, Мэтью и Элисон обменялись непонимающими взглядами.
\par
«Чёрт побери, что ещё за Симпсон?» - подумал Гарри.
\par
«О! Разве вы не знаете, кто такой Симпсон?» - спросил Робин,
\par
«Так зовут собаку профессора. Он считал Симпсона кем-то вроде дворецкого и научил его забирать почту у дверей по утрам. Но всё было не совсем так, как вы представили. Вместо того, чтобы носить почту в зубах, Симпсон складывал письма на серебряное блюдце и аккуратно, не рассыпая их, нёс профессору, который завтракал в это время в столовой. Было довольно забавно наблюдать, как пёс пытался дотащить до него тяжёлое пальто Джозефа. Из под него выглядывал его милый носик и пытался отыскать путь, не опрокинув при этом стола, стоявшего в коридоре.»
\par
Робин выглядел рассеянно, вспоминая о тех прекрасных днях, когда он жил с профессором Фергером. Лишь недавно он переехал и стал жить вместе с Мэтью.\\
\par
«Робин, может быть, ты поможешь нам подбодрить Симпсона?» - спросила Элисон,
\par
«Он совсем ничего не ест, а только лежит в углу кухни и постоянно скулит.»
\par
Она надеялась, что Робин не будет чувствовать себя таким подавленным, если ему поручат присматривать за собакой. Возможно, ему хотелось бы получить то, что было личным для профессора Фергера, когда он был жив. Во всяком случае, профессор хотел бы, чтобы Симпсон достался Робину.
\par
Все знали, что Робин был для профессора особенным, что мальчик был его любимцем. Старик относился к нему так, словно он был его родным сыном. У профессора Фергера не было своей семьи. Он был женат один раз, и у него был сын, но их отправили в газовые камеры во время гитлеровского террора в тысяча девятьсот тридцать девятом году, только потому, что они оказались евреями.
\par
Убийство жены и ребёнка опустошило профессора. Он так и не смог оправиться от печали, которую вызвала их смерть. Робин был для него всем тем, чем, как он надеялся, станет его сын. Робин обладал теми качествами, которые он хотел бы видеть в собственном сыне. Кроме того, профессор и Робин понимали друг друга, и им никогда не приходилось объяснять друг другу свои поступки или мнения.
\par
Потеряв жену и ребёнка, профессор сумел бежать в Соединённые Штаты Америки. Там он работал над "Бомбой" вместе с одним коллегой — человеком, которого впоследствии назвали её отцом. Он и его друг пришли в ужас, когда их творение было применено против Нагасаки и Хиросимы. Они и представить себе не могли, что последствия радиации будут настолько сильными и унесут жизни многих людей.
\par
Он не сумел простить себя за участие в разработке оружия, которое принесло людям горе и боль.
\par
Вскоре после падения бомб профессор Фергер оставил друга и эмигрировал из Америки, чтобы жить в Англии. Они не теряли связи друг с другом до тех пор, пока его друг не скончался от рака горла.
\par
Друга обвинили в передаче русским секретных документов, касающихся бомбы. После этого ему пришлось пережить страшное время преследования со стороны американского правительства. Предположительно, оно и стало причиной рака.
\par
Страх профессора быть преследуемым, который в конце концов заставил его покончить с собой, несомненно, был вызван тем, что он увидел, что стало с другом.
\par
Когда профессор приехал в Великобританию, он начал работать с группой людей, с которыми общался, когда жил в Америке. Он познакомился с ними на Всеобщей Конференции, посвящённой парапсихологии и её будущему. Ему довелось работать в нескольких учреждениях, исследуя различные сферы своей специализации. В 1977 году, в год Серебряного юбилея королевы Елизаветы Второй, профессор Фергер получил все полномочия по созданию Отделения Парапсихологии, главой которого он должен был стать.
\par
Он видел много студентов, которые приходили и уходили, но Гарри, Мэтью, Элисон и, в отдельности, Робин стали для него особенными. В отличие от других групп студентов, которые у него были, эта группа работала единой командой и была хорошими друзьями за пределами лаборатории. Они подбадривали друг друга перед началом любого эксперимента, каким бы простым он ни был. Кроме того, они заботились друг о друге и действовали как единое целое.\\
\par
Студенты мечтали, чтобы профессор Фергер получил Нобелевскую премию за большие успехи, которых он добился в своей области — парапсихологии. Без пристального внимания профессора и его уроков никто из них не обладал бы теми способностями, которые они имели сейчас. Всеми своими навыками они были обязаны его стараниям.
\par
Их огорчало, что никто не воспринимает парапсихологию как настоящую науку. Другие, более "традиционные" учёные считали, что если кто-либо не в силах объяснить полученные в ходе эксперимента результаты, то они должны быть признаны недействительными. Они утверждали, что даже записанные на плёнку эксперименты проводились при помощи трюков с видеокамерой. Не смотря на все старания в попытках успокоить учёных, всегда находился человек, остававшийся недовольным, и он, как правило, был влиятельным, уважаемым деятелем в мире "ортодоксальных", признанных наук.\\
\par
«Я знаю, что все мы расстроены сегодняшними событиями, но мы не должны позволять нашим эмоциям отделять нас от великой задачи, которая стоит перед нами. Нам крайне необходимо выполнить все пожелания профессора» - сказал Робин.
\par
Теперь это стало его призванием — сделать так, чтобы мир был спасён от Третьей мировой войны, которая должна была произойти совсем скоро. Он делал это ради профессора, а не ради жителей планеты. По его мнению, они не заслуживали ничего лучше, чем умереть или пережить все тяготы и ужасы, которые принесёт с собой ядерная катастрофа.\\
\par
«Ну, и что ты предлагаешь нам делать? Профессор пытался донести свои знания до британского правительства, но они его не послушали и даже не стали воспринимать всерьёз. В любом случае, я не думаю, что им следует давать возможность иметь привилегию на жизнь» - с уверенностью заявила Элисон. Эмоции заставили её увлечься.
\par
«Элисон, ты делаешь именно то, чего я тебе не советовал делать. Ты позволяешь своим чувствам взять верх над тобой. Мы сделаем это не ради людей, а потому, что так хотел профессор. Мы сделаем это в память о нём, а не по какой-то другой причине. Я чувствую то же самое, что и ты, но именно потому, что он хотел этого, я хочу, чтобы мы хотя бы попытались.»
\par
«Помните, он всегда будет среди нас, присматривая за нами. Хотите верьте, хотите нет, но то, что сделал профессор, было для нашего блага. Он всего лишь пытался защитить нас от преследования. Он не смог бы простить себя, если бы нас преследовали и обижали люди, которые не понимают нас. Он сделал то, что посчитал лучшим для нас, он спасал нас от страданий и трудностей, которые нам было бы очень трудно перенести.»
\par
Пока Робин говорил, стояла полная тишина. Его проникновенная и искренняя речь вызвала у них сильные эмоции. Все они были близки к тому, чтобы расплакаться.
\par
«Ты прав, Робин. Мы постараемся сделать всё возможное, чтобы контролировать свои чувства и не дать им помешать нашей работе. Итак, что ты предлагаешь нам делать?» - спросил Мэтью, прикуривая сигарету от своей зажигалки "Зиппо".
\par
«Во-первых, мы должны вернуться в дом Джозефа и забрать имеющиеся там файлы. Я знаю, где они спрятаны. Они нужны нам — они сэкономят время.»
\par
«О каких файлах ты говоришь, Робин? Мы ничего не знаем ни о каких файлах» - спросила Элисон.
\par
«Профессор собрал коллекцию личных файлов. В них содержится информация, о которой он не хотел, чтобы кто-то знал. Судя по тому, что я узнал от профессора, они будут полезными. Возможно, основываясь на этой информации, мы сможем придумать способ, как нам предотвратить войну.»
\par
«Легко сказать, но трудно сделать, Робин. Там может оказаться полисмен. Если профессор привлёк к себе внимание правительства, то его дом может охраняться до тех пор, пока его не обыщет правительственный чиновник.»
\par
«Гарри, если это так, то этим вечером, когда стемнеет, нам будет проще. В любом случае, я уверен, мы как-нибудь проберёмся внутрь.»
\par
«Это звучит достаточно разумно, но есть одна вещь, которая меня удивляет. Мне любопытно, откуда ты так хорошо осведомлён обо всех этих вещах, и так хорошо понимаешь, почему профессор сделал то, что он сделал? Он общался с тобой прошлым вечером?» - спросил Мэтью. Он не мог понять, почему Робин так много знал о профессоре.
\par
«Когда я нашёл профессора утром, он сжимал в руках три конверта. В одном из них было предсмертное стихотворение, во втором — письмо для всех нас, а ещё один он адресовал лично мне. Именно в этом письме он объяснил, почему покончил с собой, и рассказал о существовании файлов. Он не стал бы упоминать о них, если бы они не были нам полезны. Кроме того, он знал, что там могут быть полицейские, и не хотел, чтобы мы попали в опасную ситуацию, если только она не имеет большого значения.»
\par
Робин прервался и посмотрел на остальных.
\par
«Я ответил на твой вопрос, Мэтью? У кого-нибудь из вас есть другие вопросы? Я не хочу, чтобы между нами были секреты. Профессора с нами больше нет, и будет лучше, если мы останемся как можно ближе друг к другу. Есть возражения или идеи?»
\par
Робин сделал паузу и внимательно посмотрел на выражения лиц своих друзей.
\par
«Придётся отступить» - подумал он.
\par
«Я слишком сильно давлю на них. Я не должен позволить им узнать, насколько силён на самом деле. Мне нужно, чтобы они продолжали заботиться обо мне.»
\par
Он был прав. Им не понравилось бы, если над ними стал бы командовать мальчик одиннадцати лет.
\par
«Хорошо, Робин» - сказал Гарри,
\par
«Если ты уверен, что хочешь пройти через это… Сейчас четверть девятого. Нам пора выдвигаться — пока мы доберёмся до дома профессора, уже стемнеет.»
\par
Гарри был следующим после Робина. Он тоже был довольно близок к профессору Фергеру. Из всех студентов он дольше всех проводил время с ним, исследуя парапсихологию. Теперь он хотел выполнить последнее задание профессора. Тот много раз говорил Гарри, что не хочет ничего иного, кроме исполнения своей заветной мечты — спасти мир от Третьей мировой войны.\\
\par
«Я уверен, что говорю от имени остальных, когда говорю, что мы хотим только одного — помочь тебе предотвратить эту трагедию. Ради профессора Фергера» - сказал Гарри.
\par
Робин улыбнулся, как и все остальные. Если бы профессор был рядом, он бы гордился тем, как они стали едины.
\par
«Мне лучше не изображать из себя руководителя этой операции» - подумал Робин,
\par
«Мне будет гораздо легче заставить их делать то, чего я хочу, если они будут думать, что самые лучшие идеи принадлежат им. Кроме того, мне следует очень подробно объяснять им любые замыслы. Тогда у них не будет ощущения, что я от них отделяюсь и не забочусь об их чувствах. Будь с ними терпелив — вот что написал Джозеф в своём письме.»
\par
«Ну что, пойдём?» - спросил Мэтью. Он встал, взял пачку сигарет, зажигалку и положил их в карман пиджака. Элисон и Гарри также поднялись со своих мест.\\
\par
Робин закрыл заднюю дверь, надел свой жакет и погладил Симпсона:
\par
«Не грусти, пёсик. Джозеф хотел этого. Сейчас я вернусь в его дом. Вот что я тебе скажу: я принесу твою миску, корзину и одеяло. Что думаешь? Я позабочусь о тебе и сделаю так, чтобы ты чувствовал себя здесь как дома» - прошептал Робин, легонько поцеловав морду собаки.
\par
«Пошли, Робин. Мы ждём тебя!» - окликнул Гарри, надевая пальто.
\par
«Уже иду. Подожди, мне нужно сбегать наверх за курткой.»
\par
«Не нужно. Я уже отнесла её вниз вместо тебя» - ответила Элисон.
\par
«Где же она?»
\par
«На вешалке, где её полагается оставлять сразу после того, как входишь в дом.»
\par
«Извини, Элисон. Я постараюсь это запомнить. Хотя, ты ведь знаешь, какой я рассеянный, поэтому тебе, наверное, придётся напоминать мне об этом время от времени» - улыбнулся Робин, натягивая пальто.
\par
Они вышли из дома, оставив свет в гостиной и заперев входную дверь.
\par
«На чьей машине поедем и кто сядет за руль?» - спросил Мэтью.
\par
«Веди ты, Мэтью — я слишком устал. Но если хочешь, я поведу туда, а ты — обратно» - ответил Гарри.
\par
Элисон задрожала, обвив себя руками и переступив с ноги на ногу:
\par
«Решайте быстрее. Мне холодно стоять здесь. Пойдём, Робин, посидим и подождём в машине. Они могут решать целую вечность, а к тому времени уже рассветёт.»
\par
«Хорошо, Элисон. Вопрос решён. Гарри может вести туда, а я поведу обратно» - сказал Мэтью.
\par
«Наконец-то! Решение! Неужели чудеса никогда окончатся, спрашиваю я себя» - ответила Элисон.
\par
Гарри снял с блокировки двери машины и уселся за руль.
\par
«Как вы думаете, сколько времени это займёт?» - спросил Мэтью.
\par
«Зависит от того, есть ли в доме полиция» - ответил Гарри.
\par
«Честно говоря, я сомневаюсь, что они будут там» - заметила Элисон.
\par
«Почему ты так говоришь, Элисон?» - спросил Робин.
\par
«Ну, тело забрали сегодня утром, и судмедэксперты сделали всё, что хотели. Было совершенно ясно, что профессор покончил с собой.»
\par
«О! Я не подумал об этом. Хорошая мысль, Элисон.»
\par
«Ладно, мы решили, что полицейских там не будет, но осталась проблема с поиском файлов. Я полагаю, что если они были так ценны для профессора, то он спрятал их хорошо» - сделал заключение Мэтью.
\par
«Это не проблема. Я знаю, где они. Профессор сообщил мне об этом в своём письме. Они находятся под под половицами в его спальне.»
\par
«Отлично! Значит, нам придётся разобрать пол, чтобы добраться до них. Это займёт много времени. Если полиция снова осмотрит дом, то они обязательно заметят, что в нём кто-то был.»
\par
«Единственная половица, которую нам придётся поднять — та, что напротив старинного антикварного шкафа. Но вы правы — нам придётся не оставить никаких следов нашего визита. Мы должны быть осторожными и оставить всё как есть. О, но я обещал Симпсону вернуть его вещи. Ну, я сомневаюсь, что они заметят, если я заберу их.»
\par
«Нет, Робин. Это слишком рискованно. Завтра ты можешь пойти в полицейский участок и спросить, позволят ли тебе забрать то, что принадлежит собаке» - заявил Гарри.\\
\par
Остаток пути до Уинчестера они ехали в тишине. Уже совсем стемнело и появился туман. Гарри вставил кассету в магнитолу и прибавил громкость.
\par
«О нет! Гарри, пожалуйста, не нужно играть в вокалиста. Твой голос ни за что не сравнится с голосом Джимми Самервилля» - кричала Элисон сквозь шум. Никто из них не был против того, чтобы послушать иногда "Коммунаров", но единственное, без чего они могли обойтись — аккомпанемент Гарри.
\par
«Простите, ничего не могу поделать. Мне нравится эта песня, но я постараюсь не петь, если это вас так беспокоит.»
\par
«Гарри, поверни направо — и мы на месте» - сказал Робин.
\par
«Нам лучше медленно объехать квартал, на случай, если поблизости окажется полиция» - добавил Мэтью.
\par
Гарри сделал так, как он сказал. Судя по всему, там не было никакой активности, а свет в доме был выключен.
\par
«Который час?» - спросила Элисон.
\par
«Начало десятого. Давайте припаркуем машину у дома» - ответил Гарри.
\par
Робин молчал, временно позволив студентам руководить, решая несущественные вопросы. Он вмешивался или возражал, только если считал, что они не были правы.
\par
Они вышли из машины и подошли к задней части дома. Попытавшись открыть кухонную дверь, они обнаружили, что она была заперта.
\par
«Как мы попадём внутрь? Мы же не можем взломать дверь» - произнесла Элисон, осматривая дом,
\par
«Вообще-то… Я не знаю, возможно способ есть.»
\par
«Какой?» - спросил её Робин.
\par
«Ну, если вы посмотрите вот туда, то заметите, что окно в ванной слегка приоткрыто. Если бы ты превратился в птицу, то мог бы влететь туда и попасть в дом. Оказавшись внутри, ты мог бы превратиться обратно в себя, а затем спуститься и открыть нам входную дверь.»
\par
Наступила тишина, во время которой все обдумывали идею Элисон.
\par
«Это должно сработать. Как ты думаешь, Робин, ты сможешь это сделать?» - спросил Мэтью.
\par
«Без проблем. Просто смотрите на меня» - уверенно ответил Робин. Почти сразу же на их глазах он превратился в орла. И всё же это зрелище показалось им впечатляющим — хотя они и сами могли превращаться в любых животных, но это занимало у них больше времени.
\par
В мгновение ока Робин влетел в приоткрытое окно, снова стал собой и открыл входную дверь.
\par
«Идёмте наверх» - сказал Робин.
\par
Они последовали за ним по лестнице и вошли в спальню профессора.
\par
«Вот и шкаф. Ты сказал, что файлы находятся под половицами? Хорошо, значит они должны быть прямо здесь» - произнёс Мэтью, указывая на то место, где стоял.
\par
Осторожно, они сняли половицу. Все посмотрели в отверстие. Там находилось то, что они искали — файлы.
\par
«Давайте вытащим их и уйдём из этого дома. Мне здесь не по себе» - сказала Элисон и по её спине пробежала дрожь. Она почувствовала, как встали дыбом короткие волоски на обратной стороне её шеи.
\par
Всего было шесть папок. Некоторые из них выглядели довольно старыми. Каждая папка была набита бумагами, пахнущими плесенью и покрытыми пылью. На лицевой стороне одной из них Робин пальцем написал своё имя. Гарри пару раз чихнул, когда пыль случайно попала ему в нос.
\par
Мэтью вернул на место половицу, надёжно закрепив её в том же виде, в котором они её нашли. Затем каждый взял по две папки и понёс вниз по лестнице. Они захлопнули за собой входную дверь, положили папки в багажник машины и поехали обратно в Саутгемптон — настолько быстро, насколько позволяли ограничения скорости.\\
\par
Вернувшись, они занесли файлы в дом и сложили их в гостиной.
\par
«Прежде, чем мы начнём это читать, я предлагаю закинуть в себя что-нибудь горяченькое. Я умираю от голода».
\par
Желудок Гарри жалобно заурчал при мысли о встрече с едой.
\par
«Как ты можешь думать о том, чтобы набить свои щёки в такой момент. У нас колоссальный объём работы» - огрызнулся Мэтью.
\par
«Мэтью, не стоит так отвечать. Мы не сможем хорошо и долго работать, если не получим нормальной еды. В любом случае, на минуту мне показалось, что Гарри вышел за рамки приличия, но я ошиблась. Пойдёмте та кухню — там мы сможем поговорить, пока я готовлю.»
\par
Элисон развернулась и отправилась на кухню, а остальные быстро последовали за ней.\\
\par
Элисон заглянула в холодильник. Всё, что ей удалось в нём найти — это полдюжины яиц и несколько ломтиков копчёного бекона.
\par
«Ну… у нас нет большого выбора, что съесть этим вечером. Либо яичница с беконом, либо бекон с яичницей. Итак, что вы хотите?» - сказала она, доставая бекон и яйца.
\par
Остальные с аппетитом уселись за стол.
\par
«Не думаю, что кому-то из нас есть до этого дело. Если говорить за себя, то я сильно голоден и мне всё равно, что мы будем есть — лишь бы побыстрее» - ответил Робин.
\par
Мэтью и Гарри покивали в знак согласия.
\par
«У нас есть шансы съесть также немножечко обжаренного хлеба, Элисон? … Пожа-а-алуйста» - спросил Гарри своим вкрадчивым голосом.
\par
«Хорошо, можешь съесть и жареный хлеб — если ты найдёшь хлеб. Не сидите просто так, бекон и яйца скоро будут готовы. Достаньте тарелки и столовые приборы. Или вы ждёте, что это тоже сделаю я?»
\par
Робин поднялся и вытащил тарелки, а Мэтью разложил столовые приборы.
\par
«Кто-нибудь хочет чаю?» - спросил Гарри, чувствуя себя виноватым, глядя, как остальные занимаются чем-то полезным.
\par
«Это не такая уж плохая идея. Но я бы предпочла чашку кофе» - ответила Элисон.
\par
После сытного ужина все уселись поудобнее, их желудки были вполне довольны. Они погрузились в блаженство, попивая кофе и чай.
\par
«Раз уж ты готовила для нас, Элисон, то присядь и отдохни. Мы вымоем и вытрем посуду» - предложил Робин.
\par
«Очень мило с твоей стороны, Робин, но ты совсем не обязан…»
\par
«Нет, это самое меньшее, что мы можем сделать. Ты устала так же, как и мы. А теперь, не спорь, мы начнём уборку» - сказал Мэтью.
\par
Он подумал, что идея Робина — хороший способ показать Элисон, как они ценят то, что она приготовила для них вкусную еду.
\par
Когда они готовили самостоятельно, пища всегда получалась подгоревшей.

\chapter{ОТКРОВЕНИЕ}
\noindent\parП{\scriptsizeРИБЛИЖАЛАСЬ ПОЛНОЧЬ, НА ЧАСАХ БЫЛО ПОЧТИ БЕЗ ЧЕТВЕРТИ ДВЕНАДЦАТЬ. НИКТО НЕ ЗАМЕТИЛ, КАК БЫСТРО ПРОЛЕТЕЛО ВРЕМЯ}, пока все сидели на кухне.\\
\par
В гостиной было довольно прохладно. Мэтью включил электрический камин, и студенты расселись вокруг него. Немного погодя, Робин поднялся со своего места и вернулся на кухню.
\par
«Пойдём, Симпсон. Сегодня вечером я был эгоистом и совсем не заботился о тебе. Проходи в гостиную, чтобы посидеть с нами перед тёплым огоньком» - ласково произнёс Робин, нежно поглаживая собаку. Он взял его на руки и отнёс в гостиную, уложив напротив источника тепла.\\
\par
«Теперь, когда Робин здесь, мы можем начать просматривать эти папки? Я предлагаю каждому взять по одной и медленно читать их, отмечая то, что может оказаться полезным.»
\par
Мэтью встал и раздал всем по папке.
\par
«Итак, у нас имеется восемь папок — значит по две на каждого. Вот, держите — каждому по две. Робин, не хочешь принести нам бумагу и ещё что-нибудь, чем можно писать? В нижнем ящике секретера должно лежать что-то похожее.»
\par
Мальчик сорвался с места и принёс блокноты и ручки, как его и просили.
\par
Открыв свою папку, Робин с удивлением обнаружил в ней исчерпывающий указатель.
\par
«Что ж, это будет гораздо проще, чем я подумал. С помощью этого указателя мы сможем почти без проблем возвращаться к самым интересным заметкам. Он начинается с главного указателя, в котором отмечено, на какой странице располагается каждая тема. Затем следует другой указатель, который показывает точную страницу и строку той или иной подтемы. И, наконец, вишенка на торте — третий указатель, в котором есть наиболее интересные примечания со ссылками на страницы и номера строк. Посмотрите, он пронумеровал каждую строку на каждой странице. Боже, эти заметки так красиво составлены, что нам будет очень легко разобраться в них.»
\par
«Да… Смотрите, это не стенография и не немецкий язык. Это тоже помощь» - прокомментировал Гарри, быстро перелистывая страницы.\\
\par
Все они с трепетом рассматривали заметки и указатель.
\par
После того как студенты преодолели своё восхищение перед профессорским методом ведения записей, они приступили к чтению. Наступила безмолвная тишина, во время которой слышалось лишь тиканье часов и, иногда, их мелодичный перезвон.\\
\par
Симпсон задремал. Несколько раз он скулил во сне, после чего Робин ласкал собаку и шептал ей успокаивающие слова:
\par
«Всё будет хорошо, Симпсон. Мы будем заботиться о тебе. На самом деле Джозеф не покинул нас. Он повсюду вокруг и присматривает за нами.»\\
\par
Скрип ручек, двигавшихся по чистым листам бумаги, казался студентам намного громче обычного.
\par
Часы ещё не пробили два часа ночи, когда молчание, наконец, было нарушено. Первым заговорил Мэтью:
\par
«Если мы найдём что-то важное, то просто запишем это, и когда все мы полностью закончим просматривать файлы, обсудим то, что нашли? Или мы будем обсуждать основные моменты по ходу дела?»
\par
«Ох, я не подумала об этом. Я думала, что сейчас мы просто будем делать заметки, а затем, когда закончим, обсудим их. А ты что думаешь, Робин?»
\par
«Ну, Элисон, я думаю, что нам следует обсуждать всё, что мы посчитаем важным, сразу же, как только мы это обнаружим. Так мы не будем терять время, продираясь через массу сведений, которые могут не иметь никакого отношения к нашей главной цели.»
\par
«Да, Робин прав.»
\par
«Хорошо, Гарри. Поскольку мы сошлись во мнении, я зачитаю то, что нашёл.»
\par
«Заголовок гласит — "Нострадамус". В соответствии с этим, по-видимому, существует несколько начальных дат, от которых можно рассчитать приблизительное время, в котором сбудутся следующие пророчества. Вот они:\\
\par
а) В послании Генриху II Нострадамус утверждает, что начальной датой является четырнадцатое марта одна тысяча пятьсот пятьдесят седьмого года.
\par
б) Но чтобы запутать ситуацию ещё больше, Нострадамус также говорит, что дата начала отсчёта — от Дня сотворения мира, то есть от четыре тысячи четвёртого года до нашей эры, если верить словам архиепископа Ашшера, написавшего об этом в одна тысяча шестьсот пятидесятом году.
\par
в)  Человек по имени Воллнер говорил, что, по его мнению, начальная дата — это четыреста восемнадцатый год до нашей эры.
\par
г) Эта дата — последняя. По-видимому, широко распространено мнение, что датой начала являлся Никейский собор в триста двадцать пятом году нашей эры.
\par
Здесь он перечислил стихи — что-то вроде загадок — под которыми профессор написал своё собственное заключение о том, к чему, по его мнению, они относятся. Я сначала прочитаю стихи, а потом расскажу, о чём он написал.»
\par
Прежде чем прочитать их вслух, Мэтью прокашлялся и набрал полную грудь воздуха.
\begin{quote}
Бедствия прошлого уменьшили мир,\\
Долгое время — покой на обитаемых землях:\\
Люди пойдут по воздуху, земле и воде,\\
Затем снова начнутся войны.\\
\par
Оружие будет долго сражаться в небе,\\
Древо посреди города упадёт:\\
Крысы, болезни, сталь пред лицом революции,\\
Когда Италия падёт.\\
\par
Ночью над Нантом появится радуга,\\
Из моря поднимутся фонтаны воды;\\
Великий флот потопят в Персидском заливе,\\
В Германии — монстр, медведь и свинья.\\
\par
Подумают, что ночью увидели солнце\\
Глядя на получеловека-полусвинью:\\
Шум, крики, битву заметят в небе,\\
Послышится говор диких зверей.\\
\par
Великий голод, вызванный мором,\\
Затяжным дождём Северного полюса:\\
Сэм Р. О’Брайан в ста лигах от земли,\\
Будут жить без закона и вне политики.\\
\par
Когда документы закроют в железной рыбе,\\
Из неё выйдет тот, кто начнёт войну,\\
Его флот будет путешествовать тайно,\\
Появившись возле Италии.\\
\par
Большое войско перейдёт через горы,\\
Боги войны управляют подводными суднами:\\
В головах лосося сокрыты яды,\\
Вождь зависит от нити Полярной звезды.\\
\par
Великой звезде гореть семь дней,\\
Из-за тучи выйдут два солнца:\\
Огромный мастиф провоет всю ночь,\\
Когда Папа Римский сменит страну.\\
\par
Во время явления бородатой звезды,\\
Трое великих правителей станут врагами,\\
Удар с неба по зыбкому покою земли,\\
По, Тибр в разливе, змеи на берегах.\\
\par
После потрясения человечества грядёт ещё большее\\
Великий двигатель возобновляет цикл веков:\\
Дождь из крови и молока, пепла, голод, война и болезни,\\
В небе будет виден огонь и длинные искры.\\
\end{quote}
\par
Мэтью сделал паузу и снова глубоко вдохнул, прежде чем продолжить.\\
\begin{quote}
Мабус придёт и вскоре умрёт,\\
Людей и зверей ужасная гибель,\\
Затем увидят внезапное отмщение,\\
Кровь, руки, жажда и голод, когда пролетит комета.\\
\par
Английский вождь надолго задержится в Ниме,\\
На помощь Испании Рыжебородый идёт:\\
Многие погибнут в войне, начатой в тот день,\\
Когда в Артуа упадёт бородатая звезда.\\
\par
Воинственный арабский принц, Солнце, Венера, Лев,\\
Христиане будут разбиты на море:\\
К Персии бежит почти миллион людей,\\
В Турцию и Египет вторгнется истинный змей.
\end{quote}
\par
«Ну, вот и всё. Что вы думаете?»
\par
«Похоже, Нострадамус считал, что нас ждёт катастрофа.»
\par
«Я прочту вам заметки профессора, которые следуют за этим» - произнёс Мэтью, не дав договорить Гарри.
\par
«Он датирует следующую запись февралём тысяча девятьсот восемьдесят первого года — »\\
\par
«Я изучаю стихи Нострадамуса последние десять лет, и теперь вижу, каким великим пророком он был. Оглядываясь назад, на его ранние стихи, я нахожу весьма точным его предсказание, гласящее, что "шах Персии будет схвачен теми, кто из Египта", и понимаю, насколько он был прав относительно американских заложников в Иране. Оба этих события, по словам Нострадамуса, произойдут в одна тысяча девятьсот восьмидесятом году.
\par
Проанализировав приведённые выше стихи, я думаю, что не ошибусь, говоря, что с тысяча девятьсот восемьдесят пятого года и далее между двумя сверхдержавами будет происходить острое словесное противостояние. Моя уверенность подкрепляется упоминанием Мабуса, которым, по всей видимости, является комета Галлея, на что намекают имеющиеся в тексте отсылки к Рыжебородому и бородатой звезде. Я искренне верю, что настоящая, полномасштабная война начнётся приблизительно в тысяча девятьсот восемьдесят восьмом — восемьдесят девятом году, но когда именно, я не знаю. Похоже, она начнётся в районе Персидского залива, после того, как будет потоплен военно-морской флот НАТО.»\\
\par
Дочитав абзац до конца, Мэтью остановился.
\par
«Боже! Это кажется таким реальным! Всё это прекрасно совпадает с содержанием той газетной статьи, которую нашёл Робин, находясь в Саутгемптоне будущего — о начале Третьей мировой войны, потому что русские решили перекрыть Ормузский пролив. Сразу после объявления войны был потоплен флот НАТО.»
\par
«А что насчёт упоминания о подводных судах и головах лосося, несущих яд? Это же так очевидно — тут речь о ядерной боеголовке, Элисон.»
\par
Робин кашлянул, чтобы привлечь внимание остальных.
\par
«Что такое, Робин? У тебя для нас что-то есть?»
\par
«Ну, Гарри… Что ты думаешь об этом? Подожди, я только проверю, правильно ли всё рассчитал…»
\par
Робин сделал паузу, нажимая кнопки на своём калькуляторе.
\par
«Возможно ли, что Нострадамус использовал Полярную звезду в качестве названия ядерных ракет\footnote[1]{UGM-27 «Polaris», «Полярная звезда» — двухступенчатая твердотопливная баллистическая ракета, способная нести ядерный заряд. Состояла на вооружении ВМФ США с 1961 по 1980 год. (Прим. пер.)}, использовавшихся на подводных лодках? Я только что подсчитал вероятность того, что это он придумал для них такое название. Это один шанс из девяносто пяти тысяч миллионов миллионов, что он сделал это по чистой случайности. Довольно точно для предсказания, сделанного несколько сотен лет назад, вы не находите?
\par
«Должен признать, это довольно впечатляюще» - ответил Гарри,
\par
«Я не нашёл ничего, о чём бы стоило упомянуть. Всё, что я прочитал — это записи его прежних студентов, на которых он учился — тех, что были до нас. Кто-нибудь из вас нашёл хоть что-то, что следует обсудить сейчас?»
\par
«Вообще-то, Гарри, если взглянуть на часы, выяснится, что сейчас половина четвёртого. Я очень устала и хочу спать. Думаю, что нам не стоит продолжать, иначе мы с большой вероятностью упустим какую-нибудь важную информацию. Если сейчас она попадётся мне на глаза, то, скорее всего, я её попросту не замечу.»
\par
«Я согласен с тобой, Элисон. Зачитавшись, я даже не заметил, что уже так поздно» - поддержал девушку Робин, поднимаясь со своего места. Он склонился над Симпсоном, и прошептал ему на ухо несколько ласковых слов.
\par
«Всё, Симпсон. Мы закончили и собираемся идти спать. Ты можешь спать в моей спальне. Элисон! Гарри! Почему бы вам не не остаться здесь на ночь? Гарри и Мэтью могут провести ночь в одной спальне. Тебе, Элисон, достанется другая, а Симпсон может поспать со мной. Вы только зря потратите время, если решите возвращаться к себе домой. Ну, что скажете? Согласны?»
\par
«Мэтью, ты не против того, чтобы поделить комнату с Гарри?» - спросила Элисон.
\par
«Конечно, всё нормально. Пойдем, Гарри — я принесу для себя раскладушку.»
\par
Робин выключил электрокамин и убедился, что вилки остальных электроприборов вынуты из розеток, после чего, взяв пса на руки, отправился наверх, в свою спальню.\\
\par
Несмотря на усталость, Робин никак не мог заснуть. Он чувствовал беспокойство и был уверен, что они упустили из виду что-то важное. Мальчик притянул собаку поближе к себе:
\par
«Не волнуйся, мы обязательно предотвратим истребление людей. Я пока что не знаю, как мы с этим справимся, но мы обязательно справимся…»
\par
Он прервался и сделал паузу.
\par
«Я знаю, что нужно сделать. Видимо, всё это чтение сегодняшним вечером мешает мне спать. Сейчас я встану и порисую. Надеюсь, это меня слегка отвлечёт, и сон обязательно наступит. Симпсон, держи для меня постель тёплой!»\\
\par
Робин поднялся с кровати и придвинул стул к стоявшему поблизости мольберту, после чего сел и занялся рисованием.
\par
Примерно через полчаса его рисунок стал обретать форму — он изобразил контуры лица юной девушки. Спустя ещё немного времени его первый шедевр был готов.
\par
Робина приятно удивил результат, и для его гордости имелись веские причины. Всем было хорошо известно, что мальчик не обладал талантом художника, хотя он любил подолгу рассматривать чужие картины, и часто грустил от невозможности сделать что-то похожее. Его карандашный рисунок — портрет девочки-подростка — был потрясающим.
\par
Робин слегка отклонился назад и осмотрел всю плоскость листа, восхищаясь тем, как точно было передано выражение лица. В глазах девушки отчётливо угадывалась грусть и сильное внутреннее напряжение.\\
\par
Чем больше Робин всматривался в своё творение, тем сильнее охватывало его беспокойство. С её лицом было явно что-то не так. Глаза… Это были глаза! В них было что-то необычное, что вынуждало Робина смотреть на них всё более пристально и продолжительно. В нижней части рисунка он разместил свою подпись и дату.\\
\par
Он глубоко зарылся под толстое и плотное одеяло. Всё, что прежде тревожило его разум, исчезло, и теперь он чувствовал себя спокойно и удовлетворённо. С улыбкой на лице он медленно погрузился в сон, размышляя о достоинствах своего рисунка.\\
\par
На следующий день Робин проснулся от того, что язык Симпсона облизал его лицо. Симпсон стянул с мальчика одеяло и громко залаял. Робин поднялся с кровати и вяло поплёлся вслед за собакой, вниз по лестнице. Они добрались до задней двери в кухне, и только тогда до Робина наконец-то дошло, чего хочет Симпсон.\\
\par
«Ах, ты бедная моя собачка. Тебе нужно выйти по нужде, да? Одно предупреждение — если получится, постарайся не сильно испачкать газон, иначе Мэтью будет страшно недоволен тобой. Он, вообще-то, скосил его впервые за много лет — одному Богу известно, за сколько именно. Поэтому, пожалуйста, сделай всё возможное, чтобы поддерживать на нём порядок, иначе Мэтью голову мне оторвёт.»
\par
С этими словами Робин открыл заднюю дверь и выпустил собаку наружу. Взглянув на часы он сразу же впал в уныние, обнаружив что они показывали половину восьмого — он поспал всего около двух часов.
\par
«Было бы слишком жестоко будить остальных» - устало подумал он,
\par
«Мне нужно просто попытаться снова заснуть.»
\par
Он поднялся по лестнице и подошёл к своему рисунку.
\par
«Боже! Это не было сном — я действительно это нарисовал. Хотя я не помню, как писал имя "Хелен" вверху рисунка. Но, может быть, это тоже сделал я — прошлой ночью я чувствовал себя настолько уставшим, что мог написать на нём всё, что угодно, и даже не запомнить этого» - подумал Робин. Он снова забрался под одеяло и погрузился в глубокий сон.\\
\par
Несколько часов спустя его разбудил голос Мэтью, стоящего возле двери в его спальню:
\par
«Робин, просыпайся! Элисон готовит для нас завтрак — через несколько минут он будет готов!»
\par
На часах было почти девять. Робин закончил одеваться и уже собирался спуститься вниз по лестнице, но в последний момент решил вернуться в спальню и прихватить с собой свой маленький шедевр, чтобы показать его остальным.
\par
Рисунок удивил их не меньше, чем самого Робина:
\par
«Ты не смог бы нарисовать это сам, ведь ты совсем не умеешь рисовать. Давай, открой нам секрет, кто это сделал? Нас не проведёшь, этот рисунок слишком хорош для тебя!» - рассмеялась Элисон.»
\par
«Честное слово — портрет нарисовал я. Можете спросить Симпсона. Я и сам был удивлён не меньше вашего. Наверное, мне просто повезло. Я пытался рисовать целую вечность, и, возможно, мои старания не прошли даром…»
\par
Робин умолк и нахмурился.
\par
«По какой-то непонятной причине я назвал её Хелен. Но я не помню, как писал это имя.»
\par
«Я верю тебе, Робин. Это я вчера наводил порядок в твоей спальне, и рисунка в ней не было.»
\par
«Спасибо, Мэтью. Хоть кто-то верит моим словам. В любом случае, хватит об этом! Итак, где еда? Она пахнет великолепно, а я умираю от голода!»\\
\par
Позавтракав, они переместились в гостиную. Их лица склонились над папками с документами, и в комнате отчётливо слышались звуки дыхания. Их мысли были сосредоточены только на чтении.
\par
Часы пробили три, но до сих пор не было произнесено ни одного слова. Наконец тишину прервал отдалённый лай Симпсона. Робин оставил остальных и пошёл на кухню.
\par
Симпсон был здесь — его влажный нос прижался к стеклу кухонной двери. Он перестал шуметь сразу же, как только заметил Робина. Мальчик открыл заднюю дверь, но Симпсон вернулся на кухню и снова продолжил скулить и лаять.
\par
«Я знаю, чего ты хочешь — ты голоден. Ох! Как же я виноват! Подумать только, я забыл тебя покормить! Мы ели сегодня утром, но мне и в голову не пришло, что я тебя не накормил. Сейчас посмотрю в холодильнике, что у нас есть для тебя.»
\par
Робин открыл холодильник — он был почти пуст. В нём находился кусок сыра, яйца и полфунта свежего фарша.
\par
«Ну, Симпсон, кажется, сегодня на ночь ты будешь есть фарш. Вот, бери… И ещё для тебя есть немного воды.»
\par
Робин вернулся в гостиную и продолжил чтение. После того, как Симпсон закончил есть, он пришёл в гостиную и свернулся калачиком возле ног Робина. Время шло очень быстро — студенты были так заняты работой, что не заметили, как наступила половина восьмого.
\par
«Может быть, это та информация, которую мы искали» - сказала Элисон,
\par
«Исходя из этого, возможно, мы сможем разработать стратегию, которая предотвратит войну. Я думаю, что нам будет нужно фактически предотвратить активацию и запуск ядерных ракет.»
\par
«Об этом легко говорить, Элисон, но совсем другое дело — уметь претворять слова в жизнь.»
\par
«Я знаю, Робин. Секрет в том, что нам не нужно пытаться обезвреживать по отдельности каждую ракету — этих чёртовых штуковин слишком много. Нет, судя по тому, что я прочитала, наш лучший выбор — это компьютеры…»
\par
«Элисон, ты не совсем ясно выражаешься. Компьютеры? Какое отношение они имеют к чему-либо?»
\par
«Мэтью, мне бы хотелось, чтобы ты немного притормозил и выслушал мои слова. Конечно, ты не увидишь связи, если будешь перебивать меня на каждом слоге.»
\par
Они замолчали и позволили Элисон завершить сказанное, не прерывая её.
\par
«Итак, на чём я остановилась? О, вспомнила — на компьютерах! Да, я имела в виду, что компьютеры управляют механизмами, позволяющими загружать боеголовки в ракеты и, затем запускать их. Если бы нам каким-то образом удалось вывести из строя один из главных компьютеров, то остальные последовали бы его примеру по принципу домино. Удар по компьютерам — единственный возможный метод. Таким образом мы сможем поразить все ракеты. Я знаю, что это всего лишь идея, и что она не принесёт нам пользы до тех пор, пока не появится способ её воплотить. Но, запомните мои слова — в конечном итоге мы будем атаковать именно компьютеры.»
\par
«Да, это идея» - задумчиво сказал Гарри,
\par
«Но можете ли вы представить, как охраняется такое место? Нам не проникнуть ни на одну базу. Мы все прекрасно понимаем, что не сможем преодолеть их систему защиты, что бы мы ни делали.»
\par
«Моя идея насчёт компьютеров — самый быстрый и, насколько я понимаю, единственный способ предотвратить гибель людей.»
\par
После того, как Элисон удалось высказаться, они продолжили чтение.
\par
Робин почувствовал беспокойство — он больше не мог полностью сосредоточиться на чтении. Он положил папку на пол и оставил закладку между страницами, чтобы не потерять то место, на котором остановился. Часы на здании административного центра пробили восемь раз. Робин поднялся наверх и вытащил свой мольберт, альбом для рисования и коробку с карандашами.
\par
«Прошлой ночью это сработало — я почувствовал себя гораздо спокойнее после того, как нарисовал тот портрет» - подумал Робин.
\par
Примерно через полчаса он сотворил ещё один шедевр. Картина получилась мрачной — на ней был изображён переживший бомбёжку город, с множеством мёртвых людей, лежащих на развалинах улиц.\\
\par
Робин показал свой рисунок остальным.
\par
«Посмотрите! Что вы скажете об этом? Теперь вы верите, что предыдущий рисунок нарисовал тоже я? Я был один в той же самой комнате и вам больше нечего возразить — вы должны мне поверить.»
\par
«Он немного ненормален, ты не находишь?» - заметил Гарри,
\par
«Ради всего святого, что заставило тебя нарисовать это? Он выглядит совсем не так, как Саутгемптон, каким мы его знаем. Интересно, что это за место?»
\par
«Честно говоря, Гарри, я не знаю, почему нарисовал это. Мне просто захотелось порисовать, и я не думал о том, что в итоге получится нечто подобное.»
\par
«Я знаю, где это… Это Нью-Йорк, я в этом уверен! Я жил там несколько месяцев до того, как переехал сюда.»
\par
«Ты уверен, Мэтью?»
\par
«Ну конечно, уверен. Иначе я не стал бы говорить об этом — ты отлично меня знаешь. Я никогда не говорю ни о чём, если не уверен в своей правоте на девяносто девять процентов.»
\par
«Нет, я так не думаю. В любом случае, сейчас примерно половина девятого, и я умираю с голоду. Нам нужно поесть перед тем, как мы продолжим работать. Мы ещё не обедали.»
\par
«Да, я тоже очень голоден. Мне было интересно, когда вы решите устроить перерыв» - ответил Мэтью. Они поднялись и дружно направились на кухню, а Симпсон засеменил возле Робина, следуя за его пятками.
\par
«Куда делся фарш?» - спросил Мэтью, отворачиваясь от почти пустого холодильника,
\par
«Я вчера купил фарш… Чёрт возьми, что с ним случилось?»
\par
«Фарш?»
\par
«Да, Робин — фарш. Ты меня не расслышал или в своё свободное время ты становишься попугаем?»
\par
«Не стоит переходить на личности. Иначе я не скажу тебе, что случилось с фаршем.»
\par
«Робин, я начинаю терять терпение. Если это розыгрыш, то мне совсем не смешно — я умираю с голоду! Хватит валять дурака и давай его сюда.»
\par
«Мэтью, мне неприятно об этом говорить, но я не могу вернуть его тебе.»
\par
«Что ты имеешь в виду, говоря, что не сможешь его вернуть?»
\par
«Именно то, что сказал. Его уже не вернуть. Видишь ли, он съеден.»
\par
«Кто его съел?»
\par
«Понимаешь, Симпсон был голоден и сегодня он ничего не ел. Мне больше нечего было ему предложить.»
\par
«Ты хочешь сказать, что нам придётся умереть от голода, потому что эта тупая собака съела наш фарш?»
\par
«Он — не тупая собака. Ты слишком эгоистичен — нельзя всё время думать только о своём животе. В холодильнике есть сыр и яйца.»
\par
«Мне надоело есть яйца.»
\par
«Вы оба — прекратите!» - прервал их Гарри.
\par
«Довольно разговоров о мясном фарше — нет смысла спорить о том, что уже произошло. Слушайте, мы с Элисон угостим вас обедом в "Пицца Хат" или в "Бургер Кинг". Решайте, куда мы идём. Хватит дуться! Пожмите друг другу руки! Ничего страшного не произошло!»
\par
Мэтью и Робин извинились друг перед другом и обменялись рукопожатием.
\par
«Я думаю, нам следует сходить в "Бургер Кинг", потому что у нас нет свободного времени. В "Пицца Хат" нам придётся ждать двадцать минут до того, как подадут еду. Что скажешь, Мэтью? Это был твой фарш, а я скормил его Симпсону.»
\par
Мэтью не стал возражать против выбора Робина.
\par
Гарри отвёз всех в "Бургер Кинг".\\
\par
Примерно полчаса спустя они выкатились из ресторана, наевшись до отвала и чувствуя тяжесть и дискомфорт в своих желудках.
\par
«Я съела слишком много. В моём животе — боль, и малейшее несогласованное движение только усиливает её. Всё, чего я сейчас хочу — это лечь и не шевелиться.»
\par
«Согласен с тобой, Элисон. У меня такое ощущение, словно мой живот увеличился в обхвате на несколько метров. Как будто внутри меня всей командой пинают футбольный мяч!»
\par
Пока Мэтью вёз их обратно домой, его пассажиры издавали нескончаемую череду стонов. Схватившись за животы, пошатываясь, они кое-как вылезли из автомобиля и, поднявшись по бетонной дорожке, растянулись на ближайших свободных креслах.
\par
«Я знаю — сейчас мы не в лучшей форме, но всё-же, думаю, нам следует продолжить чтение» - произнёс Робин,
\par
«Посмотрите на положительную сторону — по крайней мере, нам не придётся слишком много двигаться. Было бы хуже, если бы это было во время завтрака, и сразу после него нам пришлось бы делать какую-то работу по дому. Согласитесь, это было бы ужасно.»
\par
«Робин, мне всё равно. В таком состоянии я ни за что не стал бы делать ничего подвижного.»
\par
«Хорошо, Гарри. Твоя точка зрения мне ясна, но, пожалуйста, давайте продолжим работать.»\\
\par
Таким образом, с большим дискомфортом, они продолжили чтение. Вскоре неприятные ощущения перестали тревожить их, и стоны прекратились. Робин выпустил Симпсона в сад, оставив приоткрытой кухонную дверь. Когда Симпсон вернулся, Робин запер её, взял пса на руки и отнёс в гостиную, уложив рядом с камином. Пролистывая страницы в папке, Робин ласково поглаживал Симпсона.
\par
«Эй! Возможно, здесь у меня что-то есть.»
\par
«Господи! Робин, как ты меня напугал! Моя душа чуть не провалилась в пятки! Что же ты нашёл?»
\par
«Извини, Элисон. Я не хотел тебя нервировать. Здесь есть немного об Ури Геллере. Судя по всему, десятого сентября тысяча девятьсот семьдесят четвёртого года, он провёл эксперимент, в ходе которого исчезла половина куска ванадиевой фольги. Фольга была заключена в капсулу, которая подскакивала словно прыгающая фасолина. Старому Ури Геллеру фактически удалось разрушить молекулярную структуру ванадия. Мы без проблем можем сделать то же самое. Также здесь имеются записи о людях, которые заставляли различные предметы проходить сквозь стены…»
\par
На какое-то время он снова погрузился в чтение.
\par
«Вот в это я не верю — здесь есть материал о том, что психические силы — это, по сути, форма активности полтергейста, которая находится под контролем. В общем, что вы думаете?»
\par
«Прекрасно, Робин! Благодаря этому и моей прежней идее — ну, вы знаете: о том, чтобы сделать что-то, способное вызвать сбой в работе компьютеров — мы сможем придумать нечто такое, что должно сработать.»
\par
«Да, Робин. Элисон предложила хорошую мысль. Есть только одна проблема, которую я предвижу — мы не знаем, какую из частей компьютера нам придётся вывести из строя.»
\par
«Может быть, это не будет слишком сложно. Всё, что нам придётся сделать — это составить несколько отдельных запросов о принципах работы компьютеров на факультете компьютерных наук в нашем университете. Там могут знать, что заставит компьютер выйти из строя. Видишь, Мэтью? Не расстраивайся и просто немного верь в нас. Это сработает, не волнуйся.»
\par
«Здорово, что вы говорите мне не волноваться. Однако, если вы ещё не забыли, нас не существует в разрушенном Саутгемптоне будущего. Наши жизни зависят от того, насколько правильно мы всё сделаем.»
\par
Гарри не подумал об этом, и его лицо внезапно поникло. Всё это время он рассматривал их задачу, как очередной эксперимент или тест, который нужно выполнить, чтобы получить за него оценку.
\par
«Слушайте, вы все! Не грустите! Помните, у нас всё получится, в этом нет никаких сомнений. Нужно только тщательно всё спланировать, и всё получится. Поверьте мне… Или никто из вас больше не верит в то, что я могу сделать с помощью своего разума? Пойдёмте, уже поздно. Включим телевизор и посмотрим какой-нибудь фильм. Сегодня Страстная пятница — должно быть что-то приличное.»\\
\par
Робин включил телевизор и они посмотрели фильм «Краденый камень», после чего отправились спать. Их спальные места остались прежними, как и в предыдущую ночь. Элисон была единственной, кому понравился фильм — возможно потому, что в нём снимался Роберт Редфорд. Как бы то ни было, она настояла на том, чтобы все остальные сидели рядом с ней и досматривали фильм до конца.\\
\par
На следующий день Робин проснулся рано и, взяв с собой Симпсона, отправился в магазин. Он купил много мяса и овощей, и на этот раз позаботился о том, чтобы купить достаточно банок собачьего корма.
\par
К тому времени, когда он вернулся домой, Элисон уже была на кухне и готовила тосты и чай. Хотя никто из них этого не знал, день должен был стать для них удачным.\\
\par
Они продолжили внимательно просматривать файлы, как это было последние несколько дней. Внезапно Робин отбросил папку и записи, которые сделал.

\chapter{ОПАСЕНИЯ}
\noindent\par«Б{\scriptsizeОЖЕ, КАК МЫ БЫЛИ ГЛУПЫ!» - ВОСКЛИКНУЛ РОБИН},
\par
«Прошло столько времени, а мы даже не думали об этом!»
\par
«Робин, о чем?»
\par
«Разве ты не видишь, Мэтью? Здесь имеется примечание, которое профессор оставил после рифм, написанных Нострадамусом. Пророк утверждал, что Третья мировая война начнётся в августе тысяча девятьсот восемьдесят седьмого, но мы знаем, что на самом деле бомбы сбросят в апреле восемьдесят восьмого.»
\par
«Очевидно, он имел в виду, что в августе начнётся словесное противостояние государств. Если бы мы остановили события, которые привели к его началу, то у нас не возникло бы проблемы предотвращения ядерной войны.»
\par
«Здорово. Вы говорите, что мы должны остановить события, которые приведут к началу войны, но мы не узнаем, что это за события, до пятого апреля восемьдесят восьмого года, а к тому времени будет уже слишком поздно.»
\par
«Мэтью, подумай об этом. Русские блокируют Ормузский пролив. Чтобы сделать это, иранцы должны дать русским разрешение на передвижение их войск по территории Ирана. Всё, что от нас требуется — это убедиться в том, что такое разрешение никогда не будет дано. Теперь понятно?»
\par
«Неплохо, Робин! Совсем неплохо! Возможно, в этом что-то есть. Итак, есть ли что-нибудь ещё, что мы могли бы сделать, или что-то упущенное нами из виду?» - спросил Гарри, подперев голову руками и размышляя.
\par
Свежий взгляд Робина неожиданно открыл перед ним множество возможных решений их задачи, и его воображение разыгралось.
\par
«Да, Гарри. Кое-что действительно приходит на ум. Это было сказано тобой. В одной из папок, которые ты просмотрел, были записи о других студентах, с которыми профессор работал до нас. Ты не мог бы рассказать об этом подробнее? Не знаю почему, но я уверен — мы что-то пропустили.»
\par
«Ладно, Робин, как скажешь. Но лично я не вижу причин, по которым нас должно интересовать всё это.»
\par
Пока Гарри искал информацию, Мэтью и Элисон скрылись на кухне, чтобы начать готовить обед.
\par
В кои-то веки обед не попал в категорию "поесть, пока не стемнело" — в этот раз он пришёлся на более традиционное время — на половину второго. Мэтью был счастлив, что им не придётся снова есть яйца, вместо которых они попотчуют себя настоящей, "правильной" пищей. Запах жарящихся на гриле бараньих отбивных пробудил в нём аппетит. Элисон занималась салатом, не забывая присматривать за вафлями из картофеля.\\
\par
Робин взял в руки подробные записи, которые сделал профессор о своих прежних учениках. Он просмотрел их, тщательно проверяя, не обладал ли кто-нибудь из студентов какими-либо исключительными способностями. Робин просмотрел восемь разных досье, но не нашёл ничего, что могло бы привлечь его внимание — ни у одного из них, похоже, не было никаких особых талантов.
\par
Наконец, Робин нашёл студента, чьи данные показались ему интересными. Мартин Гау — он работал у профессора примерно четыре года назад, и в то время ему было двадцать лет. Он подавал большие надежды, обладая способностью гнуть металл, слегка поглаживая его. Однако это была одна из особенностей Мэтью, и когда он делал то же самое, ему не требовалось прикасаться к металлу.
\par
Нет, Робин искал кого-то, кто мог управлять силами своего разума не хуже его самого. Для того, что он задумал, ему нужен был человек с удивительными способностями…
\par
Его размышления были прерваны окликом Элисон из кухни:
\par
«Робин, обед готов! Иди скорее, он на столе — не дай ему остыть!»
\par
Аромат приготовленной пищи добрался до гостиной, и он был невероятно соблазнительным. Живот подсказывал мальчику, что нужно немедленно бросить все дела и отправиться к его источнику.\\
\par
«Робин, ты нашёл что-нибудь интересное? Какими были его прежние ученики? Был ли кто-нибудь из них так же хорош, как мы? Вообще-то, это был глупый вопрос — с тобой не сможет сравниться никто.»
\par
«Спасибо за комплимент, Элисон. Нет, пока что не отыскалось ничего, что могло бы нам хоть чем-то помочь.»\\
\par
Вскоре они закончили есть, и Робин вернулся к прерванному просмотру записей о прежних студентах. В гостиную зашёл Симпсон и уселся рядом с ним на пол. Он заглянул мальчику через плечо, изо всех сил стараясь привлечь его внимание.
\par
«А-а-а, я опять забыл о тебе» - произнёс Робин, рассеянно поглаживая собаку.
\par
Внезапно он вскочил и во весь голос закричал остальным:
\par
«Нашёл! Нашёл!»
\par
Покинув кухню, они вбежали в гостиную.
\par
«Что за крики? Почему ты кричишь?»
\par
«Элисон, это здесь. Это именно то, что я пытался найти!»
\par
«Не нужно трясти толстенной пачкой бумаги перед нашими лицами — это бесполезно — так мы не сможем прочитать ничего!» - нетерпеливо сказал Мэтью.
\par
«Кого ты нашёл? У него или неё есть имя?»
\par
«Дай мне возможность ответить, Гарри. Из-за того, что вы закидали меня вопросами, у меня не остаётся ни единого шанса вставить хотя бы слово.»
\par
Робин сверился с бумагами.
\par
«Её зовут Хелен Паркер. Сейчас ей должно быть около тринадцати лет. Она работала с профессором приблизительно пять лет назад, но пробыла здесь всего один год.»
\par
Профессор отправил её обратно к родителям, в Нью-Йорк, но они продолжали поддерживать связь, посылая друг другу письма. Здесь их целые стопки! Судя по содержанию последнего письма, она была самой молодой из тех, кто когда-либо поступал в один из лучших университетов искусств Соединённых Штатов.
\par
Робин снова заглянул в свои бумаги.
\par
«Послушайте! Похоже, что она обрела свои способности после удара электричеством, когда была намного младше — в возрасте примерно четырёх лет. Её родители утверждали, что после того несчастного случая она могла рисовать очень точные и подробные зарисовки тех мест, в которых никогда не была. Профессор Фергер пришёл к выводу, что, возможно, когда её ударило током, это было равносильно перегоранию предохранителя. В свою очередь, это сделало её полностью восприимчивой к силам Земли. Так называемое "перегорание" могло либо полностью лишить её психических способностей, либо наделить её огромной психической мощью, гораздо большей, чем мы можем вообразить!»
\par
Остальные увлечённо слушали Робина.
\par
«Это было всего лишь Началом. То, что следует дальше, простите за выражение, выносит мой мозг!»
\par
Робин пролистал ещё несколько страниц.
\par
«Если бы обстоятельства не заставили профессора отправить Хелен обратно, то неизвестно, на что бы она была способна сейчас! Хотя, судя по письмам, которые она писала профессору, они продолжали экспериментировать…»
\par
Он сделал паузу, задумавшись на мгновение.
\par
«Интересно, знает ли она, что профессор совершил самоубийство.»
\par
Внезапно, его глаза загорелись, и он задрожал от возбуждения, дико озираясь по сторонам. Было ясно, что он не высматривает какой-то определённый предмет, а что-то представляет себе.
\par
«Робин! Робин! Ты отгораживаешься от нас! Помни, мы должны работать одной командой! Что происходит в твоей голове?»
\par
«Да, Робин — Гарри прав. Ты поставил барьер, не позволяя нам узнать, о чём думаешь. Никто из нас так не поступает! Мы всегда абсолютно честны с тобой.»
\par
Робин повернулся лицом к Мэтью. Его глаза выражали страшный гнев — в них бушевало пламя. Крепко сжав кулаки, от стал отвечать на обвинения Гарри и Мэтью, и его голос дрожал от гнева:
\par
«Я не делал ничего такого! Вы сами виноваты, что не всегда можете узнать, что у меня на уме. Вы оба слишком увлечены соревнованием друг с другом, пытаясь завоевать расположение Элисон, и позволяете своим чувствам взять верх над работой. Как вы можете рассчитывать на то, что сосредоточите свои мысли на чём-то другом?»
\par
Гарри собрался остановить его, но Робин оборвал его на полуслове:
\par
«Заткнись, Гарри! Послушай хотя бы раз, сосредоточь свой слабенький ум на том, что я пытаюсь объяснить, а не на том, что хочешь сказать ты сам.»
\par
Робин спешно оглянулся, чтобы посмотреть, как отреагировал Мэтью на эту вспышку, но тот, как всегда, не показывал своих чувств — его лицо не выражало абсолютно ничего.
\par
«Всё, что мне нужно от вас сейчас — это полная преданность и благожелательность, пока проект не будет завершён. После этого вы можете отдохнуть и делать всё, что захотите в своей личной жизни.»
\par
Робин осмотрел их всех, а затем продолжил — уже спокойнее:
\par
Я сожалею об этом приступе гнева. Но вы должны попытаться понять, в какой ситуации мы оказались. Я не должен был злиться на вас, но в последнее время я чувствую сильное давление. Поэтому, пожалуйста, если вы хотите участвовать в этом и помогать, то потерпите меня, и не создавайте лишних, ненужных проблем.
\par
Робин снова сделал паузу.
\par
«Итак… Мэтью, Гарри… Никаких обид, всё забыто?»\\
\par
Робин замолчал. Именно бесстрастное лицо Мэтью стало причиной внезапной перемены его настроения.
\par
Мэтью был намного сильнее Гарри и, если бы его подтолкнули достаточно далеко, он бы уступил и позволил Робину делать то, что тот хотел. Гарри, будучи довольно слабым персонажем, легко поддающимся влиянию со стороны Мэтью, вполне мог бы последовать его примеру.
\par
Не смотря на то, что у Робина психических способностей было больше, чем у всех остальных вместе взятых, они понадобятся ему для того, что он задумал. Кроме того, он был отдан под их опеку — меньше всего ему хотелось, чтобы они его бросили. В своём письме профессор предупредил его, чтобы он был терпелив с ними.
\par
«Ладно, Робин. Я прощаю тебя. Только не думай, что я буду снова терпеть твои истерики. Ты не единственный, кто чувствует давление. Из твоего путешествия в будущее следует, что нас убьют. Поэтому ты должен понимать, что этот, как ты выразился, "проект" очень важен для нас. От успеха в нём зависят наши жизни.»
\par
«Прости, Мэтью, но спасибо за то, что на этот раз ты понял. Обещаю постараться больше не выходить из себя.»
\par
К этому моменту огонь в глазах Робина окончательно погас. В них было было только спокойствие.
\par
«О чём ты тогда задумался, Робин?» - спросила Элисон, пытаясь сменить тему для разговора. Всё это время она хранила молчание, не вмешиваясь в их спор. Давным-давно она пообещала профессору, что позаботится о Робине и не бросит его, что бы ни случилось.
\par
«Я не уверен в в том, как мы на самом деле собираемся предотвратить события, которые станут причиной войны» - задумчиво произнёс Робин,
\par
«Но интуиция подсказывает мне — что бы мы в конечном итоге ни решили сделать, нам понадобится кто-то другой, обладающий такими же способностями, как и мы сами.»
\par
«Скажи, чего же ты хочешь от Хелен — ведь так, кажется, её звали?» - спросила Элисон,
\par
«Ситуация изменится, если в команде появится ещё одна девушка.»
\par
«Да, её зовут Хелен…  Хелен Паркер…»
\par
Робин внезапно повернулся к Элисон:
\par
«Господи! Это пришло мне в голову только что. Подождите, я покажу вам! Посмотрите на эту фотографию в папке — она была сделана, когда девочке было восемь лет. А теперь посмотрите на рисунок, который сделал я. Видите? Это она!»
\par
«Робин, мне не хочется разрушать твои иллюзии, но я готов поспорить на что угодно, что его нарисовал не ты — это сделала Хелен.»
\par
«Что ты имеешь в виду, Гарри?»
\par
«Это была твоя рука, но она действовала в соответствии с указаниями Хелен. Её разум заставил тебя сделать этот рисунок.»
\par
«Если ты прав, значит ту сцену с катастрофой нарисовал не я, а Хелен. Должно быть, она рисовала эту картину в то же время, что и я. И тогда Мэтью тоже прав: если это была Хелен, значит действие картины, которую я нарисовал, на самом деле происходит в Нью-Йорке, потому что она живёт в Нью-Йорке.»
\par
«Подумать только — твои руки служили инструментом общения» - сказала Элисон, ошеломлённая внезапной догадкой,
\par
«Вероятно, она посылала что-то вроде сигнала бедствия, надеясь, что кто-нибудь его получит.»\\
\par
Элисон была очень чувствительной и, учитывая ранний период её детства, было неудивительно, что она расстраивалась, видя как ссорятся люди — особенно те, кого она любила.
\par
Внезапно она почувствовала опасность со стороны Хелен. Элисон знала, что Робин никогда не любил преувеличивать. Он был в восторге от того, что нашёл её досье, и был уверен, что она обладает такой же силой, как он сам. Если это действительно так, то контролировать её будет очень сложно. Хелен могла бы использовать их, чтобы делать всё, что заблагорассудится — ей удалось управлять Робином, так что же помешает ей проделать это над остальными?
\par
«Элисон! Элисон!»
\par
Она не замечала, как остальные пытались привлечь её внимание. Мэтью помахал рукой перед её лицом — она не моргнула глазом и продолжила смотреть прямо перед собой, как будто в поле её зрения ничего не было.
\par
«Гарри, помоги мне усадить её на стул — я не хочу, чтобы она упала.»
\par
Движение и лёгкая встряска наконец-то вернули её обратно в реальный мир.
\par
«Элисон, что это было?»
\par
«Ох, ничего.»
\par
Всё ещё находясь в лёгком оцепенении, она повернулась к Мэтью. Глядя на неё можно было подумать, что её поразил этот вопрос. На краткое мгновение она не восприняла его беспокойство как невинное, хотя ему этого хотелось. И всё же она почувствовала угрозу.
\par
«Извини, что спросил! Я всего лишь беспокоился. Я не думал, что мой интерес к тому, как ты себя чувствуешь, может тебя обидеть. Не стоило так отвечать. Господи, это должно быть заразно. Сначала Робин, а вот теперь — твоя очередь.
\par
Он повернулся к Гарри и с сарказмом продолжил:
\par
«Я думаю, что для нас будет лучше хранить молчание. Всякий раз, когда мы говорим, нам в конечном итоге откусывают головы.»
\par
«Прости, Мэтью!» - сказала Элисон,
\par
«Я не хотела, чтобы это выглядело так, будто я тебе нагрубила.»
\par
«Если с тобой всё в порядке, значит всё остальное неважно. Ты уверена, что у тебя всё хорошо? Скажи правду.»
\par
«Хорошо… если ты уверен, что хочешь знать…»
\par
Мэтью кивнул в ответ.
\par
«Это глупо, но я подумала, что если Хелен может контролировать Робина, заставляя его рисовать эти картинки — то что мешает ей сделать то же самое с нами. Или что-нибудь похуже.»
\par
Она покачала головой:
\par
«Видишь? Я же говорила, что это глупо!»
\par
«Нет, Элисон» - произнёс Гарри,
\par
«Это не было глупо. Конечно, опасность существует, но нам придётся воспользоваться этим шансом.»
\par
«Гарри, я не думаю, что она действительно будет использовать свой разум, чтобы попытаться контролировать нас» - сказал Робин,
\par
«Вспомните — у неё должны быть такие же видения, как и у меня, раз она нарисовала эту картину. Я думаю, что она была испугана и искала кого-нибудь, кто мог бы помочь. Я очень сомневаюсь, что мы вообще рискуем.»
\par
«Я согласен с Робином» - произнёс Мэтью. Он повернулся к мальчику:
\par
«Как ты предполагаешь связаться с Хелен?»
\par
«Мэтью, ты же не думаешь, что я буду оплачивать телефонный счёт? Я обещаю тебе, что это не будет происходить слишком часто. Это нужно лишь для того, чтобы я мог ей представиться, а после мы сможем общаться друг с другом при помощи телепатии. Не волнуйся и попробуй довериться мне.
\par
«Ну… Как долго ты собираешься общаться с ней во время этого первого телефонного разговора, Робин?»
\par
«Не переживай, Мэтью. Я думаю, что она каким-то образом знает, кто я. На самом деле это своего рода звонок-подтверждение, который даст ей понять, что я получил её сообщения.»
\par
Робин пролистал папку и письма, ища телефонный номер. Он помнил, что где-то уже видел его, но перерыв кипу бумаг, не мог вспомнить, где именно.
\par
«Вот он! Я нашёл его!» - воскликнул Робин, размахивая листком бумаги перед остальными студентами.
\par
«А что ты собираешься ей сказать?» - спросила Элисон,
\par
«Вдруг она ничего о тебе не знает? Возможно, она не осознавала, что кто-то увидит эти картины, пока она их рисует. Вспомните, Хелен — студентка художественного университета. Вероятно, они были частью её курсовой работы. Она может не понимать, почему нарисовала картину Нью-Йорка после катастрофы.»
\par
«Это шанс, которым мне придётся воспользоваться. Но в любом случае я всего лишь хочу, чтобы она обо мне узнала. После этого я расскажу ей, что мы задумали — телепатически. Боже, мне придётся сообщить ей о смерти профессора. Интересно, как она это воспримет. Она не могла быть слишком сильно эмоционально к нему привязана — у неё всё ещё есть родители.»
\par
«Какая разница во времени между нами и Нью-Йорком, Мэтью?»
\par
«Приблизительно пять часов.»
\par
«Что ж, Робин, сейчас не самое подходящее время, чтобы звонить» - заметила Элисон,
\par
«Там сейчас двенадцать — она, наверное, обедает. Я бы подождала хотя бы два часа.»
\par
«Я даже не подумал о том, что есть разница.»
\par
«Ну, Робин. Мне только сейчас пришло в голову, что когда ты рисовал её портрет, было около двенадцати часов ночи, а последний рисунок, должно быть, сделан утром.»
\par
«Ну, в любом случае, это было то, чего я не учёл. Но я прислушаюсь к твоему совету и не стану звонить ей пару часов. Я просто пойду, покормлю Симпсона и выпущу его в сад.»\\
\par	
Как только Мэтью убедился, что Робин занят на кухне своим питомцем, он заговорил с остальными серьёзным тоном, внимательно прислушиваясь, не возвращается ли мальчик обратно:
\par
«Он что-то задумал, но не хочет об этом говорить. Мы должны как-то заставить его рассказать нам правду. Боже, он отлично притворяется — на самом деле он знает, как собирается предотвратить катастрофу. Я в этом не сомневаюсь!»
\par
«Мэтью, ты не можешь быть полностью в этом уверен. Он знает, что рано или поздно ему придётся нам рассказать — ни за что на свете ему не сделать ничего подобного в одиночку!»
\par
«Разве вы не заметили, как изменилось его отношение? С тех пор как ему пришла в голову эта идея, он стал сосредоточенно и гораздо спокойнее относиться ко всему этому делу. Подумайте — вы же видели, насколько велика его власть над своим разумом, ему незачем посвящать нас в свои планы! Если Хелен обладает такими силами, как он утверждает, и ему удастся убедить её в том, что всё, что он хочет сделать, к лучшему… Ну, мы не можем полностью доверять ему и должны внимательно за ним следить.»
\par
«Мэтью, он слишком молод и всё ещё нуждается в наших советах. А теперь расслабься, дай ему время. В чём-то он похож на тебя — ты никогда не любил озвучивать свои идеи, если не был уверен в том, что они верны. Он просто осторожничает. Как и любой другой, он не хочет, чтобы его закидали тухлыми яйцами. Успех этого проекта — самое главное, что есть в его жизни. Он старается изо всех сил, делая это для профессора.»
\par
«Элисон, ему нельзя доверять. Его не волнует, как он добивается своего, лишь бы это ему удавалось. Он убил твоего отца и своих родителей — в одно мгновение ока. Его нужно держать под строгим контролем. Он верит, что цель оправдывает средства… Хватит говорить — он возвращается!»

\chapter{РЕШЕНИЯ}
\noindent\parД{\scriptsizeЕНЬ БЫЛ НЕ СЛИШКОМ ПЛОХИМ, И БИ-БИ-СИ В КОИ-ТО ВЕКИ УДАЛОСЬ СДЕЛАТЬ ПРАВИЛЬНЫЙ} прогноз погоды. Примерно до девяти утра шёл сильный дождь. После прошедшего ливня небо прояснилось и на нём засияло солнце. Тепло от его лучей растекалось равномерно, словно от одеяла, накрывшего собой весь город.
\par
Группка молодых людей совершила ошибку, отправившись за покупками в центр города. Они полагали, что в среду большая часть людей будет работать. Но нет — здесь повсюду бурлила жизнь. Их взорам открылось море недружелюбных лиц, толкающихся и распихивающих друг друга локтями.\\
\par
Прогулка получилась катастрофической. Гарри и Робин никак не могли перестать спорить, и это привело к тому, что в их разговор вмешалась Элисон, материнский инстинкт которой подсказывал, что нужно защитить мальчика. Мэтью повёл себя мудро и молчал.
\par
На обратном пути домой не было произнесено ни одного слова. Со лба Гарри стекали капельки пота, а его руки были крепко скрещены на груди.
\par
«Гарри, перестань дуться. Так капризничают только дети.»
\par
«Я не дуюсь, Элисон. В любом случае, это касается только меня и Робина — ты здесь не при чём.»
\par
«Ладно-ладно, утешься. Детское поведение, которое ты сейчас демонстрируешь, не вредит никому, кроме тебя самого.»
\par
Робин остановился возле газетного киоска и купил мороженое для всех, кроме Гарри. Делая вид, что мороженое его совсем не интересует, Гарри горделиво задрал кверху нос. На самом деле ему было так же жарко, как и остальным, но он не мог заставить себя побороть гордость и извиниться за своё глупое поведение. Его руки опустились вниз и он засунул большие пальцы в карманы штанов. Всю оставшуюся часть пути он вяло волочил ноги.
\par
Робин был раздосадован тем, как тот вёл себя. Гарри не вызывал в нём сочувствия — наоборот, он выставлял себя круглым дураком. Но профессор Фергер дал ему указание, чтобы он не позволял разъединиться их группе. Конечно, так должно было быть лишь до тех пор, пока Робин не станет достаточно сильным, чтобы суметь позаботиться о себе…\\
\par
«Гарри, прости — это моя вина. Наверное, во всём виновата погода — из-за этой проклятой жары я начинаю выходить из себя. Может быть, заключим перемирие?»
\par
Робин протянул свою руку Гарри и тот пожал её после некоторой заминки.
\par
«Я прощаю тебя, но без дальнейших обид, ладно?»
\par
Уступка Робина заставила Гарри почувствовать себя нехорошо. Только сейчас до него дошло, что он вёл себя как капризный ребёнок, и что на самом деле именно Робин вышел победителем из сложившейся ситуации.
\par
Наступило два часа дня и все они страшно проголодались.
\par
«Разве не сегодня… Хелен должна была вернуться из Парижа? Ведь так сказали тебе её родители, Робин?»
\par
«Спасибо, что напомнил, Гарри. Я совсем забыл об этом.»
\par
«На самом деле, мне кажется, что я должен был позвонить ей ещё вчера. Пойду-ка я займусь Симпсоном, а затем сделаю этот звонок.»
\par
«Всё в порядке, я его покормлю. А ты иди и поговори с Хелен.»\\
\par
«Это Хелен?»
\par
«Нет, я её отец. А кто вы?»
\par
«Я — Робин.»
\par
«Ах, британец… Сейчас я позову её.»
\par
Отец Хелен положил трубку на стол, и Робин смог расслышать его рокочущий голос:
\par
«Хелен, тот мальчик, о котором я тебе говорил… Он на линии.»\\
\par
«Привет, это Хелен. Ты — Робин, верно? Мой папа сказал, что ты пытался до меня дозвониться несколько дней назад.»
\par
«Да, я — Робин.»
\par
«Почему ты звонишь и откуда ты узнал моё имя?»
\par
«Тебе о чём-нибудь говорит имя Джозеф Фергер?»
\par
Внезапно, её защита резко обострилась:
\par
«Что ты знаешь о Джозефе?»
\par
Никто, кроме родителей и профессора Фергера не знал о том, где она находилась.
\par
«Профессор Фергер заботился обо мне точно так же, как о тебе» - ответил Робин,
\par
«У нас много общего. Я могу объяснить всё гораздо лучше, но не словами и не по телефону, а мысленно. Ты понимаешь меня? Не отключай свои мысли после того, как я положу трубку. Мне нужна твоя помощь. Если не для меня, то хотя бы для Джозефа.»
\par
«Что это значит — "для Джозефа"? Я ему нужна? Чего ты от меня хочешь?»
\par
«Хелен, успокойся — скоро я тебе всё объясню. А сейчас положи трубку на место, иди в спальню и ложись на кровать. Расслабься, и тогда ты сможешь услышать меня. А пока что прощай!»
\par
«Хорошо, но…»
\par
Потерявшая самообладание Хелен не успела закончить фразы — Робин уже положил трубку.\\
\par
Она сидела на лестнице, озадаченная загадочным телефонным звонком. Подперев голову своими бледными тонкими руками, она размышляла над тем, что услышала, и не могла понять, чего хочет от неё этот мальчик. Ни в одном своём письме к ней профессор Фергер ни разу не упоминал о Робине. Но, вернувшись в мыслях к их содержанию, она смогла припомнить, что профессор упоминал о том, как он восхищён одним маленьким мальчиком, находившемся под его присмотром. Однако он ни разу не назвал имени мальчика, о котором шла речь…
\par
Всё это не объясняло, почему профессор Фергер не связался с ней сам. Вместо этого она неожиданно получила телефонный звонок от абсолютно незнакомого ей человека.
\par
А ещё её беспокоил тот факт, что прошлое, которое она считала оставшимся далеко позади, теперь внезапно её настигло.
\par
После обстоятельств, из-за которых пришлось покинуть Великобританию, она уже не думала, что когда-нибудь ей снова придётся использовать свои психические силы.
\par
Внезапно она почувствовала себя ужасно. Она не могла определить причину этой всепоглощающей грусти, но чувствовала, как на неё обрушивается несчастье, а по телу прокатывается волна невыносимой тоски.
\par
Медленно, она встала и поплелась вверх по лестнице, цепко держась за перила, чтобы не упасть во время пути. Шок от того, что кто-то, совершенно незнакомый ей, произнёс вслух имя её любимого друга, вытягивал из неё все силы. Она начала чувствовать слабость и тошноту.\\
\par
Хотя она и не знала об этом, но этот первый контакт разумов — Робина и её собственного — должен был стать началом союза, который изменит их судьбы, а также судьбы людей всего мира.
\par
Она легла на кровать, вытянув ноги и сложив на груди руки, и сосредоточила всё внимание на голосе незнакомца — человека, который каким-то образом сумел заставить её подчиниться. Как будто он имел над ней какую-то непреодолимую власть. Никто, кроме профессора, не мог заставить её пользоваться скрытыми силами своего разума, пока она находилась в сознании. Но ему — Робину — это удалось.
\par
Хелен провела лилейно-белыми пальцами по своим длинным волосам мышиного цвета, откидывая их со лба и от глаз. Её веки неспешно закрылись, отделяя девушку от всего, что её отвлекало — спальни, висящих на стенах картин…\\
\par
«Хелен? Это я, Робин. Я не хочу, чтобы ты говорила, пока я сам не закончу всё то, что собираюсь тебе сказать.»
\par
«Хорошо, Робин…» - ответил слабый, растерянный голос.
\par
«Я знаю, что мой звонок стал для тебя потрясением, но поверь — я не стал бы просить о помощи, если бы это не было очень важно. Начну с самого начала — так тебе будет легче понять, что я скажу.»
\par
«Не так давно мой разум каким-то образом переместился в прошлое. Я понятия не имел о том, что произошло. Мне стало плохо, и профессору пришла в голову идея погрузить меня в гипноз, чтобы выяснить, что меня беспокоит. Именно тогда мы обнаружили, что я совершил путешествие во времени — в апрель тысяча девятьсот восемьдесят восьмого года. Там я стал свидетелем ужасов жизни в Британии после свершившейся катастрофы. Когда было собрано достаточно информации о том, как началась война и кто стал победителем в этом полномасштабном конфликте, мне едва удалось перенестись обратно во времени, в наши дни. Вскоре после моего возвращения сюда профессор попытался известить правительство о надвигающейся катастрофе. Он не рассказал им, как получил эту информацию — он защищал меня и моих коллег от эксплуатации со стороны политиков. Чтобы избежать преследования, он окончил жизнь самоубийством…»
\par
«Что ты сказал? Скажи мне, что этого не может быть!» - закричал её разум,
\par
«Я тебе не верю, Джозеф не может быть мёртв. О нет, пожалуйста, скажи, что это неправда. Пожалуйста!»
\par
Она начала беспокойно крутиться в постели. Сказанное стало для неё сильным эмоциональным ударом. Профессор относился к ней как к собственному ребёнку — она могла поверить словам Робина о том, что Джозеф защищал своих друзей от правительства. Он сделал для неё то же самое.
\par
«Хелен, это правда. Я сам нашёл его. Я знаю, что ты, должно быть, чувствуешь, я плакал несколько дней. Пожалуйста, поверь мне и доверься. Перед смертью профессор написал мне письмо. В нём он объяснил, почему решил уйти из жизни — это было ради меня. Он не хотел, чтобы кто-то знал о моих способностях, пока я не стану сильнее и старше. Но, в память о нём, он просил меня каким-нибудь образом остановить приближающуюся войну. Он сказал не позволять эмоциям — моим эмоциям — затуманивать моё видение того, что нужно сделать. Ты должна поступить точно так же. Джозеф сделал то, что он сделал, чтобы спасти нас и мир. Не позволяй его смерти пропасть даром. Теперь ты понимаешь? Хотя его нет с нами на земле физически, но всё равно он повсюду вокруг нас. В своём письме он написал, что навсегда останется с нами. Я ему верю, и ты тоже должна верить. У нас осталось слишком мало времени.»
\par
Хелен молчала, выкручивая от тоски свои руки. Ей казалось, что её сердце рвётся на части а внутри неё медленно вращаются острые кинжалы, причиняя нестерпимую боль. По одной стороне её лица скатилась слеза.
\par
«Робин, что мы будем делать? Ты уверен, что в тысяча девятьсот восемьдесят восьмом начнётся ядерная война?»
\par
«Да, Хелен, я уверен. Я бы не стал шутить о чём-то настолько важном. На какое-то время меня озадачило то, что я начал рисовать картины без всякой видимой причины. Мои друзья были сильно удивлены, увидев их — ведь я не художник. Я нарисовал твой портрет ещё до того, как узнал о твоём существовании, а ещё я нарисовал то, что можно описать только как сцену Нью-Йорка после катастрофы. Скажи, ты рисовала такие картины на прошлой неделе?»
\par
«Да, я нарисовала и то, и другое. Откуда ты об этом знаешь?»
\par
«Ну, к такому выводу пришёл один из моих коллег. Видишь ли, я просматривал файлы профессора, когда наткнулся на твоё имя. Когда я назвал остальным твоё имя и место, где ты живёшь, Элисон увидела связь. Она сказала это, потому что я назвал портрет именем Хелен, а также нарисовал пейзаж Нью-Йорка — в общем, мы догадались, что я, должно быть, скопировал твои рисунки. Это был единственный разумный вывод, который мы смогли придумать.»
\par
«О, теперь я поняла. Вероятно, произошло следующее: когда я рисовала эти картины для своей курсовой работы, моё подсознание посылало какой-то сигнал. Весь последний месяц мне снились кошмары. Я просыпалась по ночам с криком и плачем, холодный пот буквально ручьями стекал с моего тела.»
\par
«Что именно ты видела, Хелен?»
\par
«Это было ужасно. Это был город, пришедший в полный упадок — там бродили голодные люди. Некоторые лежали на обочинах улиц и умирали. Ко мне протягивались худые, немощные руки, прося еды… Я не хочу об этом говорить, меня это слишком расстраивает.»
\par
«Хорошо. Хелен. больше не нужно об этом. Я понимаю, через что ты проходишь.»
\par
«Робин, мне страшно. Как мы остановим войну? Это будет чертовски трудно, почти невозможно.»
\par
«Не волнуйся. Я уверен, мы что-нибудь придумаем.»
\par
«Я не могу не бояться. Я лучше умру, чем стану жить в том мире после ядерной войны.»
\par
«Ты не должна говорить так. Джозеф верил в нас. Он не стал бы совершать самоубийства, если бы думал, что мы не справимся. Мы не должны унывать. Помни, мы делаем это для него, а не для кого-то ещё.»
\par
«Хорошо, Робин. Я сделаю всё, что смогу.»
\par
Хелен остановилась. Она услышала голос матери, зовущий её издалека — та просила её спуститься вниз.
\par
«Что случилось, Хелен? Почему ты не говоришь со мной?»
\par
«Я слышу, как зовёт мама. Мне пора идти, но мы можем продолжить позже. Слушай, сегодня вечером, ближе к десяти, я лягу спать — тогда ты сможешь поговорить со мной и нас никто не побеспокоит.»
\par
«Хорошо, я поговорю с тобой в десять часов по вашему времени.»\\
\par
Робин вздохнул с облегчением, когда их общение прекратилось. Он очень нервничал не зная, как отреагирует Хелен.
\par
Он спустился вниз, чтобы увидеть своих друзей, которые терпеливо ждали в гостиной.
\par
«Как всё прошло, Робин? Она собирается нам помогать?» - с тревогой спросила Элисон. Остальные покивали головами, прося его не медлить с ответом.
\par
«Ну? Скажи нам, Робин, всё прошло хорошо?»
\par
«Ладно, притормози. Дай мне хотя бы присесть и тогда я тебе расскажу. Я всегда говорил, что ты слишком нетерпелив, Гарри.»
\par
Робин сел в кресло и устроился в нём поудобнее, прежде чем дать студентам полный отчёт.
\par
«Она была холодна, как лёд, когда мы только начали разговор. Я думаю, что она, должно быть, решила, что я работаю на правительство. Понимаете, я задел её за живое, когда упомянул имя профессора Фергера. Сначала она не хотела мне верить, когда я сказал, что Джозеф окончил жизнь самоубийством. Но спустя немного времени, когда Хелен разрушила свой барьер, она была вполне готова меня выслушать. Я договорился поговорить с ней подробнее сегодня вечером после десяти часов по её времени. Она заняла ту же позицию, что и мы: нам нужно добиться успеха в этом проекте хотя бы потому, что он так много значил бы для профессора.»
\par
«О, слава Богу за это. Значит, она хочет нам помочь?»
\par
«Да, Гарри, она будет сотрудничать с нами и постарается сделать всё, что сможет. И, поверь мне, она обладает феноменальными способностями. Даже из этого нашего короткого разговора мне стало ясно, какой силой она наделена.»
\par
«Робин, я всё ещё не понимаю, почему участие Хелен во всём этом так важно? Если ты что-то запланировал, то что именно?»
\par
	«Я пока что не знаю точно, что именно, Мэтью, но я уверен, что нам понадобятся её силы…»
\par
«Робин, прекрати. Этот ответ меня не устраивает — предложи другой. Очевидно, что ты чертовски хорошо представляешь, что, по твоему мнению, следует сделать. Так почему бы тебе не рассказать нам об этом? Ты боишься, что мы не согласимся с тобой и посоветуем придумать что-то ещё? Я думал, что мы будем работать одной командой. В команде каждый её член так доверяет другим участникам и уважает их, что информирует их о своих возможных идеях и планах. Насколько мы можем судить, возможно, в твоей голове вынашивается ошеломительная идея. Поэтому, пожалуйста, скажи нам о ней — кто знает, а вдруг мы сможем её уточнить или дополнить чем-то полезным.»
\par
Мэтью глубоко вздохнул. Он долго ждал, чтобы честно рассказать Робину, о чём именно думал. Он был прав в том, что сказал. Остальные студенты поняли, что происходит: с каждым новым днём Робин всё меньше и меньше рассказывал им о своих планах.
\par
«Ну, ты снова прав, Мэтью. Да, у меня есть некоторые представления о том, что нужно сделать, чтобы остановить войну. Но, раньше или позже, я рассказал бы вам обо всём, это было лишь вопросом времени…»
\par
«Не пытайся увильнуть! Болтать — бесполезно. Я не забуду о том, что ты собирался нам рассказать, Робин. Я не позволю тебе выкарабкаться из этой ситуации. Итак, всё, чего я хочу — это знать, что ты предлагаешь нам делать?»
\par
«Хорошо, хорошо! Если ты действительно хочешь знать, тогда слушай!»
\par
Мэтью выжидающе посмотрел на Робина.
\par
«Помните, я сказал вам о том, что вы смотрите на проблему с неправильной точки зрения?»
\par
«Да…» - хором ответили студенты.
\par
«Ну, насколько я вижу, корень проблемы в том, что Иран совершил ошибку, позволив России войти в свою страну. Да, всем хорошо известно, что иранский лидер сильно склоняется к коммунизму, и что его народ недоволен сложившейся ситуацией. Итак, поскольку глава Ирана скоро отправляется в Америку, чтобы встретиться с президентом, что может быть лучше, чем вызвать у него сердечный приступ или кровоизлияние в мозг? Вполне ожидаемо, что рано или поздно он умрёт, ведь он уже так стар. В любом случае, как я понимаю, народ Ирана воспримет это как божий дар. Следующий избранный лидер почти наверняка будет более правым и, следовательно, не позволит русским использовать его страну.»
\par
«Не такая уж плохая идея, Робин. Это означает, что для спасения всего человечества от надвигающейся катастрофы придётся пожертвовать жизнью всего одного человека.»
\par
Мэтью похвалил изобретательность Робина.
\par
«Есть одна тонкость, Робин…»
\par
«Да, Мэтью?»
\par
«Идея, как я уже сказал, блестящая. Но зачем тебе Хелен?»
\par
«Мне было интересно, когда ты спросишь об этом. Когда придёт время — ты знаешь, когда иранский лидер будет на пресс-конференции — вскоре после его прибытия мы сосредоточим все наши силы на Хелен, которая воспользуется ими, чтобы вызвать кровоизлияние или сердечный приступ.»
\par
«Да, но разве мы не можем сделать этого без неё?»
\par
«Я думал об этом. Но что, если расстояние от нас — здесь, в Англии — слишком велико до Нью-Йорка или Вашингтона? Что, если наши силы не принесут желаемого результата? Хелен живёт в Штатах и, несмотря на то, что между ней и главой Ирана будет довольно большая дистанция, это не будет такой уж большой проблемой. Поскольку наши силы будут поддерживать её, я не вижу причин, по которым мы должны потерпеть неудачу.»
\par
«Хорошо, Робин, с этим не поспоришь. Извини, что надоедал тебе своими вопросами, но я должен был это сделать. Всё, что могу сейчас сказать — я рад, что знаю о твоих планах. Нам осталось лишь рассказать о них Хелен и молиться о том, чтобы в назначенный день всё прошло гладко. У нас не так уж и много времени, верно? Иранский лидер должен отправиться в Штаты примерно через две недели.»
\par
«Спасибо за доверие, Мэтью. Оно значит для меня очень много.»\\
\par
Остаток дня они провели вне дома. Все были довольны тем, как идут дела — никто из них не опасался, что план Робина провалится.
\par
Они решили отправиться в парк Нью-Форест. В конце концов, день выдался тёплым, и солнечный свет заставил их всех почувствовать себя немного лучше.
\par
Впервые за несколько месяцев они почувствовали себя легко — как будто с их плеч наконец-то скинули тяжёлую ношу. Весь день прошёл приятно и спокойно, и среди них не случилось никаких разногласий.
\par
Бродя среди деревьев, они играли с Симпсоном, разделившись на пары.
\par
В целом, это была очень спокойная прогулка на фоне живописной природы…

\chapter{ПОСЛЕДСТВИЯ}
\begin{center}
ГЛАВА ИРАНА УМИРАЕТ, НЕ ДОБРАВШИСЬ ДО ФИНИША.\\
\end{center}
\par
\begin{center}
Автор статьи: ДЖИММИ СТРЭЙСОН, Вашингтон.\\
\end{center}
\par
П{\scriptsizeРОШЕДШИМ ДНЁМ ПЯТЬ МИЛЛИОНОВ ТЕЛЕЗРИТЕЛЕЙ ПО ВСЕМУ МИРУ СТАЛИ СВИДЕТЕЛЯМИ} внезапной смерти главы Ирана.
\par
Он и президент Соединённых штатов Штатов Америки должны были приступить к мирным переговорам на высшем уровне в надежде завершить текущий конфликт в районе Персидского залива.\\
\par
Иранский самолёт приземлился в Вашингтоне вчера вечером, в девять-тридцать. Всего двадцать минут спустя, когда иранский лидер и президент давали пресс-конференцию в прямом эфире, аятолла потерял сознание и умер в течение нескольких секунд. У него случилась остановка сердца.\\
\par
Члены иранской делегации признались, что их лидер периодически болел на протяжении длительного времени. По словам одного из её представителей, это было всего лишь вопросом времени, когда произойдёт нечто подобное.\\
\par
Президент Рональд Рейган в течении нескольких последних месяцев наращивал давление на западных союзников, чтобы они оказали поддержку США в их действиях по сопровождению и обеспечению безопасности поставок нефти в Персидском заливе. Всего два месяца назад президент Рейган усилил оперативную группу США на Ближнем Востоке ещё тремя военными кораблями, чтобы оказать помощь уже находящимся там шести военным кораблям.\\
\par
Внезапный кризис в Персидском заливе начался, когда иракская ракета "Экзосет" поразила американский военный корабль "Старк", в результате чего погибли тридцать семь моряков.\\
\par
В ходе этих переговоров выражалась надежда, что лидеры двух стран смогут прийти к какому-то соглашению относительно всеобщего беспокойства по поводу пополнения иранского арсенала ракетами "Шелкопряд". Ходят слухи, что эти ракеты обладают дальностью действия до шестидесяти миль, и в три раза большей мощностью, чем ракеты "Экзосет".\\
\par
Весь мир находится в ожидании, когда Иран выберет нового лидера. Однако хорошо известен тот факт, что в Иране существует множество радикальных религиозных группировок, и весьма вероятно, что их ожесточённая борьба за власть приведёт к кровопролитным последствиям.
\newpage
\pagenumbering{gobble}
\noindent
\textbf{Планируемые издания в "Abstract Concepts"}
\newline
\newline
\newline
\newline
\textbf{СПОКОЙНОЙ НОЧИ, ЖЕСТОКИЙ МИР}\\
\par
К 1988 году Нью-Йорк превратился в суровый полицейский штат. Ослабленная непрекращающимися бандитскими войнами и организованной преступностью полиция отступила, оставив закрытые для неё территории под контролем наркобаронов. Широко распространены инфекционные болезни, и некоторые районы города часто изолируют, чтобы контролировать эпидемию. Тем не менее, не смотря на это, жизнь жителей Манхэттена протекает нормально, насколько это возможно.
Джон Шульц — компьютерный оператор корпорации Рослин. Он живёт со своей девушкой, Вэл, в подвальной квартире района Чайнатаун.
Однажды ночью, после ссоры, Вэл уходит из дома и бесследно исчезает. Беспокойство Джона превращается в кошмар наяву, когда он узнаёт, что полиция подозревает его в "нечестной игре". Зная политику Министерства юстиции «Сначала стреляй, потом задавай вопросы», Джон скрывается.
Его беглые поиски Вэл вскрывают более глубокие преступления и коррупцию, чем он мог представить…
\newline
\newline
\newline
\textbf{ПАРИЖСКИЕ РЫЦАРИ}\\
\par
После Первой мировой войны, на фоне распада империй, мир охватила экономическая депрессия. Международная напряжённость ослабла, но никуда не исчезла, поскольку некоторые страны искали убежища в идеологиях фашизма и коммунизма.
Он стал зоной восстаний, гражданской войны… и шпионов!
История рассказывает о карьере агента французской секретной службы месье Филиппа Груши. Вернувшись после провальной миссии в Лондоне, Филипп получает строгий выговор от начальства. Хотя он уверен в том, что его предали, он не может предъявить никаких доказательств, оправдывающих свою неудачу. Последующий успех его очередной миссии жизненно важен для восстановления утраченного авторитета. Однако, когда он обнаруживает постоянно возрастающую опасность при выполнении текущих заданий, он начинает подозревать, что, возможно, попал в ловушку, и что предатель может быть ближе, чем он думал…

\end{document}
